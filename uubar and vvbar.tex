\documentclass[12pt]{article}

\usepackage{hyperref}
\usepackage{framed}

\usepackage{graphicx}
\usepackage{amsmath}
%\usepackage{mathtools}
\usepackage{amssymb}
\usepackage{slashed}
\usepackage{bm}
\usepackage{cite}
\usepackage{setspace}
\usepackage{bigints}

\setlength{\oddsidemargin}{0in}
\setlength{\textwidth}{6.5in}
\setlength{\topmargin}{0in}
\setlength{\textheight}{9in}
\voffset-1.5truecm

\def \beq{\begin{equation}}
\def \eeq{\end{equation}}
\def \bea{\begin{eqnarray}}
\def \eea{\end{eqnarray}}
\def \ba{\begin{array}}
\def \ea{\end{array}}
\def \beg{\begin{gather}}
\def \eeg{\end{gather}}
\def \bmat{\begin{matrix}}
\def \emat{\end{matrix}}
\def \bfra{\begin{framed}}
\def \efra{\end{framed}}

\def \({\left(}
\def \){\right)}
\def \[{\left[}
\def \]{\right]}
\def \<{\left\langle}
\def \>{\right\rangle}
\def \lp{\left|}
\def \rp{\right|}
\def \lb{\left\{}
\def \rb{\right\}}
\def \l.{\left.}
\def \r.{\right.}

\def \bma{\(\bmat}
\def \ema{\emat\)}

\def \Tr{{\rm Tr}}
\def \N{\rm N}
\def \hc{{\rm h.\ c.}}

\def \ph{\phantom}
\def \nn{\nonumber}
\def \nl{\nn \\}

\def \of{\frac{1}{4}}
\def \oth{\frac{1}{3}}
\def \hf{\frac{1}{2}}
\def \oet{\frac{1}{8}}
\def \s{\sqrt{2}}
\def \st{\sqrt{3}}
\def \sx{\sqrt{6}}

\def \cB{{\cal B}}
\def \cC{{\cal C}}
\def \cZ{{\cal Z}}
\def \cL{{\cal L}}
\def \cH{{\cal H}}
\def \cO{{\cal O}}
\def \cM{{\cal M}}
\def \cl{\ell}

\def \oq{\overline{q}}
\def \of{\overline{f}}
\def \os{\overline{s}}
\def \ob{\overline{b}}
\def \ou{\overline{u}}
\def \ov{\overline{v}}
\def \oBs{\overline{B^0_s}}
\def \otau{\overline{\tau}}
\def \omu{\overline{\mu}}
\def \onu{\overline{\nu}}
\def \osi{\overline{\si}}
\def \oell{\overline{\ell}}
\def \oL{\overline{L}}
\def \oQ{\overline{Q}}

\def \al{\alpha}
\def \be{\beta}
\def \ga{\gamma}
\def \Ga{\Gamma}
\def \de{\delta}
\def \ka{\kappa}
\def \De{\Delta}
\def \ep{\epsilon}
\def \tha{\theta}
\def \la{\lambda}
\def \La{\Lambda}
\def \si{\sigma}
\def \Si{\Sigma}
\def \Up{\Upsilon}
\def \om{\omega}
\def \1{1}%\mathbb{I}}

\def \pa{\partial}

\def \SM{{\rm SM}}
\def \elm{{\rm em}}
\def \NP{{\rm NP}}
\def \expt{{\rm expt}}
\def \eff{{\rm eff}}
\def \elm{{\rm em}}
\def \RFG{{\rm RFG}}
\def \CFG{{\rm CFG}}
\def \PT{{\rm PT}}
\def \hi{{\rm hi}}
\def \lo{{\rm lo}}

\def \Re{{\rm Re}}
\def \Im{{\rm Im}}

\def \keV{{\rm keV}}
\def \MeV{{\rm MeV}}
\def \GeV{{\rm GeV}}

\def \cH{{\cal H}}
\def \cI{{\cal I}}

\def \Rqp{R_{q'}}

\def\bwt{\begin{widetext}}
\def\ewt{\end{widetext}}

\def\vev#1{\langle {#1} \rangle}
\def\eq#1{eq.~(\ref{#1})}

\def \vp{{\bf p}}
\def \vq{{\bf q}}

% Start of document
% -----------------
\pagestyle{plain}

% Roman Section & Subsection numberings
% -------------------------------------
\renewcommand{\thesection}{\Roman{section}}
\renewcommand{\thesubsection}{\Roman{subsection}}

% Start of document
% -----------------
\pagestyle{plain}
\allowdisplaybreaks

\setlength{\abovecaptionskip}{0pt}
\setlength{\belowcaptionskip}{0pt}

\begin{document}

\begin{center}
\underline{\textbf{\Large $u\ou$ and $v\ov$}}
\end{center}

\section{$u\ou$}

\begin{enumerate}

$u$ is defined to be a $4\times1$ matrix for example: $\bma w\\x\\y\\z\ema$ \nl
Then, by definition, $\ou$ will be a $1\time4$ matrix becasue the definition of $\ou$ is $(u)^\dag\ga^0$\nl 
We seek to calculate $u\ou$, by inspection we can see that it will be a $4\times4$ matrix \nl 

If we let $u~=~ \bma w\\x\\y\\z\ema$ then we can calculate $\ou$\nl 
\bea \nonumber
\ou &=& u^\dag\ga^0 ~,~~ \nonumber \\ \nonumber
&=& \bma w^*&x^*& y^*& z^*\ema \ga^0 \nl \nonumber
&=& \bma w^*&x^*& y^*& z^*\ema \bma 1 & 0 & 0 & 0 \\0 & 1 & 0 & 0\\0 & 0 & -1 & 0\\0 & 0 & 0 & -1\ema  \nl \nonumber
&=& \bma w^*& x^*& -y^* & -z^*\ema \nl\nonumber
\eea \nonumber

Now we can calculate $u\ou$ \nl
\bea \nonumber
u\ou&=& \bma w\\x\\y\\z\ema\bma w^* & x^* & -y^* & -z^*\ema\\ \nonumber
&=& \bma ww^* & wx^* & -wy^* & -wz^*\\ xw^* & xx^* & -xy^* & -xz^*\\ yw^* & yx^* & -yy^* & yz^* \\ zw^* & zx^* & -zy^* & -zz^* \ema \nonumber
\eea \nonumber

When solving the Dirac equation, there are multiple solutions that can be found. We we look at two here and find the value of the solution multiplied by its Hermitian and $\ga^0$ \nl 
If we let $u= N\bma 1 \\0 \\ \frac{(p_z)}{E+m}\\ \frac{(p_x + ip_y)}{E+m}\ema$ then $\ou = N\bma 1&0& -\frac{(p_z)}{E+m} & -\frac{(p_x + ip_y)}{E+m}\ema$ \nl 
Where $N=\sqrt{E+m}$ which is the normalization constant, found by the identities of $u, v$ and $u^\dag u = 2E$ \nl
From this and as shown above, $u\ou = N^2\bma 1 & 0 & \frac{-p_z }{E+m}& \frac{-p_x+ip_z}{E+m}\\ 0 & 0& 0 & 0\\ \frac{p_z}{E+m} & 0 & \frac{-p_z^2}{(E+m)^2} & \frac{-p_zp_x+ip_zp_y}{(E+m)^2} \\ \frac{p_x+ip_y}{E+m} & 0 &  \frac{-p_zp_x-ip_zp_y}{(E+m)^2} & \frac{-p_x^2-p_y^2}{(E+m)^2} \ema$ \nl
When you multiply through by $N^2$ the matric becomes:\nl
$u\ou =\bma E+m & 0 & -p_z & -p_x+ip_z\\ 0 & 0& 0 & 0\\ p_z & 0 & \frac{-p_z^2}{E+m} & \frac{-p_zp_x+ip_zp_y}{E+m} \\ p_x+ip_y & 0 &  \frac{-p_zp_x-ip_zp_y}{E+m} & \frac{-p_x^2-p_y^2}{E+m} \ema$ \nl
\nl
This matrix is interesting because its trace is related to the product of $u$ and $\ou$\nl
Note: the equation $E^2 = m^2 + p^2$ is useful 
\bea \nonumber
Tr[u\ou]&=&  E+m + 0 +\frac{-p_z^2}{E+m}+\frac{-p_x^2-p_y^2}{E+m}\nl 
&=& \frac{(E+m)^2 - \mid p \mid^2}{E+m}\nl  
&=& \frac{(E+m)^2 + m^2 - E^2}{E+m}\nl 
&=& \frac{2Em +2m^2}{E+m}\nl 
&=& \frac{(E+m)(m+m)}{E+m}\nl 
&=& (m+m)\nl 
&=& 2m \nonumber
\eea \nonumber \nl
Similarly it can be shown that $\ou u = 2m$
\bea \nonumber
\ou u &=& N\bma 1&0& -\frac{(p_z)}{E+m} & -\frac{(p_x + ip_y)}{E+m}\ema N\bma 1 \\0 \\ \frac{(p_z)}{E+m}\\ \frac{(p_x + ip_y)}{E+m}\ema\nl
&=& N^2 (1 + 0 +-p_z^2+-p_x^2-p_y^2) \nl
&=& E+m +\frac{-p_z^2}{E+m}+\frac{-p_x^2-p_y^2}{E+m}\nl
&=& \frac{(E+m)^2 - \mid p \mid^2}{E+m}\nl \nonumber 
&=& \frac{(E+m)^2 + m^2 - E^2}{E+m}\nl
&=& \frac{2Em +2m^2}{E+m}\nl 
&=& \frac{(E+m)(m+m)}{E+m}\nl 
&=& (m+m)\nl 
&=& 2m \nonumber
\eea \nonumber
Thus, $Tr[u\ou] = \ou u$
\end{enumerate}
\end{document}
