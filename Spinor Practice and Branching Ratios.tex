\documentclass[12pt]{article}

\usepackage{hyperref}
\usepackage{framed}

\usepackage{graphicx}
\usepackage{amsmath}
%\usepackage{mathtools}
\usepackage{amssymb}
\usepackage{slashed}
\usepackage{bm}
\usepackage{cite}
\usepackage{setspace}
\usepackage{bigints}
\usepackage{color}

\setlength{\oddsidemargin}{0in}
\setlength{\textwidth}{6.5in}
\setlength{\topmargin}{0in}
\setlength{\textheight}{9in}
\voffset-1.5truecm

\def \beq{\begin{equation}}
\def \eeq{\end{equation}}
\def \bea{\begin{eqnarray}}
\def \eea{\end{eqnarray}}
\def \ba{\begin{array}}
\def \ea{\end{array}}
\def \beg{\begin{gather}}
\def \eeg{\end{gather}}
\def \bmat{\begin{matrix}}
\def \emat{\end{matrix}}
\def \bfra{\begin{framed}}
\def \efra{\end{framed}}

\def \({\left(}
\def \){\right)}
\def \[{\left[}
\def \]{\right]}
\def \<{\left\langle}
\def \>{\right\rangle}
\def \lp{\left|}
\def \rp{\right|}
\def \lb{\left\{}
\def \rb{\right\}}
\def \l.{\left.}
\def \r.{\right.}

\def \bma{\(\bmat}
\def \ema{\emat\)}

\def \Tr{{\rm Tr}}
\def \N{\rm N}
\def \hc{{\rm h.\ c.}}

\def \ph{\phantom}
\def \nn{\nonumber}
\def \nl{\nn \\}

\def \of{\frac{1}{4}}
\def \oth{\frac{1}{3}}
\def \hf{\frac{1}{2}}
\def \oet{\frac{1}{8}}
\def \s{\sqrt{2}}
\def \st{\sqrt{3}}
\def \sx{\sqrt{6}}

\def \cB{{\cal B}}
\def \cC{{\cal C}}
\def \cZ{{\cal Z}}
\def \cL{{\cal L}}
\def \cH{{\cal H}}
\def \cO{{\cal O}}
\def \cM{{\cal M}}
\def \cl{\ell}

\def \oq{\overline{q}}
\def \of{\overline{f}}
\def \os{\overline{s}}
\def \ob{\overline{b}}
\def \ou{\overline{u}}
\def \ov{\overline{v}}
\def \oBs{\overline{B^0_s}}
\def \otau{\overline{\tau}}
\def \omu{\overline{\mu}}
\def \onu{\overline{\nu}}
\def \osi{\overline{\si}}
\def \oell{\overline{\ell}}
\def \oL{\overline{L}}
\def \oQ{\overline{Q}}

\def \al{\alpha}
\def \be{\beta}
\def \ga{\gamma}
\def \Ga{\Gamma}
\def \de{\delta}
\def \ka{\kappa}
\def \De{\Delta}
\def \ep{\epsilon}
\def \tha{\theta}
\def \la{\lambda}
\def \La{\Lambda}
\def \si{\sigma}
\def \Si{\Sigma}
\def \Up{\Upsilon}
\def \om{\omega}
\def \1{1}%\mathbb{I}}

\def \pa{\partial}

\def \SM{{\rm SM}}
\def \elm{{\rm em}}
\def \NP{{\rm NP}}
\def \expt{{\rm expt}}
\def \eff{{\rm eff}}
\def \elm{{\rm em}}
\def \RFG{{\rm RFG}}
\def \CFG{{\rm CFG}}
\def \PT{{\rm PT}}
\def \hi{{\rm hi}}
\def \lo{{\rm lo}}

\def \Re{{\rm Re}}
\def \Im{{\rm Im}}

\def \keV{{\rm keV}}
\def \MeV{{\rm MeV}}
\def \GeV{{\rm GeV}}

\def \cH{{\cal H}}
\def \cI{{\cal I}}

\def \Rqp{R_{q'}}

\def\bwt{\begin{widetext}}
\def\ewt{\end{widetext}}

\def\vev#1{\langle {#1} \rangle}
\def\eq#1{eq.~(\ref{#1})}

\def \vp{{\bf p}}
\def \vq{{\bf q}}

\def \cre{\color{red}}
\def \cgr{\color{green}}

\usepackage{graphicx} %package to manage images
\graphicspath{ {images/} }

\usepackage[rightcaption]{sidecap}

\usepackage{wrapfig}

% Start of document
% -----------------
\pagestyle{plain}

% Roman Section & Subsection numberings
% -------------------------------------
\renewcommand{\thesection}{\Roman{section}}
\renewcommand{\thesubsection}{\Roman{subsection}}

% Start of document
% -----------------
\pagestyle{plain}
\allowdisplaybreaks

\setlength{\abovecaptionskip}{0pt}
\setlength{\belowcaptionskip}{0pt}

\begin{document}

\begin{center}
\underline{\textbf{\Large Practice with Spinors}}
\end{center}

\section{Algebra with Dirac $\ga$ matrices}

Notation:
\begin{enumerate} \itemsep=-15pt
\item $S= \ou u$\\
\item $P=\ou \ga^5 u$ \\
\item $V^\mu = \ou \ga^\mu u$ \\
\item $A^\mu = \ou \ga^\mu \ga^5 u $ \\
\item $T^{\mu\nu} = \ou \si^{\mu\nu} u$
\end{enumerate}
\bigskip

\begin{enumerate}

\item$(\ou_1\ga^\mu u_2)^*$ ~=~?

Note: $(\ga^0)^\dag~=~\ga^0$ and $(\ga^\mu)^\dag~=~ \ga^0\ga^\mu\ga^0$

$(\ou_1\ga^\mu u_2)$ is a $1\times1$ matrix. Therefore, its complex conjugate is the same as its Hermitian conjugate, i.e. if we call
$V^\mu = (\ou_1\ga^\mu u_2)$, then $(V^\mu)^* = (V^\mu)^\dag$. We can then express this quantity as follows:
% This is the first equation/set of equations
\bea
V^\mu &=& \ou_1\ga^\mu u_2 ~,~~ \\
\Rightarrow (V^\mu)^* &=& (V^\mu)^\dag ~,~~ \\
&=& (\ou_1\ga^\mu u_2)^\dag ~,~~ \\
&=& ((u_1)^\dag\ga^0\ga^\mu u_2)^\dag ~~~{\rm using}~(A\ldots Z)^\dag = Z^\dag\ldots A^\dag ~,~~ \\
&=& (u_2^\dag)(\ga^\mu)^\dag(\ga^0)^\dag(u_1)\\
&=& (u_2^\dag)\ga^0\ga^\mu\ga^0\ga^0(u_1)\\
&=& (u_2^\dag)\ga^0\ga^\mu(u_1)\\
&=& \ou_2\ga^\mu(u_1)
\eea
Therefore $(\ou_1\ga^\mu u_2)^*~=~\ou_2\ga^\mu u_1 $. To solve for $|V^\mu|^2$ we simply use $|V^\mu|^2 ~=~ \Tr[\ou_1 \ga^\mu u_2 \ou_2 \ga^\nu u_1]$.


\underline{Question:} \\
Why is $|V^\mu|^2 ~\ne~ \Tr[\ou_1 \ga^\mu u_2 \ou_2 \ga_\mu u_1]$? \\
Why is $|V^\mu|^2 ~=~ \Tr[\ou_1 \ga^\mu u_2 \ou_2 \ga^\nu u_1]$? \\ \\

You start with one index $\mu$. When you square why do you get two indices and not a sum over two of the same index? \\
When you square you have two indices because you must increase the number of components. The number of components when you square should go as $n^2$ not simply $n$. When you have one index, you restrict the number of components, because you have 4 components and not 16.

\bigskip

Note: ~$\Tr[\ga^\mu\ga^\nu]~=~4 g^{\mu\nu}$, $\Tr[\ga^\mu \ga^\nu \ga^\la \ga^\si]= 4(g^{\mu\nu}g^{\la\si} - g^{\mu\la} g^{\nu\si} + g^{\mu\si}g^{\nu\la})$, The trace over the product of an odd number of gamma matrices is zero.

\bea
|V^{\mu}|^2 &=& \Tr[\ou_1 \ga^\mu u_2 \ou_2 \ga^\nu u_1] \\
&=& \Tr[\ou_1 \ga^\mu (\slashed{p}_2+m) \ga^\nu u_1]\\
&=& \Tr[u_1\ou_1 \ga^\mu (\slashed{p}_2+m) \ga^\nu]\\
&=& \Tr[(\slashed{p}_1+m) \ga^\mu (\slashed{p}_2+m) \ga^\nu]\\
&=& \Tr[\slashed{p}_1\ga^\mu \slashed{p}_2 \ga^\nu] + m[\Tr(\ga^\mu \slashed{p}_1\ga^\nu) + \Tr(\ga^\mu \ga^\nu \slashed{p}_2)] + m^2\Tr[\ga^\mu \ga^\nu]\\
&=& \Tr[\slashed{p}_1\ga^\mu \slashed{p}_2 \ga^\nu] + m^2\Tr[\ga^\mu \ga^\nu]\\
&=& \Tr[(p_1)_\la\ga^\la \ga^\mu (p_2)_\si\ga^\si \ga^\nu  ] + 4m^2g^{\mu\nu}\\
&=& (p_1)_\la (p_2)_\si \Tr[\ga^\la \ga^\mu\ga^\si \ga^\nu  ] + 4m^2g^{\mu\nu}\\
&=& (p_1)_\la(p_2)_\si 4(g^{\mu\nu}g^{\la\si} - g^{\mu\la} g^{\nu\si} + g^{\mu\si}g^{\nu\la}) + 4m^2g^{\mu\nu}\\
&=& 4[p_1^\mu p_2^\nu - g^{\mu\nu}(p_1 \cdot p_2) + p_2^\mu p_1^\nu] + 4m^2g^{\mu\nu} 
\eea

\item $(\ou_1\ga^\mu \ga^5 u_2)^*$ is also a $1\times1$ Matrix so the same reasoning applies as above in 1. Note: $(\ga^5)^\dag = \ga^5$
We define: $A^\mu$ as $\ou_1\ga^\mu\ga^5 u_2$ thus:
% This is the second equation/set of equations
\bea
(A^\mu)^* &=& (A^\mu)^\dag \\
&=&(\ou_1\ga^\mu\ga^5 u_2)^\dag \\
&=& ((u_1)^\dag\ga^0\ga^\mu\ga^5 u_2)^\dag \\
&=& (u_2^\dag)(\ga^5)^\dag(\ga^\mu)^\dag(\ga^0)^\dag(u_1) \\
&=& (u_2^\dag)\ga^5\ga^0\ga^\mu\ga^0\ga^0 u_1 \\
&=& (u_2^\dag)\ga^5\ga^0\ga^\mu(1) u_1 \\
&=& -(u_2^\dag)\ga^0\ga^5\ga^\mu u_1 \\
&=& -\ou_2\ga^5\ga^\mu u_1 \\
&=& \ou_2\ga^\mu\ga^5 u_1
\eea
%
Therefore $(\ou_1\ga^\mu \ga^5 u_2)^*~=~\ou_2 \ga^\mu \ga^5 u_1$

We also are able to calculate $|A^\mu|^2$
\bea
|A^{\mu}|^2 &=& \Tr[(\ou_1\ga^\mu \ga^5 u_2)(\ou_2 \ga^\nu \ga^5 u_1)] \\
&=& \Tr[\ou_1\ga^\mu \ga^5 (\slashed{p}_2+m) \ga^\nu \ga^5 u_1]  \\
&=&\Tr[u_1\ou_1\ga^\mu \ga^5 (\slashed{p}_2+m) \ga^\nu \ga^5 ]\\
&=&\Tr[(\slashed{p}_1+m)\ga^\mu \ga^5 (\slashed{p}_2+m) \ga^\nu \ga^5 ]\\
&=& \Tr[\slashed{p}_1\ga^\mu\ga^5 \slashed{p}_2\ga^\nu \ga^5 + m( \slashed{p}_1\ga^\mu\ga^5\ga^\nu \ga^5 +\ga^\mu\ga^5 \slashed{p}_2\ga^\nu \ga^5) + m^2(\ga^\mu\ga^5\ga^\nu \ga^5)] \\
&=&\Tr[\slashed{p}_1\ga^\mu\ga^5 \slashed{p}_2\ga^\nu \ga^5 + m( \slashed{p}_1\ga^\mu(-\ga^5 \ga^5)\ga^\nu +\slashed{p}_2\ga^\mu(-\ga^5\ga^5) \ga^\nu ) + m^2(\ga^\mu\ga^5\ga^\nu \ga^5)] \nl
&=&\Tr[\slashed{p}_1\ga^\mu\ga^5 \slashed{p}_2\ga^\nu \ga^5 - m( \slashed{p}_1\ga^\mu\ga^\nu +\slashed{p}_2\ga^\mu\ga^\nu ) + m^2(\ga^\mu\ga^5\ga^\nu \ga^5)] \\
&=&\Tr[\slashed{p}_1\ga^\mu\ga^5 \slashed{p}_2\ga^\nu \ga^5] -mTr[ \slashed{p}_1\ga^\mu\ga^\nu]-mTr[ \slashed{p}_2\ga^\mu\ga^\nu]+ m^2(\ga^\mu\ga^5\ga^\nu \ga^5)] \\
&=&\Tr[\slashed{p}_1\ga^\mu\ga^5 \slashed{p}_2\ga^\nu \ga^5 + m^2(\ga^\mu\ga^5\ga^\nu \ga^5)] \\
&=& \Tr[(p_1)_\la\ga^\la\ga^\mu\ga^5 (p_2)_\si\ga^\si\ga^\nu \ga^5] + m^2\Tr[\ga^\mu\ga^5\ga^\nu \ga^5]\\
&=& (p_1)_\la(p_2)_\si \Tr[\ga^\la\ga^\mu\ga^5 \ga^\si\ga^\nu \ga^5] - m^2\Tr[\ga^\mu\ga^5\ga^5 \ga^\nu]\\
&=& (p_1)_\la(p_2)_\si \Tr[\ga^\la\ga^\mu\ga^5\ga^5 \ga^\si\ga^\nu] - m^2\Tr[\ga^\mu\ga^\nu]\\
&=& (p_1)_\la(p_2)_\si \Tr[\ga^\la\ga^\mu\ga^\si\ga^\nu] - m^2(g^{\mu\nu})\\
&=& (p_1)_\la(p_2)_\si 4(g^{\mu\nu}g^{\la\si} - g^{\mu\la} g^{\nu\si} + g^{\mu\si}g^{\nu\la}) - 4m^2g^{\mu\nu}\\
&=& 4[p_1^\mu p_2^\nu - g^{\mu\nu}(p_1 \cdot p_2) + p_2^\mu p_1^\nu] - 4m^2g^{\mu\nu} 
\eea

\item $ (\ou_1 u_2)^* ~=~ ?$
We let $S ~=~ \ou_1 u_2$
\bea
(S)^* &=& (S)^\dag \\
&=& (\ou_1 u_2)^\dag ~,~~ \\
&=& ((u_1)^\dag\ga^0 u_2)^\dag \\
&=& (u_2)^\dag(\ga^0)^\dag(u_1)\\
&=& (u_2)^\dag\ga^0(u_1)\\
&=& \ou_2(u_1)
\eea
Therefore$(\ou_1 u_2)^* ~=~ \ou_2 u_1$.
In order to find $|S|^2$ we simply do the following:
\bea
|S|^2&=& \Tr[\ou_1 u_2 \ou_2 u_1]  \\
&=& \Tr[(\slashed{p}_1+m)(\slashed{p}_2+m)] \\
&=& \Tr[\slashed{p}_1\slashed{p}_2+m(\slashed{p}_1+\slashed{p}_2)+m^2]\\
&=& \Tr[\slashed{p}_1\slashed{p}_2]+\Tr[m(\slashed{p}_1+\slashed{p}_2)]+\Tr[m^2] \\
&=& \Tr[\slashed{p}_1\slashed{p}_2]+m(\Tr[\slashed{p}_1]+\Tr[\slashed{p}_2])+m^2\Tr[1] \\
&=& \Tr[\slashed{p}_1\slashed{p}_2]+4m^2 \\
&=& 4(p_1 \cdot p_2)+4m^2
\eea

\item  By the same reasoning as shown above it can be shown that $(\ou_1 \ga^5 u_2)^* ~=~ \ou_2 \ga^5 u_1$ \\
If we let $P ~=~ \ou_1\ga^5 u_2$ then:
\bea
(P)^* &=& (P)^\dag  \\\
&=& (\ou_1\ga^5 u_2)^\dag ~,~~ \\
&=& ((u_1)^\dag\ga^0\ga^5 u_2)^\dag \\
&=& (u_2)^\dag(\ga^5)^\dag (\ga^0)^\dag(u_1)\\
&=& (u_2)^\dag(\ga^5) \ga^0(u_1)\\
&=& -(u_2)^\dag \ga^0\ga^5(u_1)\\
&=& -\ou_2\ga^5(u_1)
\eea
Therefore  $(\ou_1 \ga^5 u_2)^* ~=~ -\ou_2 \ga^5 u_1$

In order to square $P$ we do the following:
\bea
|P|^2&=& \Tr[\ou_1 \ga^5 u_2(-\ou_2 \ga^5 u_1)] \\
&=& \Tr[u_1 \ou_1 \ga^5(-\slashed{p}_2 - m) \ga^5] \\
&=& \Tr[(\slashed{p}_1+m) \ga^5 (-\slashed{p}_2-m) \ga^5] \\
&=& \Tr[((p_1)_\mu \ga^\mu+m) \ga^5 ((-p_2)_\nu\ga^\nu-m)  \ga^5] \\
&=& \Tr[((p_1)_\mu \ga^\mu\ga^5+m\ga^5) ((-p_2)_\nu\ga^\nu\ga^5-m\ga^5)] \\
&=& \Tr[(p_1)_\mu \ga^\mu\ga^5(-p_2)_\nu\ga^\nu\ga^5+ m\ga^5(-p_2)_\nu\ga^\nu\ga^5 -m\ga^5(p_1)_\mu \ga^\mu\ga^5 - \ga^5\ga^5m^2)] \nl
&=& \Tr[(p_1)_\mu \ga^\mu\ga^5(-p_2)_\nu\ga^\nu\ga^5] + Tr[ m\ga^5(-\ga^5)(-p_2)_\nu\ga^\nu] - Tr[m\ga^5(-\ga^5)(p_1)_\mu \ga^\mu] - Tr[m^2] \nl
&=& \Tr[(p_1)_\mu \ga^\mu\ga^5(-p_2)_\nu\ga^\nu\ga^5] -Tr[ m(-p_2)_\nu\ga^\nu] + Tr[m(p_1)_\mu \ga^\mu] - 4m^2 \\
&=& \Tr[(p_1)_\mu \ga^\mu\ga^5(-p_2)_\nu\ga^\nu\ga^5] - 4m^2 \\
&=& (p_1)_\mu(-p_2)_\nu \Tr[\ga^\mu\ga^5\ga^\nu\ga^5]- 4m^2 \\
&=& (p_1)_\mu(-p_2)_\nu \Tr[\ga^\mu\ga^5(-\ga^5\ga^\nu)]- 4m^2\\
&=& (p_1)_\mu(-p_2)_\nu (-\Tr[\ga^\mu\ga^\nu])- 4m^2\\
&=& (p_1)_\mu(-p_2)_\nu (-4g^{\mu\nu})- 4m^2\\
&=& 4(p_1)(p_2)- 4m^2
\eea

\item  While the above identities could be shown to be trivial, the identity: $(\ou_1 \si^{\mu\nu} u_2)^* ~=~\ou_2 \si^{\nu\mu} u_1$ is more difficult to solve. The identity:$(\si^{\mu\nu})^\dag =\si^{\mu\nu}$ is needed
\bea
(\si^{\mu\nu})^\dag &=& (\frac{i}{2}[\ga^\mu,\ga^\nu])^\dag \\
&=& -\frac{i}{2}([\ga^\mu,\ga^\nu])^\dag\\
&=& -\frac{i}{2}(\ga^\mu\ga^\nu-\ga^\nu\ga^\mu)^\dag \\
&=& -\frac{i}{2}((\ga^\nu)^\dag(\ga^\mu)^\dag-(\ga^\mu)^\dag(\ga^\nu)^\dag) \\
&=& -\frac{i}{2}(\ga^0\ga^\nu\ga^0\ga^0\ga^\mu\ga^0-\ga^0\ga^\mu\ga^0\ga^0\ga^\nu\ga^0)\\
&=& -\frac{i}{2}(\ga^0\ga^\nu\ga^\mu\ga^0-\ga^0\ga^\mu \ga^\nu\ga^0)\\
&=& -\frac{i}{2}((-1)^2\ga^\nu\ga^\mu-(-1)^2\ga^\mu \ga^\nu) \\
&=& -\frac{i}{2}(\ga^\nu\ga^\mu-\ga^\mu \ga^\nu)\\
&=& -\si^{\nu\mu}
\eea
After showing $(\si^{\mu\nu})^\dag =\si^{\nu\mu}$ is true it is trivial to show $(\ou_1 \si^{\mu\nu} u_2)^* ~=~ -\ou_2 \si^{\nu\mu} u_1$.\\
We let $T^{\mu\nu} ~=~ \ou_1 \si^{\mu\nu} u_2$
\bea
(T^{\mu\nu})^* &=& (T^{\mu\nu})^\dag ~,~~ \\
&=& (\ou_1 \si^{\mu\nu} u_2)^\dag \\
&=& ((u_1)^\dag \ga^0 \si^{\mu\nu} u_2)^\dag \\
&=& (u_2^\dag)(\si^{\mu\nu})^\dag (\ga^0)^\dag(u_1)\\
&=& (u_2^\dag)(-\si^{\nu\mu})\ga^0(u_1)\\
&=& (u_2^\dag)(-\ga^0)(-\si^{\nu\mu})(u_1)\\
&=& \ou_2 \si^{\mu\nu}(u_1)
\eea

In order to find the value of $|T^{\mu\nu}|^2$ one needs to find the value of $Tr[\si^{\si\la}\si^{\mu\nu}]$\\
\bea
Tr[\si^{\si\la}\si^{\mu\nu}]&=& Tr[\frac{i}{2}(\ga^\si\ga^\la - \ga^\la\ga^\si)\frac{i}{2}(\ga^\mu\ga^\nu - \ga^\nu\ga^\mu)]\\
&=& Tr[\frac{i}{2}(\ga^\si\ga^\la - \ga^\la\ga^\si)\frac{i}{2}(\ga^\mu\ga^\nu - \ga^\nu\ga^\mu)]\\
&=& -\frac{1}{4}Tr[(\ga^\si\ga^\la - \ga^\la\ga^\si)(\ga^\mu\ga^\nu - \ga^\nu\ga^\mu)]\\
&=& -\frac{1}{4}Tr[\ga^\si\ga^\la\ga^\mu\ga^\nu]+\frac{1}{4}Tr[\ga^\si\ga^\la\ga^\nu\ga^\mu]+\frac{1}{4}Tr[\ga^\la\ga^\si\ga^\mu\ga^\nu]-\frac{1}{4}Tr [\ga^\la\ga^\si\ga^\nu\ga^\mu]\nl
\nonumber
\eea \nonumber
\nonumber
Here we must label each of the traces individually: 
\bea
A &=&-\frac{1}{4}Tr[\ga^\si\ga^\la\ga^\mu\ga^\nu]\\
&=&-(g^{\si\la}g^{\mu\nu}-g^{\si\mu}g^{\la\nu}+g^{\si\nu}g^{\la\mu})
\eea
\bea
B&=&+\frac{1}{4}Tr[\ga^\si\ga^\la\ga^\nu\ga^\mu]\\
&=&+(g^{\si\la}g^{\nu\mu}-g^{\si\nu}g^{\la\mu}+g^{\si\mu}g^{\la\nu})
\eea
\bea
C&=& \frac{1}{4}Tr[\ga^\la\ga^\si\ga^\mu\ga^\nu]\\
&=& (g^{\la\si}g^{\mu\nu}-g^{\la\mu}g^{\si\nu}+g^{\la\nu}g^{\si\mu})
\eea
\bea 
D&=& -\frac{1}{4}Tr [\ga^\la\ga^\si\ga^\nu\ga^\mu]\\
&=& -(g^{\la\si}g^{\nu\mu}-g^{\la\nu}g^{\si\mu}+g^{\la\mu}g^{\si\nu})
\eea
\bea
Tr[\si^{\si\la}\si^{\mu\nu}]&=& A + B+ +C+D\\
&=& 2g^{\si\mu}g^{\la\nu} - 2 g^{\si\nu}g^{\la\mu} - 2 g^{\la\mu}g^{\si\nu} +  2g^{\la\nu}g^{\si\mu})
\eea

In order to find the value of $|T^{\mu\nu}|^2$ one needs to do the following:
\bea
|T^{\mu\nu}|^2 &=& \Tr[\ou_1 \si^{\mu\nu} u_2\ou_2 \si^{\si\la}u_1]\\
&=& \Tr[\ou_1 \si^{\mu\nu}(\slashed{p}_2+m) \si^{\si\la} u_1]\\
&=& \Tr[(\slashed{p}_1+m)\si^{\mu\nu}(\slashed{p}_2+m) \si^{\si\la}]\\
&=& \Tr[((p_1)_\ka \ga^\ka +m)\si^{\mu\nu}((p_2)_\ga \ga^\ga+m) \si^{\si\la}]\\
&=& \Tr[((p_1)_\ka  \ga^\ka\si^{\mu\nu} +m\si^{\mu\nu})((p_2)_\ga \ga^\ga\si^{\si\la}+m\si^{\si\la})]\\
&=& \Tr[((p_1)_\ka \ga^\ka\si^{\mu\nu}(p_2)_\ga \ga^\ga\si^{\si\la} + (p_1)_\ka \ga^\ka\si^{\mu\nu}m\si^{\si\la} +(p_2)_\ga \ga^\ga\si^{\si\la}m\si^{\mu\nu}+m\si^{\si\la}m\si^{\mu\nu})]\\
&=& \Tr[((p_1)_\ka  \ga^\ka\si^{\mu\nu}(p_2)_\ga \ga^\ga\si^{\si\la}] +Tr[(p_1)_\ka \ga^\ka\si^{\mu\nu}m\si^{\si\la} + (p_2)_\ga \ga^\ga\si^{\si\la}m\si^{\mu\nu}]+Tr[m\si^{\si\la}m\si^{\mu\nu}]\nl
&=& \Tr[((p_1)_\ka \ga^\ka\si^{\mu\nu}(p_2)_\ga \ga^\ga\si^{\si\la}] +Tr[m\si^{\si\la}m\si^{\mu\nu}]\\
&=& \Tr[((p_1)_\ka  \ga^\ka\si^{\mu\nu}(p_2)_\ga \ga^\ga \si^{\si\la}] +m^2Tr[\si^{\si\la}\si^{\mu\nu}]{\rm~Let~ B~=m^2Tr[\si^{\si\la}\si^{\mu\nu}]}\nl
&=& \Tr[((p_1)_\ka  \ga^\ka\si^{\mu\nu}(p_2)_\ga \ga^\ga \si^{\si\la}] +B\nl
&=& (p_1)_\ka (p_2)_\ga \Tr[\ga^\ka \si^{\mu\nu}\ga^\ga \si^{\si\la}]+B\nl
&=& (p_1)_\ka (p_2)_\ga \Tr[\ga^\ka (\frac{i}{2}((\ga^\mu \ga^\nu - \ga^\nu\ga^\mu)\ga^\ga(\frac{i}{2}(\ga^\si \ga^\la - \ga^\la \ga^\si))]+B\\
&=& -\frac{1}{4} (p_1)_\ka (p_2)_\ga\Tr[\ga^\ka (\ga^\mu \ga^\nu - \ga^\nu \ga^\mu) \ga^\ga (\ga^\si \ga^\la - \ga^\la \ga^\si)]+B{\rm~(Let~ A~=-\frac{1}{4} (p_1)_\ka (p_2)_\ga)}\nl
&=& (A)\Tr[(\ga^\ka \ga^\mu \ga^\nu -\ga^\ka \ga^\nu \ga^\mu) (\ga^\ga\ga^\si \ga^\la - \ga^\ga\ga^\la \ga^\si)]+B \nl
&=& (A)\Tr[(\ga^\ka \ga^\mu \ga^\nu\ga^\ga\ga^\si \ga^\la -\ga^\ka \ga^\mu \ga^\nu\ga^\ga\ga^\la \ga^\si -  \ga^\ka \ga^\nu \ga^\mu\ga^\ga\ga^\si \ga^\la + \ga^\ka \ga^\nu \ga^\mu\ga^\ga\ga^\la \ga^\si)]+B\nl
&=& (A)\Tr[(\ga^\ka \ga^\mu \ga^\nu\ga^\ga\ga^\si \ga^\la -\ga^\ka \ga^\mu \ga^\nu\ga^\ga(-\ga^\si \ga^\la) - \ga^\ka \ga^\nu \ga^\mu\ga^\ga\ga^\si \ga^\la + \ga^\ka \ga^\nu \ga^\mu\ga^\ga(-\ga^\si \ga^\la))]+B\nl
&=& (A)\Tr[(\ga^\ka \ga^\mu \ga^\nu\ga^\ga\ga^\si \ga^\la +\ga^\ka \ga^\mu \ga^\nu\ga^\ga\ga^\si \ga^\la - \ga^\ka \ga^\nu \ga^\mu\ga^\ga\ga^\si \ga^\la - \ga^\ka \ga^\nu \ga^\mu\ga^\ga\ga^\si \ga^\la))]+B\nl
&=& (A)\Tr[2\ga^\ka \ga^\mu \ga^\nu\ga^\ga\ga^\si\ga^\la- 2\ga^\ka \ga^\nu \ga^\mu\ga^\ga\ga^\si\ga^\la]+B\\
&=& (A)\Tr[2\ga^\ka (-\ga^\nu \ga^\mu)\ga^\ga\ga^\si\ga^\la- 2\ga^\ka \ga^\nu \ga^\mu\ga^\ga\ga^\si\ga^\la]+B\\
&=& (A)\Tr[-2\ga^\ka \ga^\nu \ga^\mu\ga^\ga\ga^\si\ga^\la- 2\ga^\ka \ga^\nu \ga^\mu\ga^\ga\ga^\si\ga^\la]+B\\
&=& (A) \Tr[-4\ga^\ka \ga^\nu \ga^\mu \ga^\ga \ga^\si \ga^\la]+B\\
&=& -\frac{1}{4} (p_1)_\ka (p_2)_\ga \Tr[-4\ga^\ka \ga^\mu \ga^\nu \ga^\ga \ga^\si \ga^\la]+B\\
&=& (p_1)_\ka (p_2)_\ga\Tr[\ga^\ka \ga^\ga]+B~{\rm (Using~ a~ similar~ identity~ as~ shown~ in~ eq.~ 91)}\\
&=& 4(p_1)_\ka (p_2)_\ga g^{\ka\ga}+B\\
&=& 4p_1p_2 + m^2Tr[\si^{\si\la}\si^{\mu\nu}]\\
&=& 4p_1p_2 + m^24(g^{\mu\nu}g^{\la\si}- g^{\mu\la}g^{\nu\si}+g^{\mu\si}g^{\nu\la})
\eea

\pagestyle{plain}

% Roman Section & Subsection numberings
% -------------------------------------
\renewcommand{\thesection}{\Roman{section}}
\renewcommand{\thesubsection}{\Roman{subsection}}

% Start of document
% -----------------
\pagestyle{plain}
\allowdisplaybreaks

\setlength{\abovecaptionskip}{0pt}
\setlength{\belowcaptionskip}{0pt}


\begin{center}
\underline{\textbf{\Large Branching Ratios}}
\end{center}

\section{The branching Ratio of a Pion}
In order to calculate the Branching ratio of a charged pion, one must be familiar with the Feynman Rules of calculating amplitudes and the trace identities. \\
\\
We must start with recognizing that the pion decay is a charged, weak, interaction which arises from the fact that a pion is made of quarks, and the decay is mediated by a massive W boson. \\
\\
A diagram of the decay may been seen below: 
\begin{figure}[h]
\centering
\includegraphics[scale=.35]{pion2}
\caption{Pion Decay}
\end{figure}

\scalebox{.6}{Picture courtesy of Qora.com}\\ 
\\
Where the up and ani-down quark are the pion and the muon and muon neutrino are on the right. However, the muon and the muon neutrino could be the electron and electron neutrino.  \\
\\
The formula given to describe the branching ratio of the decay is given by\cite{Griffiths}: 
\bea 
\Ga &=& \frac{S|\textbf{p}|}{8\pi\hbar m_1^2c} |\mathcal{M}|^2
\eea 
Where $|\textbf{p}|$ is the outgoing momentum, $S$ is the product of statistical factors (in our case it will be equal to 1), and $\mathcal{M}$ is the Feynman amplitude. In our notation, we will use the Natural Units so our expression becomes: 
\bea 
\Ga &=& \frac{S|\textbf{p}|}{8\pi m_1^2} |\mathcal{M}|^2
\eea 
The next step is to determine the Feynman Amplitude which can be done in a few steps 
In the charged weak lepton decays, we have different notation for the vertices and propagators. 
\begin{enumerate}
    \item For each vertex add a factor of $\frac{-ig_w}{2\sqrt{2}}(\ga^\nu(1-\ga^5))$ where $g_w= \sqrt{4\pi\al_w}$
    \item For each propagator we add a factor of $\frac{-ig_{\mu\nu}-\frac{q_\mu q_\nu}{m^2}}{q^2-m^2}$ where m is the mass of the boson. In our case, $m_w \gg\ q$ so the expression simplifies to $\frac{ig_{\mu\nu}}{m_w^2}$
\end{enumerate}
From these rules and Figure 1 we are able to calculate the value of $\mathcal{M}$ 
\bea
-i\mathcal{M}&=&[\ou(3)(\frac{-ig_w}{2\sqrt{2}}(\ga^\nu(1-\ga^5))v(2)] [ \frac{ig_{\mu\nu}}{m_w^2} ][\frac{-ig_w}{2\sqrt{2}}F^\mu]
\eea
We have a factor of -i with the $\mathcal{M}$ so that we obtain the real part of the expression and $F^\mu$ is the form factor of the coupling of the pion to the $W$ boson. $F^\mu$ has the form of $f_\pi p^\mu$ 
\bea
-i\mathcal{M}&=&\frac{-ig_w^2}{8 m_w^2}[\ou(3)(\ga^\mu(1-\ga^5))v(2)]F^\mu\\
\mathcal{M}&=&\frac{g_w^2}{8 m_w^2}[\ou(3)(\ga^\mu(1-\ga^5))v(2)]F^\mu
\eea
In order to square the amplitude we do the following: 
\bea
\langle|\mathcal{M}^2|\rangle&=&(\frac{g_w^2}{8 m_w^2}f_\pi)^2Tr[(\ou(3)(\ga^\mu(1-\ga^5))v(2)p_\mu(\ov(2)(\ga^\nu(1-\ga^5)u(3))]p_\nu\\
\langle|\mathcal{M}^2|\rangle&=&(\frac{g_w^2}{8 m_w^2}f_\pi)^2p_\mu p_\nu Tr[(u(3)\ou(3)(\ga^\mu(1-\ga^5))v(2) \ov(2)(\ga^\nu(1-\ga^5))]\\
\langle|\mathcal{M}^2|\rangle&=&(\frac{g_w^2}{8 m_w^2}f_\pi)^2p_\mu p_\nu Tr[((\slashed{p}_3 + m_l)(\ga^\mu(1-\ga^5)) \slashed{p}_2(\ga^\nu(1-\ga^5))]\\
\langle|\mathcal{M}^2|\rangle&=&(\frac{g_w^2}{8m_w^2}f_\pi)^2p_\mu p_\nu (2Tr[\slashed{p}_3\ga^\mu\slashed{p}_2\ga^\nu] -2Tr[\slashed{p}_3\ga^\mu\slashed{p}_2\ga^\nu\ga^5])\\
\langle|\mathcal{M}^2|\rangle&=&(\frac{g_w^2}{8m_w^2}f_\pi)^2p_\mu p_\nu (8[p_3^\mu p_2^\nu + p_2^\mu p_3^\nu -(p_3 \cdot p_2)g^{\mu\nu}]+8i\epsilon^{\mu\la\nu\si}p_{3\la}p_{2\si}
\eea
Summing over the spins gives us: 
\bea 
\langle|\mathcal{M}^2|\rangle&=&(\frac{g_w^2}{8m_w^2}f_\pi)^2 [2(p_1\cdot p_2)(p_1 \cdot p_3) -p^2(p_2 \cdot p_3)] 
\eea 
Since $p=p_2 + p_3$, we can simplify the equation further 
\bea 
\langle|\mathcal{M}^2|\rangle&=&(\frac{g_w^2}{2m_w^2}f_\pi)^2 [m_l^2(m_\pi^2-m_l^2)] 
\eea 
In this way we are able to calculate the branching ratio of a pion, since we know have the Feyneman Amplitude we simply return to equation (2)
\bea 
\Ga &=& \frac{f_\pi^2}{\pi m_\pi^3}(\frac{g_w}{4m_w})^4 m_l^2(m_\pi^2-m_l^2)^2
\eea 
Expanding on this idea, we are able to graph $\frac{\Ga_l}{\Ga_\pi}$ using the equation:\nl 
\bea
\Ga_\pi = \frac{1}{\tau_\pi}
\eea
\\
\\
\\
\\
\\
\\
\\
\\
The graph of the equation is:
\begin{figure}[h]
\centering
\includegraphics[scale=.65]{brpion}
\caption{$\frac{\Ga_l}{\Ga_\pi}$}
\end{figure}
\\ 
Using the values from PDG \cite{Agashe:2014kda} we can do calculations with the branching ratios. \\
\begin{center}
 \begin{tabular}{||c c c c||} 
 \hline
 Observables & $e$ & $\mu$ & $\pi$  \\ [0.5ex] 
 \hline\hline
 $\tau$ & $6.6 \times 10^{28}$ yr &$2.1969811(22) \times 10^-6$ s  & $2.6033(5) \times 10^-8$ s \\ 
 \hline
 Mass(MeV) & $0.5109989461(31)$ & $105.6583745(24)$  & $139.57061(24)$ \\[1ex] 
 \hline
\end{tabular}
\end{center}
If we would like to calculate the ratio of the $\pi^- \rightarrow e^- + \onu_e$ and  $\pi^- \rightarrow \mu^- + \onu_\mu$ we simply do the following\cite{Agashe:2014kda}: 
\bea 
\frac{\Ga_e}{\Ga_\mu} &=& \frac{m_e^2(m_\pi^2-m_e^2)^2}{m_\mu^2(m_\pi^2-m_\mu^2)^2}\\
\frac{\Ga_e}{\Ga_\mu} &=& 1.28334(73) \times 10^{-4}
\eea 
\begin{thebibliography}{9}
\bibitem{Griffiths} 
Griffiths, D. (2014). Introduction to elementary particles. Weinheim: Wiley-VCH Verlag

%\cite{Agashe:2014kda}
\bibitem{Agashe:2014kda} 
  K.~A.~Olive {\it et al.} [Particle Data Group],
  %``Review of Particle Physics,''
  Chin.\ Phys.\ C {\bf 38}, 090001 (2014).
  doi:10.1088/1674-1137/38/9/090001
  %%CITATION = doi:10.1088/1674-1137/38/9/090001;%%
  %6385 citations counted in INSPIRE as of 03 Jul 2017
\end{thebibliography}
\end{enumerate}
\end{document}
