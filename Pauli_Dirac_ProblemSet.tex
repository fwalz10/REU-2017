% Fun with gamma matrices
% May 05, 2016

\documentclass[12pt]{article}

\usepackage{hyperref}
\usepackage{framed}

\usepackage{graphicx}
\usepackage{amsmath}
%\usepackage{mathtools}
\usepackage{amssymb}
\usepackage{slashed}
\usepackage{bm}
\usepackage{cite}
\usepackage{setspace}
\usepackage{bigints}

\setlength{\oddsidemargin}{0in}
\setlength{\textwidth}{6.5in}
\setlength{\topmargin}{0in}
\setlength{\textheight}{9in}
\voffset-1.5truecm

\def \beq{\begin{equation}}
\def \eeq{\end{equation}}
\def \bea{\begin{eqnarray}}
\def \eea{\end{eqnarray}}
\def \ba{\begin{array}}
\def \ea{\end{array}}
\def \beg{\begin{gather}}
\def \eeg{\end{gather}}
\def \bmat{\begin{matrix}}
\def \emat{\end{matrix}}
\def \bfra{\begin{framed}}
\def \efra{\end{framed}}

\def \({\left(}
\def \){\right)}
\def \[{\left[}
\def \]{\right]}
\def \<{\left\langle}
\def \>{\right\rangle}
\def \lp{\left|}
\def \rp{\right|}
\def \lb{\left\{}
\def \rb{\right\}}
\def \l.{\left.}
\def \r.{\right.}

\def \bma{\(\bmat}
\def \ema{\emat\)}

\def \Tr{{\rm Tr}}
\def \N{\rm N}
\def \hc{{\rm h.\ c.}}

\def \ph{\phantom}
\def \nn{\nonumber}
\def \nl{\nn \\}

\def \of{\frac{1}{4}}
\def \oth{\frac{1}{3}}
\def \hf{\frac{1}{2}}
\def \oet{\frac{1}{8}}
\def \s{\sqrt{2}}
\def \st{\sqrt{3}}
\def \sx{\sqrt{6}}

\def \cB{{\cal B}}
\def \cC{{\cal C}}
\def \cZ{{\cal Z}}
\def \cL{{\cal L}}
\def \cH{{\cal H}}
\def \cO{{\cal O}}
\def \cM{{\cal M}}
\def \cl{\ell}

\def \oq{\overline{q}}
\def \of{\overline{f}}
\def \os{\overline{s}}
\def \ob{\overline{b}}
\def \ou{\overline{u}}
\def \ov{\overline{v}}
\def \oBs{\overline{B^0_s}}
\def \otau{\overline{\tau}}
\def \omu{\overline{\mu}}
\def \onu{\overline{\nu}}
\def \osi{\overline{\si}}
\def \oell{\overline{\ell}}
\def \oL{\overline{L}}
\def \oQ{\overline{Q}}

\def \al{\alpha}
\def \be{\beta}
\def \ga{\gamma}
\def \Ga{\Gamma}
\def \de{\delta}
\def \ka{\kappa}
\def \De{\Delta}
\def \ep{\epsilon}
\def \tha{\theta}
\def \la{\lambda}
\def \La{\Lambda}
\def \si{\sigma}
\def \Si{\Sigma}
\def \Up{\Upsilon}
\def \om{\omega}
\def \1{1}%\mathbb{I}}

\def \pa{\partial}

\def \SM{{\rm SM}}
\def \elm{{\rm em}}
\def \NP{{\rm NP}}
\def \expt{{\rm expt}}
\def \eff{{\rm eff}}
\def \elm{{\rm em}}
\def \RFG{{\rm RFG}}
\def \CFG{{\rm CFG}}
\def \PT{{\rm PT}}
\def \hi{{\rm hi}}
\def \lo{{\rm lo}}

\def \Re{{\rm Re}}
\def \Im{{\rm Im}}

\def \keV{{\rm keV}}
\def \MeV{{\rm MeV}}
\def \GeV{{\rm GeV}}

\def \cH{{\cal H}}
\def \cI{{\cal I}}

\def \Rqp{R_{q'}}

\def\bwt{\begin{widetext}}
\def\ewt{\end{widetext}}

\def\vev#1{\langle {#1} \rangle}
\def\eq#1{eq.~(\ref{#1})}

\def \vp{{\bf p}}
\def \vq{{\bf q}}

% Start of document
% -----------------
\pagestyle{plain}

% Roman Section & Subsection numberings
% -------------------------------------
\renewcommand{\thesection}{\Roman{section}}
\renewcommand{\thesubsection}{\Roman{subsection}}

% Start of document
% -----------------
\pagestyle{plain}
\allowdisplaybreaks

\setlength{\abovecaptionskip}{0pt}
\setlength{\belowcaptionskip}{0pt}

\begin{document}

\begin{center}
\underline{\textbf{\Large Pauli ($\si$) and Dirac ($\gamma$) matrices}}
\end{center}

\begin{enumerate}

\item {\bf Pauli matrices}: Consider the following 2$\times$2 matrices:
\bea
\si_1 ~=~ \bma 0 & 1 \\ 1 & 0 \ema ~,~~ \si_2 ~=~ \bma 0 & -i \\ i & 0 \ema ~,~~
\si_3 ~=~ \bma 1 & 0 \\ 0 & -1 \ema ~.~~
\eea
\begin{enumerate}
  \item [a.] Show that these matrices are both Hermitian, and Unitary.
  \item [b.] Show that $\si^2_i = \1$ for any $i = 1, 2, 3$ where $\1$ represents the
  2$\times$2 identity matrix. (Note that in general $\1$ will represent the
  corresponding N$\times$N identity matrix where N is the dimension of the representation
  that fits the situation.)
  \item [c.] Establish the commutation relationship: $[\si_i, \si_j] ~=~ 2 i
  \ep_{ijk}\si_k$.
  \item [d.] Establish the anticommutation relationship: $\{\si_i,\si_j\} ~=~ 2\de
  _{ij}$.
  \item [e.] Show that $\si_i\si_j = \de_{ij} + i \ep_{ijk}\si_k$.
  \item [f.] Find the determinant and the trace of each Pauli matrix.
  \item [g.] What is the trace of $\si_i\si_j$? (Hint: use the cyclic property of
  matrix traces: $\Tr[AB] = \Tr[BA]$, $\Tr[AB...YZ] = \Tr[ZAB...Y]$, etc., where $A,
  B, ...$ are matrices.)
  \item [f.] What is the trace of $\si_i\si_j\si_k$? Do you notice a patern?
\end{enumerate}
The symbols $\de_{ij}$ and $\ep_{ijk}$ are very useful. $\de_{ij}$ is known as
the Kronecker delta. If $i$ and $j$ are equal then the delta is 1, or otherwise
it is 0, i.e. $\de_{12} ~=~ 0$, but $\de_{11} ~=~ 1$. The Levi-Civita symbol
($\ep_{ijk}$) is even more interesting. It is non-zero only if all three indices
($i, j, k$) are unequal, and it is $+1(-1)$ when $i, j, k$ is an even(odd) permutation
of $1, 2, 3$, i.e. $\ep_{223} ~=~ 0, \ep_{123} ~=~ 1, \ep_{213} ~=~ -1$.

\item {\bf More fun with Pauli matrices}: The three Pauli matrices together with the
2$\times$2 identity matrix $\1$ form a basis for all 2$\times$2 matrices. In this
problem we want to show this. There are two steps:
\begin{enumerate}
  \item [a.] Show that the four matrices ($\1$ and $\si_i$ for $i = 1, 2, 3$) are linearly
  independent. (Hint: You have worked the math out in problem 1. Here you simply have to
  identify the relevant steps.)
  \item [b.] Consider the most general 2$\times$2 matrix:
   \bea
    M ~=~ \bma a & b \\ c & d \ema ~,~~
   \eea
  where $a, b, c, d$ are in general four different complex numbers. Show, by explicit
  construction, that the matrix $M$ can be expressed as a unique linear superposition
  of the 4 matrices.
  (Hint: Write $M = A~\1 + B_i\cdot\si_i$ and solve for the four coefficients)

\end{enumerate}

\item {\bf Dirac $\ga$ matrices}: The equation of motion for a fermion, also known as the
Dirac equation, is:
\bea
(i\ga^\mu\pa_\mu - m)\psi &=& 0 ~.
\eea
In four spacetime dimensions (Minkowski spacetime), it turns out that each component
of the four vector represented by $\ga^\mu$ has to be (at least) a 4$\times$4 matrix.
It is possible to have different representations of these matrices, but for now we will
consider the following representation (known as the Chiral or Weyl basis):
%
\bea
\ga^0 ~=~ \bma 0 & 1 \\ 1 & 0 \ema ~,~~ \ga^i ~=~ \bma 0 & \si^i \\ -\si^i & 0 \ema ~.~~
\eea
%
A shorthand notation for the four matrices is:
%
\bea
\ga^\mu ~=~ \bma 0 & \si^\mu \\ \osi^\mu & 0 \ema ~,~~~~ \si^\mu \equiv (1 , \si^i) ~,~~
\osi^\mu \equiv (1, -\si^i) ~.~~
\eea
%
(Note that in the above notation $0$ and $1$ represent 2$\times$2 matrices.)
With the above notation in mind show the following properties of $\ga$ matrices:
\begin{enumerate}
 \item [a.] What is $(\ga^\mu)^2$? (Work it out for each value that $\mu$ can take.)
 \item [b.] Show that $(\ga^\mu)^\dag ~=~ \ga^0\ga^\mu\ga^0$.
 \item [c.] Show that $\{\ga^\mu,\ga^\nu\} ~=~ 2g^{\mu\nu}$.
 \item [d.] Define a new object $\si^{\mu\nu} ~\equiv~ \dfrac{i}{2}[\ga^\mu,\ga^\nu]$.
 Show that $\ga^\mu\ga^\nu ~=~ g^{\mu\nu} + i\si^{\nu\mu}$
 \item [e.] Define another new object $\ga^5 ~=~ i\ga^0\ga^1\ga^2\ga^3$. Express this
 as a simple 4$\times$4 matrix.
 \item [f.] Find the properties of $\ga^5$ : What is $(\ga^5)^2, \{\ga^\mu,\ga^5\}$,
 and $(\ga^5)^\dag$?
 \item [g.] What is the determinant and the trace of each of the five matrices ($\ga^\mu$
 and $\ga^5$)?
 \item [h.] What is the trace of $\ga^\mu\ga^\nu$? (Hint: Remember the cyclic property of
 matrix traces.)
 \item [i.] Consider the operators $P_L ~=~ (1 - \ga^5)/2~,~~ P_R ~=~ (1 + \ga^5)/2$. Find
 their properties. (Find $P_L^2, P_R^2, P_LP_R$. Do you notice anything interesting? Do
 these (anti)commute with the four gamma matrices? Find the (anti)commutation relationships.)
 \item [j.*] What is the trace of $\ga^\mu\ga^\nu\ga^5$?
 \item [k.*] What is the trace of $\ga^\mu\ga^\nu\ga^\la$?
 \item [l.*] What is the trace of $\ga^\mu\ga^\nu\ga^\la\ga^\si$?
\end{enumerate}
* These problems need more advanced work. Work on these for extra credit!
\end{enumerate}

The answer for (3l.) is as follows:

We will assume the following relationships: $\Tr[ABCD] ~=~ \Tr[DABC]~,~\{\ga^\mu,\ga^\nu\}~=~2 g^{\mu\nu}$ and
$\Tr\[\ga^\mu\ga^\nu\]~=~ 4 g^{\mu\nu}$. We then get:
%
\bea
\Tr\[\ga^\mu\ga^\nu\ga^\la\ga^\si\] &=& \Tr\[(2 g^{\mu\nu}-\ga^\nu\ga^\mu)\ga^\la\ga^\si\] ~,~~ \nl
&=& 2 g^{\mu\nu}\Tr\[\ga^\la\ga^\si\] - \Tr\[\ga^\nu(2 g^{\mu\la} - \ga^\la\ga^\mu)\ga^\si\] ~,~~ \nl
&=& 8 g^{\mu\nu}g^{\la\si} - 2g^{\mu\la}\Tr\[\ga^\nu\ga^\si\] + \Tr\[\ga^\nu\ga^\la(2 g^{\mu\si} - \ga^\si\ga^\mu)\] ~,~~ \nl
&=& 8(g^{\mu\nu}g^{\la\si} - g^{\mu\la}g^{\nu\si} + g^{\mu\si}g^{\nu\la}) - \Tr[\ga^\nu\ga^\la\ga^\si\ga^\mu] ~,~~ \nl
&=& 8(g^{\mu\nu}g^{\la\si} - g^{\mu\la}g^{\nu\si} + g^{\mu\si}g^{\nu\la}) - \Tr[\ga^\mu\ga^\nu\ga^\la\ga^\si] ~,~~ \\
\Rightarrow \Tr\[\ga^\mu\ga^\nu\ga^\la\ga^\si\] &=& 4(g^{\mu\nu}g^{\la\si} - g^{\mu\la}g^{\nu\si} + g^{\mu\si}g^{\nu\la})~.~~
\eea
%

\end{document}