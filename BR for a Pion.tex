\documentclass[12pt]{article}

\usepackage{hyperref}
\usepackage{framed}

\usepackage{graphicx}
\usepackage{amsmath}
%\usepackage{mathtools}
\usepackage{amssymb}
\usepackage{slashed}
\usepackage{bm}
\usepackage{cite}
\usepackage{setspace}
\usepackage{bigints}
\usepackage{color}

\setlength{\oddsidemargin}{0in}
\setlength{\textwidth}{6.5in}
\setlength{\topmargin}{0in}
\setlength{\textheight}{9in}
\voffset-1.5truecm

\def \beq{\begin{equation}}
\def \eeq{\end{equation}}
\def \bea{\begin{eqnarray}}
\def \eea{\end{eqnarray}}
\def \ba{\begin{array}}
\def \ea{\end{array}}
\def \beg{\begin{gather}}
\def \eeg{\end{gather}}
\def \bmat{\begin{matrix}}
\def \emat{\end{matrix}}
\def \bfra{\begin{framed}}
\def \efra{\end{framed}}

\def \({\left(}
\def \){\right)}
\def \[{\left[}
\def \]{\right]}
\def \<{\left\langle}
\def \>{\right\rangle}
\def \lp{\left|}
\def \rp{\right|}
\def \lb{\left\{}
\def \rb{\right\}}
\def \l.{\left.}
\def \r.{\right.}

\def \bma{\(\bmat}
\def \ema{\emat\)}

\def \Tr{{\rm Tr}}
\def \N{\rm N}
\def \hc{{\rm h.\ c.}}

\def \ph{\phantom}
\def \nn{\nonumber}
\def \nl{\nn \\}

\def \of{\frac{1}{4}}
\def \oth{\frac{1}{3}}
\def \hf{\frac{1}{2}}
\def \oet{\frac{1}{8}}
\def \s{\sqrt{2}}
\def \st{\sqrt{3}}
\def \sx{\sqrt{6}}

\def \cB{{\cal B}}
\def \cC{{\cal C}}
\def \cZ{{\cal Z}}
\def \cL{{\cal L}}
\def \cH{{\cal H}}
\def \cO{{\cal O}}
\def \cM{{\cal M}}
\def \cl{\ell}

\def \oq{\overline{q}}
\def \of{\overline{f}}
\def \os{\overline{s}}
\def \ob{\overline{b}}
\def \ou{\overline{u}}
\def \ov{\overline{v}}
\def \oBs{\overline{B^0_s}}
\def \otau{\overline{\tau}}
\def \omu{\overline{\mu}}
\def \onu{\overline{\nu}}
\def \osi{\overline{\si}}
\def \oell{\overline{\ell}}
\def \oL{\overline{L}}
\def \oQ{\overline{Q}}

\def \al{\alpha}
\def \be{\beta}
\def \ga{\gamma}
\def \Ga{\Gamma}
\def \de{\delta}
\def \ka{\kappa}
\def \De{\Delta}
\def \ep{\epsilon}
\def \tha{\theta}
\def \la{\lambda}
\def \La{\Lambda}
\def \si{\sigma}
\def \Si{\Sigma}
\def \Up{\Upsilon}
\def \om{\omega}
\def \1{1}%\mathbb{I}}

\def \pa{\partial}

\def \SM{{\rm SM}}
\def \elm{{\rm em}}
\def \NP{{\rm NP}}
\def \expt{{\rm expt}}
\def \eff{{\rm eff}}
\def \elm{{\rm em}}
\def \RFG{{\rm RFG}}
\def \CFG{{\rm CFG}}
\def \PT{{\rm PT}}
\def \hi{{\rm hi}}
\def \lo{{\rm lo}}

\def \Re{{\rm Re}}
\def \Im{{\rm Im}}

\def \keV{{\rm keV}}
\def \MeV{{\rm MeV}}
\def \GeV{{\rm GeV}}

\def \cH{{\cal H}}
\def \cI{{\cal I}}

\def \Rqp{R_{q'}}

\def\bwt{\begin{widetext}}
\def\ewt{\end{widetext}}

\def\vev#1{\langle {#1} \rangle}
\def\eq#1{eq.~(\ref{#1})}

\def \vp{{\bf p}}
\def \vq{{\bf q}}

\def \cre{\color{red}}
\def \cgr{\color{green}}

\usepackage{graphicx} %package to manage images
\graphicspath{ {images/} }

\usepackage[rightcaption]{sidecap}

\usepackage{wrapfig}

% Start of document
% -----------------
\pagestyle{plain}

% Roman Section & Subsection numberings
% -------------------------------------
\renewcommand{\thesection}{\Roman{section}}
\renewcommand{\thesubsection}{\Roman{subsection}}

% Start of document
% -----------------
\pagestyle{plain}
\allowdisplaybreaks

\setlength{\abovecaptionskip}{0pt}
\setlength{\belowcaptionskip}{0pt}

\begin{document}

\begin{center}
\underline{\textbf{\Large Branching Ratios}}
\end{center}

\section{The branching Ratio of a Pion}
In order to calculate the Branching ratio of a charged pion, one must be familiar with the Feynman Rules of calculating amplitudes and the trace identities. \\
\\
We must start with recognizing that the pion decay is a charged, weak, interaction which arises from the fact that a pion is made of quarks, and the decay is mediated by a massive W boson. \\
\\
A diagram of the decay may been seen below: 
\begin{figure}[h]
\centering
\includegraphics[scale=.35]{pion2}
\caption{Pion Decay}
\end{figure}

\scalebox{.6}{Picture courtesy of Qora.com}\\ 
\\
Where the up and ani-down quark are the pion and the muon and muon neutrino are on the right. However, the muon and the muon neutrino could be the electron and electron neutrino.  \\
\\
The formula given to describe the branching ratio of the decay is given by\cite{Griffiths}: 
\bea 
\Ga &=& \frac{S|\textbf{p}|}{8\pi\hbar m_1^2c} |\mathcal{M}|^2
\eea 
Where $|\textbf{p}|$ is the outgoing momentum, $S$ is the product of statistical factors (in our case it will be equal to 1), and $\mathcal{M}$ is the Feynman amplitude. In our notation, we will use the Natural Units so our expression becomes: 
\bea 
\Ga &=& \frac{S|\textbf{p}|}{8\pi m_1^2} |\mathcal{M}|^2
\eea 
The next step is to determine the Feynman Amplitude which can be done in a few steps 
In the charged weak lepton decays, we have different notation for the vertices and propagators. 
\begin{enumerate}
    \item For each vertex add a factor of $\frac{-ig_w}{2\sqrt{2}}(\ga^\nu(1-\ga^5))$ where $g_w= \sqrt{4\pi\al_w}$
    \item For each propagator we add a factor of $\frac{-ig_{\mu\nu}-\frac{q_\mu q_\nu}{m^2}}{q^2-m^2}$ where m is the mass of the boson. In our case, $m_w \gg\ q$ so the expression simplifies to $\frac{ig_{\mu\nu}}{m_w^2}$
\end{enumerate}
From these rules and Figure 1 we are able to calculate the value of $\mathcal{M}$ 
\bea
-i\mathcal{M}&=&[\ou(3)(\frac{-ig_w}{2\sqrt{2}}(\ga^\nu(1-\ga^5))v(2)] [ \frac{ig_{\mu\nu}}{m_w^2} ][\frac{-ig_w}{2\sqrt{2}}F^\mu]
\eea
We have a factor of -i with the $\mathcal{M}$ so that we obtain the real part of the expression and $F^\mu$ is the form factor of the coupling of the pion to the $W$ boson. $F^\mu$ has the form of $f_\pi p^\mu$ 
\bea
-i\mathcal{M}&=&\frac{-ig_w^2}{8 m_w^2}[\ou(3)(\ga^\mu(1-\ga^5))v(2)]F^\mu\\
\mathcal{M}&=&\frac{g_w^2}{8 m_w^2}[\ou(3)(\ga^\mu(1-\ga^5))v(2)]F^\mu
\eea
In order to square the amplitude we do the following: 
\bea
\langle|\mathcal{M}^2|\rangle&=&(\frac{g_w^2}{8 m_w^2}f_\pi)^2Tr[(\ou(3)(\ga^\mu(1-\ga^5))v(2)p_\mu(\ov(2)(\ga^\nu(1-\ga^5)u(3))]p_\nu\\
\langle|\mathcal{M}^2|\rangle&=&(\frac{g_w^2}{8 m_w^2}f_\pi)^2p_\mu p_\nu Tr[(u(3)\ou(3)(\ga^\mu(1-\ga^5))v(2) \ov(2)(\ga^\nu(1-\ga^5))]\\
\langle|\mathcal{M}^2|\rangle&=&(\frac{g_w^2}{8 m_w^2}f_\pi)^2p_\mu p_\nu Tr[((\slashed{p}_3 + m_l)(\ga^\mu(1-\ga^5)) \slashed{p}_2(\ga^\nu(1-\ga^5))]\\
\langle|\mathcal{M}^2|\rangle&=&(\frac{g_w^2}{8m_w^2}f_\pi)^2p_\mu p_\nu (2Tr[\slashed{p}_3\ga^\mu\slashed{p}_2\ga^\nu] -2Tr[\slashed{p}_3\ga^\mu\slashed{p}_2\ga^\nu\ga^5])\\
\langle|\mathcal{M}^2|\rangle&=&(\frac{g_w^2}{8m_w^2}f_\pi)^2p_\mu p_\nu (8[p_3^\mu p_2^\nu + p_2^\mu p_3^\nu -(p_3 \cdot p_2)g^{\mu\nu}]+8i\epsilon^{\mu\la\nu\si}p_{3\la}p_{2\si}
\eea
Summing over the spins gives us: 
\bea 
\langle|\mathcal{M}^2|\rangle&=&(\frac{g_w^2}{8m_w^2}f_\pi)^2 [2(p_1\cdot p_2)(p_1 \cdot p_3) -p^2(p_2 \cdot p_3)] 
\eea 
Since $p=p_2 + p_3$, we can simplify the equation further 
\bea 
\langle|\mathcal{M}^2|\rangle&=&(\frac{g_w^2}{2m_w^2}f_\pi)^2 [m_l^2(m_\pi^2-m_l^2)] 
\eea 
In this way we are able to calculate the branching ratio of a pion, since we know have the Feyneman Amplitude we simply return to equation (2)
\bea 
\Ga &=& \frac{f_\pi^2}{\pi m_\pi^3}(\frac{g_w}{4m_w})^4 m_l^2(m_\pi^2-m_l^2)^2
\eea 
If we would like to calculate the ratio of the $\pi^- \rightarrow e^- + \onu_e$ and  $\pi^- \rightarrow \mu^- + \onu_\mu$ we simply do the following\cite{Agashe:2014kda}: 
\bea 
\frac{\Ga_e}{\Ga_\mu} &=& \frac{m_e^2(m_\pi^2-m_e^2)^2}{m_\mu^2(m_\pi^2-m_\mu^2)^2}\\
\frac{\Ga_e}{\Ga_\mu} &=& \frac{(.511^2(139.57^2-.511^2)^2}{105.66^2(139.57^2-105.66^2)^2}\\
\frac{\Ga_e}{\Ga_\mu} &=& 0.000128
\eea 
\begin{thebibliography}{9}
\bibitem{Griffiths} 
Griffiths, D. (2014). Introduction to elementary particles. Weinheim: Wiley-VCH Verlag

%\cite{Agashe:2014kda}
\bibitem{Agashe:2014kda} 
  K.~A.~Olive {\it et al.} [Particle Data Group],
  %``Review of Particle Physics,''
  Chin.\ Phys.\ C {\bf 38}, 090001 (2014).
  doi:10.1088/1674-1137/38/9/090001
  %%CITATION = doi:10.1088/1674-1137/38/9/090001;%%
  %6385 citations counted in INSPIRE as of 03 Jul 2017
\end{thebibliography}
\end{document}
