\documentclass[12pt]{article}

\usepackage{hyperref}
\usepackage{framed}

\usepackage{graphicx}
\usepackage{amsmath}
%\usepackage{mathtools}
\usepackage{amssymb}
\usepackage{slashed}
\usepackage{bm}
\usepackage{cite}
\usepackage{setspace}
\usepackage{bigints}
\usepackage{color}

\setlength{\oddsidemargin}{0in}
\setlength{\textwidth}{6.5in}
\setlength{\topmargin}{0in}
\setlength{\textheight}{9in}
\voffset-1.5truecm

\def \beq{\begin{equation}}
\def \eeq{\end{equation}}
\def \bea{\begin{eqnarray}}
\def \eea{\end{eqnarray}}
\def \ba{\begin{array}}
\def \ea{\end{array}}
\def \beg{\begin{gather}}
\def \eeg{\end{gather}}
\def \bmat{\begin{matrix}}
\def \emat{\end{matrix}}
\def \bfra{\begin{framed}}
\def \efra{\end{framed}}

\def \({\left(}
\def \){\right)}
\def \[{\left[}
\def \]{\right]}
\def \<{\left\langle}
\def \>{\right\rangle}
\def \lp{\left|}
\def \rp{\right|}
\def \lb{\left\{}
\def \rb{\right\}}
\def \l.{\left.}
\def \r.{\right.}

\def \bma{\(\bmat}
\def \ema{\emat\)}

\def \Tr{{\rm Tr}}
\def \N{\rm N}
\def \hc{{\rm h.\ c.}}

\def \ph{\phantom}
\def \nn{\nonumber}
\def \nl{\nn \\}

\def \of{\frac{1}{4}}
\def \oth{\frac{1}{3}}
\def \hf{\frac{1}{2}}
\def \oet{\frac{1}{8}}
\def \s{\sqrt{2}}
\def \st{\sqrt{3}}
\def \sx{\sqrt{6}}

\def \cB{{\cal B}}
\def \cC{{\cal C}}
\def \cZ{{\cal Z}}
\def \cL{{\cal L}}
\def \cH{{\cal H}}
\def \cO{{\cal O}}
\def \cM{{\cal M}}
\def \cl{\ell}

\def \oq{\overline{q}}
\def \of{\overline{f}}
\def \os{\overline{s}}
\def \ob{\overline{b}}
\def \ou{\overline{u}}
\def \ov{\overline{v}}
\def \oBs{\overline{B^0_s}}
\def \otau{\overline{\tau}}
\def \omu{\overline{\mu}}
\def \onu{\overline{\nu}}
\def \osi{\overline{\si}}
\def \oell{\overline{\ell}}
\def \oL{\overline{L}}
\def \oQ{\overline{Q}}

\def \al{\alpha}
\def \be{\beta}
\def \ga{\gamma}
\def \Ga{\Gamma}
\def \de{\delta}
\def \ka{\kappa}
\def \De{\Delta}
\def \ep{\epsilon}
\def \tha{\theta}
\def \la{\lambda}
\def \La{\Lambda}
\def \si{\sigma}
\def \Si{\Sigma}
\def \Up{\Upsilon}
\def \om{\omega}
\def \1{1}%\mathbb{I}}

\def \pa{\partial}

\def \SM{{\rm SM}}
\def \elm{{\rm em}}
\def \NP{{\rm NP}}
\def \expt{{\rm expt}}
\def \eff{{\rm eff}}
\def \elm{{\rm em}}
\def \RFG{{\rm RFG}}
\def \CFG{{\rm CFG}}
\def \PT{{\rm PT}}
\def \hi{{\rm hi}}
\def \lo{{\rm lo}}

\def \Re{{\rm Re}}
\def \Im{{\rm Im}}

\def \keV{{\rm keV}}
\def \MeV{{\rm MeV}}
\def \GeV{{\rm GeV}}

\def \cH{{\cal H}}
\def \cI{{\cal I}}

\def \Rqp{R_{q'}}

\def\bwt{\begin{widetext}}
\def\ewt{\end{widetext}}

\def\vev#1{\langle {#1} \rangle}
\def\eq#1{eq.~(\ref{#1})}

\def \vp{{\bf p}}
\def \vq{{\bf q}}

\def \cre{\color{red}}
\def \cgr{\color{green}}

% Start of document
% -----------------
\pagestyle{plain}

% Roman Section & Subsection numberings
% -------------------------------------
\renewcommand{\thesection}{\Roman{section}}
\renewcommand{\thesubsection}{\Roman{subsection}}

% Start of document
% -----------------
\pagestyle{plain}
\allowdisplaybreaks

\setlength{\abovecaptionskip}{0pt}
\setlength{\belowcaptionskip}{0pt}

\begin{document}

\begin{center}
\underline{\textbf{\Large Practice with Spinors}}
\end{center}

\section{Algebra with Dirac $\ga$ matrices}

Notation:
\begin{enumerate} \itemsep=-15pt
\item $S= \ou u$\nl
\item $P=\ou \ga^5 u$ \nl
\item $V^\mu = \ou \ga^\mu u$ \nl
\item $A^\mu = \ou \ga^\mu \ga^5 u $ \nl
\item $T^{\mu\nu} = \ou \si^{\mu\nu} u$
\end{enumerate}
\bigskip
\begin{enumerate}

\item$(\ou_1\ga^\mu u_2)^*$ ~=~?

Note: $(\ga^0)^\dag~=~\ga^0$ and $(\ga^\mu)^\dag~=~ \ga^0\ga^\mu\ga^0$

$(\ou_1\ga^\mu u_2)$ is a $1\times1$ matrix. Therefore, its complex conjugate is the same as its Hermitian conjugate, i.e. if we call
$L^\mu = (\ou_1\ga^\mu u_2)$, then $(L^\mu)^* = (L^\mu)^\dag$. We can then express this quantity as follows:
% This is the first equation/set of equations
\bea
L^\mu &=& \ou_1\ga^\mu u_2 ~,~~ \nl
\Rightarrow (L^\mu)^* &=& (L^\mu)^\dag ~,~~ \nl
&=& (\ou_1\ga^\mu u_2)^\dag ~,~~ \nl
&=& ((u_1)^\dag\ga^0\ga^\mu u_2)^\dag ~~~{\rm using}~(A\ldots Z)^\dag = Z^\dag\ldots A^\dag ~,~~ \nl
&=& (u_2^\dag)(\ga^\mu)^\dag(\ga^0)^\dag(u_1)\nl
&=& (u_2^\dag)\ga^0\ga^\mu\ga^0\ga^0(u_1)\nl
&=& (u_2^\dag)\ga^0\ga^\mu(u_1)\nl
&=& \ou_2\ga^\mu(u_1)
\eea
Therefore $(\ou_1\ga^\mu u_2)^*~=~\ou_2\ga^\mu u_1 $. To solve for $|L^\mu|^2$ we simply use $|L^\mu|^2 ~=~ \Tr[\ou_1 \ga^\mu u_2 \ou_2 \ga^\nu u_1]$.


\subsubsection*{Question: Do not erase! Answer it.}
{\cgr $L^\mu ~=~ \ou_1\ga^\mu u_2$ clearly has one Lorentz index $\mu$. But, $|L^\mu|^2 ~=~ \Tr[\ou_1 \ga^\mu u_2 \ou_2 \ga^\nu u_1]$ can be written as some other quantity ${\cal L}^{\mu\nu}$, i.e. it has two Lorentz indices $\mu$ and $\nu$. Why?}\\
The reason is because you cannot have two of the same indices written together in the same line, if so they cannot be contracted due to index notation rules.\\
{\cre But what's wrong with $\Tr[\ou_1 \ga^\mu u_2 \ou_2 \ga_\mu u_1]$? Why increase the number of indices and not decrease it?}\\

If you have a term such as $\ou_1 \ga^\mu u_2 \ou_2 \ga_\mu u_1$ you need to state explicitly that the indices are not summed because that is the convention with an expression such as $\ga^\mu \ga_\mu$. \\

{\cre That is not correct. $\ga^\mu\ga^\mu = \ga^\mu \ga^\nu g_{\mu\nu} = \ga^0 \ga^0 - \ga^1\ga^1 - \ga^2\ga^2 - \ga^3\ga^3$. This is just an expression. The question is why is the left-hand side not the same as the right-hand side?}

Also with the upper and lowering of matrices that implies a change of basis since you cannot arbitrarily raise/lower an index.
{\cre Nothing wrong with raising/lowering indices: see above.}

\bigskip
Note: ~$\Tr[\ga^\mu\ga^\nu]~=~4 g^{\mu\nu}$, $\Tr[\ga^\mu \ga^\nu \ga^\la \ga^\si]= 4(g^{\mu\nu}g^{\la\si} - g^{\mu\la} g^{\nu\si} + g^{\mu\si}g^{\nu\la})$, The trace over the product of an odd number of gamma matrices is zero.
\bea
|L^{\mu}|^2 &=& \Tr[\ou_1 \ga^\mu u_2 \ou_2 \ga^\nu u_1] \nl
&=& \Tr[\ou_1 \ga^\mu (\slashed{p}_2+m) \ga^\nu u_1]\nl
&=& \Tr[u_1\ou_1 \ga^\mu (\slashed{p}_2+m) \ga^\nu]\nl
&=& \Tr[(\slashed{p}_1+m) \ga^\mu (\slashed{p}_2+m) \ga^\nu]\nl
&=& \Tr[\slashed{p}_1\ga^\mu \slashed{p}_2 \ga^\nu] + m[\Tr(\ga^\mu \slashed{p}_1\ga^\nu) + \Tr(\ga^\mu \ga^\nu \slashed{p}_2)] + m^2\Tr[\ga^\mu \ga^\nu]\nl
&=& \Tr[\slashed{p}_1\ga^\mu \slashed{p}_2 \ga^\nu] + m^2\Tr[\ga^\mu \ga^\nu]\nl
&=& \Tr[(p_1)_\la\ga^\la \ga^\mu (p_2)_\si\ga^\si \ga^\nu  ] + 4m^2g^{\mu\nu}\nl
&=& (p_1)_\la (p_2)_\si \Tr[\ga^\la \ga^\mu\ga^\si \ga^\nu  ] + 4m^2g^{\mu\nu}\nl
&=& (p_1)_\la(p_2)_\si 4(g^{\mu\nu}g^{\la\si} - g^{\mu\la} g^{\nu\si} + g^{\mu\si}g^{\nu\la}) + 4m^2g^{\mu\nu}\nl
&=& 4[p_1^\mu p_2^\nu - g^{\mu\nu}(p_1 \cdot p_2) + p_2^\mu p_1^\nu] + 4m^2g^{\mu\nu} \nl
&=& 4[p_1^\mu p_2^\nu - 4g^{\mu\nu}(p_1 \cdot p_2 + m^2) + p_2^\mu p_1^\nu]
\eea

\item $(\ou_1\ga^\mu \ga^5 u_2)^*$ is also a $1\times1$ Matrix so the same reasoning applies as above in 1. Note: $(\ga^5)^\dag = \ga^5$
We define: $R^\mu$ as $\ou_1\ga^\mu\ga^5 u_2$ thus:
% This is the second equation/set of equations
\bea
(R^\mu)^* &=& (R^\mu)^\dag \nl
&=&(\ou_1\ga^\mu\ga^5 u_2)^\dag \nl
&=& ((u_1)^\dag\ga^0\ga^\mu\ga^5 u_2)^\dag \nl
&=& (u_2^\dag)(\ga^5)^\dag(\ga^\mu)^\dag(\ga^0)^\dag(u_1) \nl
&=& (u_2^\dag)\ga^5\ga^0\ga^\mu\ga^0\ga^0 u_1 \nl
&=& (u_2^\dag)\ga^5\ga^0\ga^\mu(1) u_1 \nl
&=& -(u_2^\dag)\ga^0\ga^5\ga^\mu u_1 \nl
&=& -\ou_2\ga^5\ga^\mu u_1 \nl
&=& \ou_2\ga^\mu\ga^5 u_1
\eea
%
Therefore $(\ou_1\ga^\mu \ga^5 u_2)^*~=~\ou_2 \ga^\mu \ga^5 u_1$

We also are able to calculate $|R^\mu|^2$
\bea
|R^{\mu}|^2 &=& \Tr[(\ou_1\ga^\mu \ga^5 u_2)(\ou_2 \ga^\nu \ga^5 u_1)] \nl
&=& \Tr[\ou_1\ga^\mu \ga^5 (\slashed{p}_2+m) \ga^\nu \ga^5 u_1]  \nl
&=&\Tr[u_1\ou_1\ga^\mu \ga^5 (\slashed{p}_2+m) \ga^\nu \ga^5 ]\nl
&=&\Tr[(\slashed{p}_1+m)\ga^\mu \ga^5 (\slashed{p}_2+m) \ga^\nu \ga^5 ]\nl
&=& \Tr[\slashed{p}_1\ga^\mu\ga^5 \slashed{p}_2\ga^\nu \ga^5 + m( \slashed{p}_1\ga^\mu\ga^5\ga^\nu \ga^5 +\ga^\mu\ga^5 \slashed{p}_2\ga^\nu \ga^5) + m^2(\ga^\mu\ga^5\ga^\nu \ga^5)] \nl
&=&\Tr[\slashed{p}_1\ga^\mu\ga^5 \slashed{p}_2\ga^\nu \ga^5 + m^2(\ga^\mu\ga^5\ga^\nu \ga^5)] \nl
&=& \Tr[(p_1)_\la\ga^\la\ga^\mu\ga^5 (p_2)_\si\ga^\si\ga^\nu \ga^5] + m^2\Tr[\ga^\mu\ga^5\ga^\nu \ga^5]\nl
&=& (p_1)_\la(p_2)_\si \Tr[\ga^\la\ga^\mu\ga^5 \ga^\si\ga^\nu \ga^5] - m^2\Tr[\ga^\mu\ga^5\ga^5 \ga^\nu]\nl
&=& (p_1)_\la(p_2)_\si \Tr[\ga^\la\ga^\mu\ga^5\ga^5 \ga^\si\ga^\nu] - m^2\Tr[\ga^\mu\ga^\nu]\nl
&=& (p_1)_\la(p_2)_\si \Tr[\ga^\la\ga^\mu\ga^\si\ga^\nu] - m^2(g^{\mu\nu})\nl
&=& (p_1)_\la(p_2)_\si 4(g^{\mu\nu}g^{\la\si} - g^{\mu\la} g^{\nu\si} + g^{\mu\si}g^{\nu\la}) - 4m^2g^{\mu\nu}\nl
&=& 4[p_1^\mu p_2^\nu - g^{\mu\nu}(p_1 \cdot p_2) + p_2^\mu p_1^\nu] - 4m^2g^{\mu\nu} \nl
&=& 4[p_1^\mu p_2^\nu - 4g^{\mu\nu}(p_1 \cdot p_2 - m^2) + p_2^\mu p_1^\nu]
\eea

\item $ (\ou_1 u_2)^* ~=~ ?$
We let:$P ~=~ \ou_1 u_2$
\bea
(P)^* &=& (P)^\dag \nl
&=& (\ou_1 u_2)^\dag ~,~~ \nl
&=& ((u_1)^\dag\ga^0 u_2)^\dag \nl
&=& (u_2)^\dag(\ga^0)^\dag(u_1)\nl
&=& (u_2)^\dag\ga^0(u_1)\nl
&=& \ou_2(u_1)
\eea
Therefore$(\ou_1 u_2)^* ~=~ \ou_2 u_1$.
In order to find $|P|^2$ we simply do the following:
\bea
|P|^2&=& \Tr[\ou_1 u_2 \ou_2 u_1]  \\
&=& \Tr[(\slashed{p}_1+m)(\slashed{p}_2+m)] \nl
&=& \Tr[\slashed{p}_1\slashed{p}_2+m(\slashed{p}_1+\slashed{p}_2)+m^2]\nl
&=& \Tr[\slashed{p}_1\slashed{p}_2]+\Tr[m(\slashed{p}_1+\slashed{p}_2)]+\Tr[m^2] \nl
&=& \Tr[\slashed{p}_1\slashed{p}_2]+m(\Tr[\slashed{p}_1]+\Tr[\slashed{p}_2])+m^2\Tr[1] \nl
&=& \Tr[\slashed{p}_1\slashed{p}_2]+4m^2 \nl
&=& 4(p_1 \cdot p_2)+4m^2
\eea

\item By the same reasoning as shown above it can be shown that $(\ou_1 \ga^5 u_2)^* ~=~ \ou_2 \ga^5 u_1$ \\
If we let $T ~=~ \ou_1\ga^5 u_2$ then:
\bea
\Rightarrow (T)^* &=& (T)^\dag  \\\
&=& (\ou_1\ga^5 u_2)^\dag ~,~~ \nl
&=& ((u_1)^\dag\ga^0\ga^5 u_2)^\dag \nl
&=& (u_2)^\dag(\ga^5)^\dag (\ga^0)^\dag(u_1)\nl
&=& (u_2)^\dag(-\ga^5) \ga^0(u_1)\nl
&=& (u_2)^\dag \ga^0\ga^5(u_1)\nl
&=& \ou_2\ga^5(u_1)
\eea
Therefore  $(\ou_1 \ga^5 u_2)^* ~=~ \ou_2 \ga^5 u_1$

In order to square $T$ we do the following:
\bea
|T|^2&=& \Tr[\ou_1 \ga^5 u_2\ou_2 \ga^5 u_1] \\
&=& \Tr[u_1 \ou_1 \ga^5 \slashed{p}_2 \ga^5] \nl
&=& \Tr[\slashed{p}_1 \ga^5 \slashed{p}_2 \ga^5] \nl
&=& \Tr[(p_1)_\mu \ga^\mu \ga^5 (p_2)_\nu \ga^\nu \ga^5] \nl
&=& (p_1)_\mu(p_2)_\nu \Tr[\ga^\mu\ga^5\ga^\nu\ga^5] \nl
&=& -(p_1)_\mu(p_2)_\nu \Tr[\ga^\mu\ga^5\ga^5\ga\nu]\nl
&=& -(p_1)_\mu(p_2)_\nu \Tr[\ga^\mu\ga^\nu]\nl
&=& -(p_1)_\mu(p_2)_\nu (4g^{\mu\nu})\nl
&=& -4(p_1)(p_2) \nl
\eea

\item While the above identities could be shown to be trivial, the identity: $(\ou_1 \si^{\mu\nu} u_2)^* ~=~\ou_2 \si^{\mu\nu} u_1$ is more difficult to solve. The identity:$(\si^{\mu\nu})^\dag =\si^{\mu\nu}$ is needed
\bea
(\si^{\mu\nu})^\dag &=& (\frac{i}{2}[\ga^\mu,\ga^\nu])^\dag \\
&=&(u_2)^\dag (\si^{\mu\nu})^\dag (\ga^0)^\dag u_1 \nl
&=& \frac{i}{2}([\ga^\mu,\ga^\nu])^\dag\nl
&=& \frac{i}{2}(\ga^\mu\ga^\nu-\ga^\nu\ga^\mu)^\dag \nl
&=& \frac{i}{2}((\ga^\mu)^\dag(\ga^\nu)^\dag-(\ga^\nu)^\dag(\ga^\mu)^\dag) \nl
&=& \frac{i}{2}(\ga^0\ga^\mu\ga^0\ga^0\ga^\nu\ga^0-\ga^0\ga^\nu\ga^0\ga^0\ga^\mu\ga^0)\nl
&=& \frac{i}{2}(\ga^0\ga^\mu\ga^\nu\ga^0-\ga^0\ga^\nu \ga^\mu\ga^0)\nl
&=& \frac{i}{2}((-1)^2\ga^\mu\ga^\nu-(-1)^2\ga^\nu \ga^\mu) \nl
&=& \frac{i}{2}(\ga^\mu\ga^\nu-\ga^\nu \ga^\mu)\nl
&=& \si^{\mu\nu}
\eea
After showing $(\si^{\mu\nu})^\dag =\si^{\mu\nu}$ is true it is trivial to show $(\ou_1 \si^{\mu\nu} u_2)^* ~=~\ou_2 \si^{\mu\nu} u_1$.
We let $B^{\mu\nu} ~=~ \ou_1 \si^{\mu\nu} u_2$
\bea
(B^{\mu\nu})^* &=& (B^{\mu\nu})^\dag ~,~~ \\
&=& (\ou_1 \si^{\mu\nu} u_2)^\dag \nl
&=& ((u_1)^\dag \ga^0 \si^{\mu\nu} u_2)^\dag \nl
&=& (u_2^\dag)(\si^{\mu\nu})^\dag (\ga^0)^\dag(u_1)\nl
&=& (u_2^\dag)\si^{\mu\nu}\ga^0(u_1)\nl
&=& (u_2^\dag)(\ga^0)\si^{\mu\nu}(u_1)\nl
&=& \ou_2 \si^{\mu\nu}(u_1)
\eea

An interesting thing to note is that the expression $\ou \si^{\mu\nu} \ga^5 u$ is not an independent quantity. Since $\ga^5 =i\ga^0\ga^1\ga^2\ga^3$ it follows that the product of $\si^{\mu\nu}$ and $\ga^5$ can be simplified to an expression with only 2 $\ga$ matrices which has been defined as a pseudoscalar. For example, let $\mu =0$ and $\nu=1$:
\bea
\ou \si^{01} \ga^5 u &=& \ou \si^{01} (i\ga^0\ga^1\ga^2\ga^3) u \\
&=&  \ou ((\frac{i}{2})(\ga^0\ga^1 -\ga^1\ga^0)) (i\ga^0\ga^1\ga^2\ga^3) u \nl
&=&  \ou (\frac{i}{2})[\ga^0\ga^1(i\ga^0\ga^1\ga^2\ga^3) -\ga^1\ga^0(i\ga^0\ga^1\ga^2\ga^3)] u \nl
&=&  \ou (\frac{-1}{2})[\ga^0\ga^1\ga^0\ga^1\ga^2\ga^3 -\ga^1\ga^0\ga^0\ga^1\ga^2\ga^3)] u \nl
&=& \ou (\frac{-1}{2})[-\ga^2\ga^3 -\ga^2\ga^3)] u\nl
&=& \ou (\frac{-1}{2})[-2\ga^2\ga^3] u\nl
&=& 2\ou \ga^2\ga^3 u
\eea
This expression (because it contains two gamma matrices) is a pseudoscalar. Any values of $\mu$ and $\nu$ can be shown to be similar to this because of the communal and identity properties of the gamma matrices.

In order to find the value of $|B^{\mu\nu}|^2$ one needs to do the following:
\bea
|B^{\mu\nu}|^2 &=& \Tr[\ou_1 \si^{\mu\nu} u_2\ou_2 \si^{\si\la}u_1]\nl
&=& \Tr[\ou_1 \si^{\mu\nu}\slashed{p}_2 \si^{\si\la} u_1]\nl
&=& \Tr[\slashed{p}_1\si^{\mu\nu}\slashed{p}_2 \si^{\si\la}]\nl
&=& \Tr[p_\ka \ga^\ka \si^{\mu\nu}p_\ga \ga^\ga \si^{\si\la}]\nl
&=& p_\ka p_\ga \Tr[\ga^\ka \si^{\mu\nu}\ga^\ga \si^{\si\la}]\nl
&=& -\frac{1}{2} p_\ka p_\ga \Tr[\ga^\ka (\ga^\mu \ga^\nu - \ga^\nu \ga^\mu) \ga^\ga (\ga^\si \ga^\la - \ga^\la \ga^\si)]\nl
&=& -\frac{1}{2} p_\ka p_\ga \Tr[2\ga^\ka \ga^\mu \ga^\nu \ga^\ga \ga^\si \ga^\la  - 2\ga^\ka \ga^\nu \ga^\mu \ga^\ga \ga^\si \ga^\la]\nl
&=& -\frac{1}{2} p_\ka p_\ga \Tr[4\ga^\ka \ga^\mu \ga^\nu \ga^\ga \ga^\si \ga^\la]\nl
&=& -2p_\ka p_\ga \Tr[\ga^\ka \ga^\ga]~{\rm Using~ a~ similar~ identity~ as~ shown~ in~ eq.~ 23}\nl
&=& -8p_\ka p_\ga g^{\ka\ga}
\eea

\end{enumerate}
\end{document}
