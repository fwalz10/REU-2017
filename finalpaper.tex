\documentclass[12pt]{article}

\usepackage{hyperref}
\usepackage{framed}

\usepackage{graphicx}
\usepackage{amsmath}
%\usepackage{mathtools}
\usepackage{amssymb}
\usepackage{slashed}
\usepackage{bm}
\usepackage{cite}
\usepackage{setspace}
\usepackage{bigints}
\usepackage{color}
\usepackage{siunitx}

\setlength{\oddsidemargin}{0in}
\setlength{\textwidth}{6.5in}
\setlength{\topmargin}{0in}
\setlength{\textheight}{9in}
\voffset-1.5truecm

\def \beq{\begin{equation}}
\def \eeq{\end{equation}}
\def \bea{\begin{eqnarray}}
\def \eea{\end{eqnarray}}
\def \ba{\begin{array}}
\def \ea{\end{array}}
\def \beg{\begin{gather}}
\def \eeg{\end{gather}}
\def \bmat{\begin{matrix}}
\def \emat{\end{matrix}}
\def \bfra{\begin{framed}}
\def \efra{\end{framed}}

\def \({\left(}
\def \){\right)}
\def \[{\left[}
\def \]{\right]}
\def \<{\left\langle}
\def \>{\right\rangle}
\def \lp{\left|}
\def \rp{\right|}
\def \lb{\left\{}
\def \rb{\right\}}
\def \l.{\left.}
\def \r.{\right.}

\def \bma{\(\bmat}
\def \ema{\emat\)}

\def \Tr{{\rm Tr}}
\def \N{\rm N}
\def \hc{{\rm h.\ c.}}

\def \ph{\phantom}
\def \nn{\nonumber}
\def \nl{\nn \\}

\def \of{\frac{1}{4}}
\def \oth{\frac{1}{3}}
\def \hf{\frac{1}{2}}
\def \oet{\frac{1}{8}}
\def \s{\sqrt{2}}
\def \st{\sqrt{3}}
\def \sx{\sqrt{6}}

\def \cB{{\cal B}}
\def \cC{{\cal C}}
\def \cZ{{\cal Z}}
\def \cL{{\cal L}}
\def \cH{{\cal H}}
\def \cO{{\cal O}}
\def \cM{{\cal M}}
\def \cl{\ell}

\def \oq{\overline{q}}
\def \of{\overline{f}}
\def \os{\overline{s}}
\def \ob{\overline{b}}
\def \ou{\overline{u}}
\def \ov{\overline{v}}
\def \oBs{\overline{B^0_s}}
\def \otau{\overline{\tau}}
\def \omu{\overline{\mu}}
\def \onu{\overline{\nu}}
\def \osi{\overline{\si}}
\def \oell{\overline{\ell}}
\def \oL{\overline{L}}
\def \oQ{\overline{Q}}

\def \al{\alpha}
\def \be{\beta}
\def \ga{\gamma}
\def \Ga{\Gamma}
\def \de{\delta}
\def \ka{\kappa}
\def \De{\Delta}
\def \ep{\epsilon}
\def \tha{\theta}
\def \la{\lambda}
\def \La{\Lambda}
\def \si{\sigma}
\def \Si{\Sigma}
\def \Up{\Upsilon}
\def \om{\omega}
\def \1{1}%\mathbb{I}}

\def \pa{\partial}

\def \SM{{\rm SM}}
\def \elm{{\rm em}}
\def \NP{{\rm NP}}
\def \expt{{\rm expt}}
\def \eff{{\rm eff}}
\def \elm{{\rm em}}
\def \RFG{{\rm RFG}}
\def \CFG{{\rm CFG}}
\def \PT{{\rm PT}}
\def \hi{{\rm hi}}
\def \lo{{\rm lo}}

\def \Re{{\rm Re}}
\def \Im{{\rm Im}}

\def \keV{{\rm keV}}
\def \MeV{{\rm MeV}}
\def \GeV{{\rm GeV}}

\def \cH{{\cal H}}
\def \cI{{\cal I}}

\def \Rqp{R_{q'}}

\def\bwt{\begin{widetext}}
\def\ewt{\end{widetext}}

\def\vev#1{\langle {#1} \rangle}
\def\eq#1{eq.~(\ref{#1})}

\def \vp{{\bf p}}
\def \vq{{\bf q}}

\def \cre{\color{red}}
\def \cgr{\color{green}}

\usepackage{graphicx} %package to manage images
\graphicspath{ {images/} }

\usepackage[rightcaption]{sidecap}

\usepackage{wrapfig}

% Start of document
% -----------------
\pagestyle{plain}
\allowdisplaybreaks

\setlength{\abovecaptionskip}{0pt}
\setlength{\belowcaptionskip}{0pt}

\begin{document}
\begin{titlepage}
    \begin{center}
        \vspace*{1cm}
        
        \Huge
        \textbf{The $R_K$ and $R_{D^{(*)}}$ Puzzles}
        
        \vspace{0.5cm}
        \large
        
        \vspace{1.5cm}
        
        Francis Walz\footnote{frankwalz10@gmail.com}\\ Towson University
        
        \vfill
        
        \vspace{0.8cm}
        
        \Large
        2017 Summer REU\\
        Wayne State University\\
        Department of Physics and Astronomy\\
        8/10/17
        
    \end{center}
\end{titlepage}
\tableofcontents
\newpage
\section{Introduction}
The standard model of particle physics has been an excellent model to describe decay rates and different types of processes that occur at the subatomic level. However, recent experimental findings have challenged the notion that the standard model is the only model to predict what we see in the real world and in experiments. Included in the standard model are particles called quarks, which have $\frac{1}{2}$ spin and either a $\frac{2}{3}$ or $-\frac{1}{3}$ charge. These quarks make up what we know as matter due to their interaction with the Higgs. 
\begin{table}[h!]
\centering
\begin{tabular}{ |p{3cm}||p{3cm}|p{3cm}|p{3cm}|  }
 \hline
 \multicolumn{4}{|c|}{Quarks} \\
 \hline
 Type & Mass & Charge & Spin\\
 \hline
 $u$ & $2.2 ^{+0.6}_{\num{-0.4}}$ MeV & $\frac{2}{3}$ & $\frac{1}{2}$\\ [1ex]
 $d$ &  $4.7 ^{+0.5}_{\num{-0.4}}$ MeV & $-\frac{1}{3}$   & $\frac{1}{2}$\\ [1ex]
 $s$ &$96 ^{+8}_{\num{-4}}$ MeV & $-\frac{1}{3}$ & $\frac{1}{2}$\\[1ex]
 $c$ & $1.28 \pm 0.03$ GeV &$\frac{2}{3}$ & $\frac{1}{2}$\\[1ex]
 $b$ & $4.18 ^{+0.04}_{\num{-0.03}}$ GeV  &$-\frac{1}{3}$ & $\frac{1}{2}$\\[1ex]
 $t$ & $173.1 \pm 0.6$ GeV &$\frac{2}{3}$ & $\frac{1}{2}$\\[1ex]
 \hline 
\end{tabular}
\caption{Table of Quarks in the SM \cite{Agashe:2014kda}}
\label{table:1}
\end{table}\\
However, the question remains, why are there differences to the masses of quarks? To date, the SM has not been able to explain this phenomenon. Thus, by exploring the properties and characteristics of quarks, we may be able to probe into the possibility of new physics. The way in which we can know about the possibility new physics is to piece together scattering processes after collisions. Since a particle always decays to a lighter particle, due to the lighter particle being more stable than the heavier ones, we can look at the processes that include the heaviest quarks to see if there are additional physics beyond the SM. A excellent candidate of these tests would be the $B$ meson. A $B$ meson is made of two different quarks but always includes the $b$ quark. In this paper we will look at both the $B^+$ and $B_s$ mesons which are $b\bar{u}$ and $s\bar{b}$ respectively. \cite{Griffiths} (The bar above the quark indicates the anti quark, which is the same as the quark in all ways except having the opposite charge) The certain decays of the $B$ meson could give us some hints into the new physics present beyond the standard model because a possible new quark must decay into a $b$ quark. The realization of this fact lead to the creation of ``$B$ factories". These ``$B$ factories" focus on creating $B$ mesons and analyzing their decays. Two of these factories are BaBar and Belle, which provides some of the data included in this article. The $R_K$ and $R_{D^{(*)}}$ are two ratios of these $B$ meson decays that show some puzzling experimental results. The $R_K$ ratio is defined as $\mathcal{B}(B^+\rightarrow K^+\mu^+\mu^-)/ \mathcal{B}(B^+\rightarrow K^+e^+e^-)$ The experimental value found by the LHCb Collaboration is \cite{Aaij:2014ora}
\begin{center}
$R^{expt}_{K} = 0.745 _{\num{-0.074}}^{+0.090}$ (stat) $\pm 0.036$ (syst)
\end{center}
Which differs from the SM prediction of $R^{expt}_{K} = 1 \pm 0.01$ differs from the standard model prediction by $2.6\si$. \cite{Bordone:2016gaq} The next puzzle is the $R_{D^{(*)}}$ ratio which is defined as $\mathcal{B}(\bar{B} \rightarrow D^{(*)}\tau^- \onu_{\tau}/ \mathcal{B}(\bar{B} \rightarrow D^{(*)}l^- \onu_{l}$ where $(l= \mu$ or $e)$ this ratio has been measured by the LHCb  Collaboration, BaBar, and Belle. The experimental values of $R_D$ and $R_{D^*}$The ratios are as follows \cite{Lees:2013uzd, Huschle:2015rga, Aaij:2015yra} :
\begin{center}
$R_D= 1.29 \pm 0.17$ , $R_{D^*} = 1.28 \pm 0.09$
\end{center}$R_D$ differs from the standard model by $1.7\si$ and the $R_{D^*}$ differs by $3.1\si$. We will focus on analyzing the possible NP contributions to the Wilson Coefficients arising from the effective Hamiltonian of the $b \rightarrow s\mu^+\mu^-$ transition. The complete Hamiltonian is as follows:\cite{Bhattacharya:2016mcc}. 
\begin{center}
\bea
H = -\frac{\alpha G_f}{\sqrt{2}\pi} V_{tb}V^*_{ts}\sum_{a= 9,10}^{}(C_aO_a + C'_aO'_a)
\eea
\end{center}
This article will take the Hamiltonian and use the Wilson Coefficients found in the branching ratio of the $B_s \rightarrow \mu^+ \mu^- $ decay. The paper begins by first exploring and computing different observable and particulars of the $\pi^+ \rightarrow l^+ \nu_l $ decay. (Where $l= \mu$ or $e$) This will provide us with a background in manipulating and understanding the theoretical expression for branching ratios and decays. It will also serve the reader as a guide to the format of the rest of the paper. In section 3 the Wilson coefficients from the branching ratio of the $B_s \rightarrow \mu^+\mu^-$ decay will be extracted and constrained with the experimental numbers. We will use the Python package Flavio\cite{flavio} to generate our data and construct our plots. Flavio's packages focus on flavor physics and allows the user to find the contributions of different NP Wilson Coefficients for varying observable. Finally in section 4 plots of the Branching ratio based upon the Wilson Coefficients will be shown and fits to the data will be provided. We conclude the article in section 5. 
\section{The branching Ratio of a Charged Pion decay}
In order to calculate the Branching ratio of a charged pion, we will use the Feynman Rules of calculating amplitudes and the trace identities. \\
\\
We start by with recognizing that the pion decay is a charged cuurent interaction which arises from the fact that a pion is made of quarks, and the decay is mediated by a massive W boson. \\
\\
A diagram of the decay may be seen below: 
\begin{figure}[h]
\centering
\includegraphics[scale=.75]{piondecay}
\caption{Pion Decay (where $\ell$ is $\mu$ or $e$)}
\end{figure}
Where the up and ani-down quark are the pion and the muon and muon neutrino are on the right. However, the muon and the muon neutrino could be the electron and electron neutrino.  \\
\\
The formula given to describe the branching ratio of the decay is given by\cite{Griffiths}: 
\bea 
\Ga &=& \frac{S|\textbf{p}|}{8\pi\hbar m_1^2c} |\mathcal{M}|^2
\eea 
Where $|\textbf{p}|$ is the outgoing momentum, $S$ is the product of statistical factors (in our case it will be equal to 1), and $\mathcal{M}$ is the Feynman amplitude. In our notation, we will use the Natural Units so our expression becomes: 
\bea 
\Ga &=& \frac{S|\textbf{p}|}{8\pi m_1^2} |\mathcal{M}|^2
\eea 
The next step is to determine the Feynman Amplitude which can be done in a few steps 
In the charged weak lepton decays, we have different notation for the vertices and propagators. 
\begin{enumerate}
    \item For each vertex add a factor of $\frac{-ig_w}{2\sqrt{2}}(\ga^\nu(1-\ga^5))$ where $g_w= \sqrt{4\pi\al_w}$
    \item For each propagator we add a factor of $\frac{-ig_{\mu\nu}-\frac{q_\mu q_\nu}{m^2}}{q^2-m^2}$ where m is the mass of the boson. In our case, $m_w \gg\ q$ so the expression simplifies to $\frac{ig_{\mu\nu}}{m_w^2}$
\end{enumerate}
From these rules and Figure 1 we are able to calculate the value of $\mathcal{M}$ 
\bea
-i\mathcal{M}&=&\bigg[\ou(3)\bigg(\frac{-ig_w}{2\sqrt{2}}(\ga^\nu(1-\ga^5)\bigg)v(2)\bigg] \bigg[\frac{ig_{\mu\nu}}{m_w^2} \bigg]\bigg[\frac{-ig_w}{2\sqrt{2}}F^\mu\bigg]
\eea
We have a factor of -i with the $\mathcal{M}$ so that we obtain the real part of the expression and $F^\mu$ is the form factor of the coupling of the pion to the $W$ boson. $F^\mu$ has the form of $f_\pi p^\mu$ 
\bea
\mathcal{M}&=&\frac{g_w^2}{8 m_w^2}[\ou(3)(\ga^\mu(1-\ga^5))v(2)]F^\mu
\eea
In order to square the amplitude we do the following: 
\bea
\langle|\mathcal{M}^2|\rangle&=&\bigg(\frac{g_w^2}{8 m_w^2}f_\pi\bigg)^2Tr[(\ou(3)(\ga^\mu(1-\ga^5))v(2)p_\mu(\ov(2)(\ga^\nu(1-\ga^5)u(3))]p_\nu\\
\langle|\mathcal{M}^2|\rangle&=&\bigg(\frac{g_w^2}{8m_w^2}f_\pi\bigg)^2p_\mu p_\nu (8[p_3^\mu p_2^\nu + p_2^\mu p_3^\nu -(p_3 \cdot p_2)g^{\mu\nu}]+8i\epsilon^{\mu\la\nu\si}p_{3\la}p_{2\si}
\eea
Summing over the spins gives us: 
\bea 
\langle|\mathcal{M}^2|\rangle&=&8\bigg(\frac{g_w^2}{8m_w^2}f_\pi\bigg)^2 [2(p_1\cdot p_2)(p_1 \cdot p_3) -p^2(p_2 \cdot p_3)] 
\eea 
Since $p=p_2 + p_3$, we can simplify the equation further\\
For simplicity and consistency we will use the following notation: $p_1 = p_\pi , p_2=p_l, p_3=p_{\nu_l}$\\ 
We begin with showing the value of the 4-momentum squared: 
\bea 
p_1 &=& (E, \vec{p}_1) \\
(p_1)^2 &=& p_\mu p_\nu g_{\mu\nu}\nl
(p_1)^2 &=& p_1p_1(1) + p_2p_2(-1) + ...\nl
(p_1)^2 &=& E^2 - (\vec{p}_1)^2\nl 
(p_1)^2 &=& m_1^2
\eea 
This can be also shown to be true for the other 4-Momenta thus:\\
$(p_1)^2 =(p_\pi)^2 = m_\pi^2 , (p_2)^2 = (p_l)^2 = m_l^2 , (p_3)^2= (p_{\nu_l})^2 = (m_{\nu_l})^2 = 0$\\
Using this we can further simplify Equation 7:
\bea 
\frac{1}{2}[(m_\pi)^2 -(m_l)^2] &=& (p_2 \cdot p_3)\\
\frac{1}{2}[(m_\pi)^2 -(m_l)^2] &=& (p_1 \cdot p_3)\\
\frac{1}{2}[(m_\pi)^2 +(m_l)^2] &=& (p_1 \cdot p_2)
\eea 
Returning to Equation 7 we now have
\bea 
\langle|\mathcal{M}^2|\rangle&=&8\bigg(\frac{g_w^2}{8m_w^2}f_\pi\bigg)^2[\frac{1}{2}(m_l)^2((m_\pi)^2- (m_l)^2)]
\eea 
In this way we are able to calculate the branching ratio of a pion, since we know have the Feyneman Amplitude we simply return to equation (112)
\bea 
\Ga &=& \frac{S|\textbf{p}|}{8\pi m_1^2} |\mathcal{M}|^2\\
\Ga &=& \frac{S|\textbf{p}|}{\pi m_\pi^2}\bigg(\frac{g_w^2}{8m_w^2}f_\pi\bigg)^2\bigg[\frac{1}{2}(m_l)^2((m_\pi)^2 -(m_l)^2)\bigg]
\eea 
We can remove S because in this case S=1\\
We then need to find the value of $|\textbf{p}|$\\
\bea
|\textbf{p}| &=& \frac{\sqrt{(m_1+m_2+m_3)(m_1-m_2-m_3)(m_1+m_2-m_3)(m_1-m_2+m_3)}}{2m_1}\\
|\textbf{p}| &=& \frac{\sqrt{m_1^4+m_2^4 + m_3^4 -2m_1^2m_2^2 - 2m_1^2m_3^2 - 2m_2^2m_3^2}}{2m_1}\nl
|\textbf{p}| &=& \frac{\sqrt{m_1^4+m_2^4-m_1^2m_2^2}}{2m_1}
\eea
(Because the neutrino is massless)
\bea
|\textbf{p}| &=& \frac{\sqrt{(m_1^2-m_2^2)^2}}{2m_1}
\eea 
For our case $m_1=m_\pi$ and $m_2=m_l$
\bea
|\textbf{p}| &=& \frac{(m_\pi^2-m_l^2)}{2m_\pi}
\eea
Combining with Equation 14: 
\bea 
\Ga &=& \frac{|\textbf{p}|}{2\pi m_1^2}\bigg(\frac{g_w^2}{8m_w^2}f_\pi\bigg)^2\bigg[(m_l)^2((m_\pi)^2 -(m_l)^2)\bigg]\\
\Ga &=& \frac{1}{4\pi}\bigg(\frac{g_w^2f_\pi}{8m_w^2}\bigg)^2\frac{1}{m_\pi^3}m_l^2(m_\pi^2-m_l^2)^2
\eea
Because $\frac{g_w^2}{8m_w^2}= \frac{G_f}{\sqrt{2}}$ we are able to say:
\bea
\Ga &=& \frac{1}{8\pi}(G_f f_\pi)^2\frac{1}{m_\pi^3}m_l^2(m_\pi^2-m_l^2)^2
\eea 
Expanding on this idea, we are able to graph $\frac{\Ga_l}{\Ga_\pi}$ by: 
\bea
\Ga &=& \frac{1}{8\pi}(G_f f_\pi)^2m_\pi^3\bigg(\frac{m_l}{m_\pi}\bigg)^2\bigg(1-\bigg(\frac{m_l}{m_\pi}\bigg)^2\bigg)^2
\eea
In order to find $\frac{\Ga_{l}}{\Ga_\pi}$ we need to divide the above expression by $\frac{1}{\tau_\pi}$ but normalize it with $\hbar$ so our final expression is: 
\bea
\frac{\Ga_l}{\Ga_\pi}&=& \frac{1}{8\pi}(G_f f_\pi)^2m_\pi^3\bigg(\frac{\tau_\pi}{\hbar}\bigg)\bigg(\frac{m_l}{m_\pi}\bigg)^2\bigg(1-\bigg(\frac{m_l}{m_\pi}\bigg)^2\bigg)^2
\eea 
\newpage
The graph of the equation (with $m_l = m_e$) is:
\\
\begin{figure}[h]
\centering
\includegraphics[scale=.90]{brpion1}
\caption{Graph of $\frac{\Ga_{m_e}}{\Ga_\pi}$ and the plots of $\frac{m_e}{m_\pi}$ and the value of $Br(\pi^{-} \rightarrow e^{-} + \nu_e)$}
\end{figure}
\begin{figure}[h]
\centering
\includegraphics[scale=.85]{brpionmu1}
\caption{Graph of $\frac{\Ga_{m_\mu}}{\Ga_\pi}$ and the plots of $\frac{m_\mu}{m_\pi}$ and the value of $Br(\pi^{-} \rightarrow \mu^{-} + \nu_\mu)$}
\end{figure}
\\
\\
\\
\\
\\
\\
\\
\\
\\
\\
\\
\\
\\
\\
\\
\\
\\
Using the values from PDG \cite{Agashe:2014kda} we can do calculations with the branching ratios. \\
\begin{center}
 \begin{tabular}{||c c c c||} 
 \hline
 Observables & $e$ & $\mu$ & $\pi$  \\ [0.5ex] 
 \hline\hline
 $\tau$ & $6.6 \times 10^{28}$ yr &$2.1969811(22) \times 10^-6$ s  & $2.6033(5) \times 10^-8$ s \\ 
 \hline
 Mass(MeV) & $0.5109989461(31)$ & $105.6583745(24)$  & $139.57061(24)$ \\[1ex] 
 \hline
\end{tabular}
\end{center}
If we would like to calculate the ratio of the $\pi^{-} \rightarrow e^{-} + \nu_e$ and  $\pi^{-} \rightarrow \mu^{-} + \nu_\mu$ we simply do the following: 
\bea 
\frac{\Ga_e}{\Ga_\mu} &=& \frac{m_e^2(m_\pi^2-m_e^2)^2}{m_\mu^2(m_\pi^2-m_\mu^2)^2}\\
\frac{\Ga_e}{\Ga_\mu} &=& 1.28334(73) \times 10^{-4}
\eea 
This is an interesting observation because the value of $ \frac{\Ga_e}{\Ga_\mu}$ suggests that the probability of $\pi^{-} \rightarrow \mu^{-} + \nu_\mu$ is higher than $\pi^{-} \rightarrow e^{-} + \nu_e$. This is somewhat striking because the mass of a muon is greater than the mass of an electron, indicating that the pion does not decay into the lightest particle most frequently. 
\section{The Branching Ratio of the $B_s \rightarrow \mu^+\mu^-$ decay}
Following the procedure outlined above and comparing to \cite{Dighe:2012df} we are able to find an expression for the decay rate of the $B_s \rightarrow \mu^+\mu^-$ with some coefficients of new physics included.\\
The decay rate is as follows:
\begin{center}
\bea
\mathcal{B}(B_s\rightarrow \mu^+\mu^-) = \frac{G_F^2 \al_{em}^2 m_{B_s}^5 f_{B_s}^2 \tau_{B_s}}{64\pi^3}\sqrt{1-\frac{4m_\mu^2}{m_{B_s}^2}}\times \nl \bigg\{\bigg(1-\frac{4m_\mu^2}{m_{B_s}^2}\bigg)\bigg|\zeta\frac{C_S-C_S^{'}}{m_b+m_s}\bigg|^2+\bigg|\zeta\frac{C_P-C_P^{'}}{m_b+m_s}+\frac{2m_\mu}{m_{B_s}^2}[|V_{tb} V_{ts}^*|C_{10}+\zeta (C_A-C_A^{'})]\bigg|^2\bigg\}
\eea
\end{center}
Where $\zeta \equiv (\frac{g_{NP}^2}{\Lambda^2})(\frac{\sqrt{2}}{4G_F})(\frac{4\pi}{\alpha_{em}})$\\
\\
The Feynman Diagram of the decay can be seen below: 
\begin{figure}[h]
\centering
\includegraphics[scale=.85]{B_sdecay}
\caption{Graph of $B_s \rightarrow \mu^+ + \mu^-$}
\end{figure}
Since we seek to find the constraints on the parameters $C_S, C_S^{'},C_P, C_P^{'},C_A,$ and $C_A^{'}$ we set the equation equal to the branching ratio given in \cite{flavio} and using the values for $C_{10},|V_{tb} V_{ts}^*|, g_{NP},$ and $\Lambda$ given in \cite{Dighe:2012df} thus we are able to find the constraints on the parameters. In order to solve for one parameter, we allow the other two to be equal to 0, this will simply our calculations.We will solve each coefficient by the upper and lower limits of the branching ratio then take the average by setting the expression with the unknown coefficient equal to the branching ratio, the upper limit of the branching ratio, and its lower limit. \\
\\
The values for the Branching ratio of $B_s\rightarrow \mu^+\mu^- $ decay is equal to $2.4_{-0.7}^{+0.9} \times 10^{-9}$ \cite{Agashe:2014kda}
\bea
C_S&=& \pm 1.6215i \times 10^{-4} ~~{\rm For~ the~ lower~ BR~ value}\nl
&=& \pm 4.63981\times 10^{-5}~~ {\rm For~ the~ upper~ BR~ value}\nn
\eea
\bea
C_P&=& 6.3832 \times 10^{-5}~ {\rm or}~ 4.1125\times 10^{-4}~~{\rm For~ the~lower~ BR~ value}\nl
&=&-4.482\times 10^{-6}~ {\rm or}~ 4.7957\times 10^{-4}~~{\rm For~ the~ upper~BR~ value}\nn
\eea
\bea
C_A&=& 2.0348\times 10^{-3}~ {\rm or}~ 1.31095\times 10^{-2}~~{\rm For~ the~ lower~BR~ value}\nl
&=& 1.48237\times 10^{-4} ~ {\rm or}~ 1.5287\times 10^{-2}~~{\rm For~ the~upper~ BR~ value}\nn
\eea
\textbf{Notes about the Wilson Coefficients:}\\
The $C_S$ coefficient is equal to $\pm (0.811i + 0.232)\times 10^{-4}$ (We are able to calculate the mean by a simply average)\\
\\
The $C_P$ has two values the first value is equal to $2.967\times 10^{-5}$ and the second value is equal to $4.454 \times 10^{-4}$ \\
\\
The last coefficient $C_{A}$ also has two values, the first one is equal to $1.089 \times 10^{-3}$ and the second value is equal to $1.420 \times 10^{-2}$\\
\\
After finding the Wilson Coefficients, I was able to use the Python package Flavio \cite{flavio} to compute the NP values of $B(B_s\rightarrow \mu^+\mu^-)$\\
\\
We again only apply one Wilson Coefficient at a time to find the NP branching ratio:\\ 
For the $C_S$:
\begin{center}
 $C_S = \pm (0.811i + 0.232)\times 10^{-4}$\\
 Flavio Prediction: BR = $3.610 \times 10^{-9}$\\
\end{center}
For $C_P$:
\begin{center}
$C_{P} = 2.967\times 10^{-5}$ \\
Flavio Prediction: $Br = 3.603 \times 10^{-9}$\\
$C_{P} = 4.454\times 10^{-4}$ \\
Flavio Prediction: $Br = 3.508 \times 10^{-9}$\\
\end{center}
For $C_A$:
\begin{center}
$C_{A} = 1.089\times 10^{-3}$ \\
Flavio Prediction: $Br = 3.608 \times 10^{-9}$\\
$C_{A} = 1.420\times 10^{-2}$ \\
Flavio Prediction: $Br = 3.584 \times 10^{-9}$\\
\end{center}
\newpage
\section{Plots of $Br(WC)$}
\begin{figure}[h]
\centering
\includegraphics[scale=.65]{B(WC)forB_smumu}
\caption{Graph of $Br(B_s\rightarrow \mu^+\mu^-)$ as functions of the $C_P$ and $C_S$ Wilson Coefficients}
\end{figure}
\begin{figure}[h]
\centering
\includegraphics[scale=.65]{B(WC)forB_smumu2}
\caption{Graph of $Br(B_s\rightarrow \mu^+\mu^-)$ as functions of the $C_A$ Wilson Coefficient}
\end{figure}
The above graphs gives an indication of how drastically the variation of a Wilson Coefficient could change the value of the Branching ratio.\\
For each of the plots I have conducted a fit to the lines in order to find the coefficient values to the lines. \\
$Br(C_S)=3.22046\times10^{-6} x^2$\\
$Br(C_P)=3.57491\times10^{-6} x^2$\\
$Br(C_A)=2.79479\times10^{-10} x^2$\\
\begin{figure}[h]
\centering
\includegraphics[scale=.65]{BrofB-Kmumu}
\caption{Graph of $Br(B\rightarrow K\mu^+\mu^-)$ as functions of the Wilson Coefficients}
\end{figure}\\
\\
\\
Here, in this decay I was also able to compute fits to the plots.\\
\\
$Br(C_S)=5.73539\times10^{-9} x^2$\\
$Br(C_P)=5.81325\times10^{-9} x^2$\\
$Br(C_A)=1.7963\times10^{-9} x^2$\\
$Br(C_9)=1.75406\times10^{-9} x^2$\\
\begin{figure}[h]
\centering
\includegraphics[scale=.65]{BrofB-Kstarmumu}
\caption{Graph of $Br(B\rightarrow K^*\mu^+\mu^-)$ as functions of the Wilson Coefficients}
\end{figure}\\
The fits for the plots are as follows:\\
$Br(C_S)=5.97644\times10^{-9}x^2$\\
$Br(C_P)=6.056\times10^{-9}x^2$\\
$Br(C_A)=2.45199\times10^{-9} x^2$\\
$Br(C_9)=2.40283\times10^{-9} x^2$\\
\section{Conclusion}
As one can see, the work done with the Wilson Coefficients gives new suggestions to the possibility of new physics. In the future, I would like to continue to explore these possibilities. Some work that was not included in this paper is working to use a chi squared minimization technique to help limit and constrain the coefficients. The chi squared technique would use the observables that have been calculated experimentally. Although I was not able to accomplish this during the summer, I hope to continue working on this is the future and produce significant results.
\section{Acknowledgements}
I would like to thank the Department of Physics and Astronomy at Wayne State University for this opportunity to conduct this research.This project was supported through the Wayne State University REU program, under NSF grant phy-1460853 I would also like to thank Dr.Bhubanjyoti Bhattacharya for his guidance and suggestions and doctoral student Cody Grant who's advice and discussions were invaluable. Finally I would like to thank my parents Andrew and Liz Walz for giving me the love and support to pursue my interests and begin my career as a physicist.  
\newpage
\begin{thebibliography}{9}
\bibitem{Griffiths} 
Griffiths, D. (2014). Introduction to elementary particles. Weinheim: Wiley-VCH Verlag
%\cite{Agashe:2014kda}
\bibitem{Agashe:2014kda} 
  K.~A.~Olive {\it et al.} [Particle Data Group],
  %``Review of Particle Physics,''
  Chin.\ Phys.\ C {\bf 38}, 090001 (2014).
  doi:10.1088/1674-1137/38/9/090001
  %%CITATION = doi:10.1088/1674-1137/38/9/090001;%%
  %6385 citations counted in INSPIRE as of 03 Jul 2017
  %\cite{Dighe:2012df}
\bibitem{Dighe:2012df} 
  A.~Dighe and D.~Ghosh,
  %``How large can the branching ratio of $B_s \to \tau^+ \tau^-$ be ?,''
  Phys.\ Rev.\ D {\bf 86}, 054023 (2012)
  doi:10.1103/PhysRevD.86.054023
  [arXiv:1207.1324 [hep-ph]].
  %%CITATION = doi:10.1103/PhysRevD.86.054023;%%
  %20 citations counted in INSPIRE as of 06 Jul 2017
  %\cite{Dighe:2012df}
 \bibitem{flavio}
    David Straub, Peter Stangl, ChristophNiehoff, Ece Gurler, wzeren, Jacky Kumar, … Frederik Beaujean. (2017, June 29). flav-io/flavio v0.22.1. Zenodo. http://doi.org/10.5281/zenodo.821015
%\cite{Bhattacharya:2016mcc}
\bibitem{Bhattacharya:2016mcc} 
  B.~Bhattacharya, A.~Datta, J.~P.~Guévin, D.~London and R.~Watanabe,
  %``Simultaneous Explanation of the $R_K$ and $R_{D^{(*)}}$ Puzzles: a Model Analysis,''
  JHEP {\bf 1701}, 015 (2017)
  doi:10.1007/JHEP01(2017)015
  [arXiv:1609.09078 [hep-ph]].
  %%CITATION = doi:10.1007/JHEP01(2017)015;%%
%\cite{Aaij:2014ora}
\bibitem{Aaij:2014ora} 
  R.~Aaij {\it et al.} [LHCb Collaboration],
  %``Test of lepton universality using $B^{+}\rightarrow K^{+}\ell^{+}\ell^{-}$ decays,''
  Phys.\ Rev.\ Lett.\  {\bf 113}, 151601 (2014)
  doi:10.1103/PhysRevLett.113.151601
  [arXiv:1406.6482 [hep-ex]].
  %%CITATION = doi:10.1103/PhysRevLett.113.151601;%%
  %416 citations counted in INSPIRE as of 02 Aug 2017
  %\cite{Bordone:2016gaq}
\bibitem{Bordone:2016gaq} 
  M.~Bordone, G.~Isidori and A.~Pattori,
  %``On the Standard Model predictions for $R_K$ and $R_{K^*}$,''
  Eur.\ Phys.\ J.\ C {\bf 76}, no. 8, 440 (2016)
  doi:10.1140/epjc/s10052-016-4274-7
  [arXiv:1605.07633 [hep-ph]].
  %%CITATION = doi:10.1140/epjc/s10052-016-4274-7;%%
  %70 citations counted in INSPIRE as of 02 Aug 2017
  %\cite{Lees:2013uzd}
\bibitem{Lees:2013uzd} 
  J.~P.~Lees {\it et al.} [BaBar Collaboration],
  %``Measurement of an Excess of $\bar{B} \to D^{(*)}\tau^- \bar{\nu}_\tau$ Decays and Implications for Charged Higgs Bosons,''
  Phys.\ Rev.\ D {\bf 88}, no. 7, 072012 (2013)
  doi:10.1103/PhysRevD.88.072012
  [arXiv:1303.0571 [hep-ex]].
  %%CITATION = doi:10.1103/PhysRevD.88.072012;%%
  %250 citations counted in INSPIRE as of 02 Aug 2017
  %\cite{Huschle:2015rga}
\bibitem{Huschle:2015rga} 
  M.~Huschle {\it et al.} [Belle Collaboration],
  %``Measurement of the branching ratio of $\bar{B} \to D^{(\ast)} \tau^- \bar{\nu}_\tau$ relative to $\bar{B} \to D^{(\ast)} \ell^- \bar{\nu}_\ell$ decays with hadronic tagging at Belle,''
  Phys.\ Rev.\ D {\bf 92}, no. 7, 072014 (2015)
  doi:10.1103/PhysRevD.92.072014
  [arXiv:1507.03233 [hep-ex]].
  %%CITATION = doi:10.1103/PhysRevD.92.072014;%%
  %214 citations counted in INSPIRE as of 02 Aug 2017
  %\cite{Aaij:2015yra}
\bibitem{Aaij:2015yra} 
  R.~Aaij {\it et al.} [LHCb Collaboration],
  %``Measurement of the ratio of branching fractions $\mathcal{B}(\bar{B}^0 \to D^{*+}\tau^{-}\bar{\nu}_{\tau})/\mathcal{B}(\bar{B}^0 \to D^{*+}\mu^{-}\bar{\nu}_{\mu})$,''
  Phys.\ Rev.\ Lett.\  {\bf 115}, no. 11, 111803 (2015)
  Erratum: [Phys.\ Rev.\ Lett.\  {\bf 115}, no. 15, 159901 (2015)]
  doi:10.1103/PhysRevLett.115.159901, 10.1103/PhysRevLett.115.111803
  [arXiv:1506.08614 [hep-ex]].
  %%CITATION = doi:10.1103/PhysRevLett.115.159901, 10.1103/PhysRevLett.115.111803;%%
  %255 citations counted in INSPIRE as of 02 Aug 2017
  \bibitem{quarks}
  MissMJ on Wikipedia Commons. Standard model of elementary particles, 2006. [Online;   accessed August 2, 2017].
\end{thebibliography}
\newpage
\section{Appendix}
\begin{center}
\underline{\textbf{\Large Practice with Spinors}}
\end{center}
Notation:
\begin{enumerate} \itemsep=-15pt
\item $S= \ou u$\\
\item $P=\ou \ga^5 u$ \\
\item $V^\mu = \ou \ga^\mu u$ \\
\item $A^\mu = \ou \ga^\mu \ga^5 u $ \\
\item $T^{\mu\nu} = \ou \si^{\mu\nu} u$
\end{enumerate}
\bigskip
It was necessary to practice with spinor notation and the different mathematical techniques before starting this project. This appendix summarizes what I was able to do in that regard. 
\begin{enumerate}

\item$(\ou_1\ga^\mu u_2)^*$ ~=~?

Note: $(\ga^0)^\dag~=~\ga^0$ and $(\ga^\mu)^\dag~=~ \ga^0\ga^\mu\ga^0$

$(\ou_1\ga^\mu u_2)$ is a $1\times1$ matrix. Therefore, its complex conjugate is the same as its Hermitian conjugate, i.e. if we call
$V^\mu = (\ou_1\ga^\mu u_2)$, then $(V^\mu)^* = (V^\mu)^\dag$. We can then express this quantity as follows:
% This is the first equation/set of equations
\bea
V^\mu &=& \ou_1\ga^\mu u_2 ~,~~ \\
\Rightarrow (V^\mu)^* &=& (V^\mu)^\dag ~,~~ \nl
&=& (\ou_1\ga^\mu u_2)^\dag ~,~~ \nl
&=& ((u_1)^\dag\ga^0\ga^\mu u_2)^\dag ~~~{\rm using}~(A\ldots Z)^\dag = Z^\dag\ldots A^\dag ~,~~ \nl
&=& (u_2^\dag)(\ga^\mu)^\dag(\ga^0)^\dag(u_1)\nl
&=& (u_2^\dag)\ga^0\ga^\mu\ga^0\ga^0(u_1)\nl
&=& (u_2^\dag)\ga^0\ga^\mu(u_1)\nl
&=& \ou_2\ga^\mu(u_1)
\eea
Therefore $(\ou_1\ga^\mu u_2)^*~=~\ou_2\ga^\mu u_1 $. To solve for $|V^\mu|^2$ we simply use $|V^\mu|^2 ~=~ \Tr[\ou_1 \ga^\mu u_2 \ou_2 \ga^\nu u_1]$.

Note: ~$\Tr[\ga^\mu\ga^\nu]~=~4 g^{\mu\nu}$, $\Tr[\ga^\mu \ga^\nu \ga^\la \ga^\si]= 4(g^{\mu\nu}g^{\la\si} - g^{\mu\la} g^{\nu\si} + g^{\mu\si}g^{\nu\la})$, The trace over the product of an odd number of gamma matrices is zero.

\bea
|V^{\mu}|^2 &=& \Tr[\ou_1 \ga^\mu u_2 \ou_2 \ga^\nu u_1] \\
&=& \Tr[\ou_1 \ga^\mu (\slashed{p}_2+m) \ga^\nu u_1]\nl
&=& \Tr[u_1\ou_1 \ga^\mu (\slashed{p}_2+m) \ga^\nu]\nl
&=& \Tr[(\slashed{p}_1+m) \ga^\mu (\slashed{p}_2+m) \ga^\nu]\nl
&=& \Tr[\slashed{p}_1\ga^\mu \slashed{p}_2 \ga^\nu] + m[\Tr(\ga^\mu \slashed{p}_1\ga^\nu) + \Tr(\ga^\mu \ga^\nu \slashed{p}_2)] + m^2\Tr[\ga^\mu \ga^\nu]\nl
&=& \Tr[\slashed{p}_1\ga^\mu \slashed{p}_2 \ga^\nu] + m^2\Tr[\ga^\mu \ga^\nu]\nl
&=& \Tr[(p_1)_\la\ga^\la \ga^\mu (p_2)_\si\ga^\si \ga^\nu  ] + 4m^2g^{\mu\nu}\nl
&=& (p_1)_\la (p_2)_\si \Tr[\ga^\la \ga^\mu\ga^\si \ga^\nu  ] + 4m^2g^{\mu\nu}\nl
&=& (p_1)_\la(p_2)_\si 4(g^{\mu\nu}g^{\la\si} - g^{\mu\la} g^{\nu\si} + g^{\mu\si}g^{\nu\la}) + 4m^2g^{\mu\nu}\nl
&=& 4[p_1^\mu p_2^\nu - g^{\mu\nu}(p_1 \cdot p_2) + p_2^\mu p_1^\nu] + 4m^2g^{\mu\nu} 
\eea

\item $(\ou_1\ga^\mu \ga^5 u_2)^*$ is also a $1\times1$ Matrix so the same reasoning applies as above in 1. Note: $(\ga^5)^\dag = \ga^5$
We define: $A^\mu$ as $\ou_1\ga^\mu\ga^5 u_2$ thus:
% This is the second equation/set of equations
\bea
(A^\mu)^* &=& (A^\mu)^\dag \\
&=&(\ou_1\ga^\mu\ga^5 u_2)^\dag \nl
&=& ((u_1)^\dag\ga^0\ga^\mu\ga^5 u_2)^\dag \nl
&=& (u_2^\dag)(\ga^5)^\dag(\ga^\mu)^\dag(\ga^0)^\dag(u_1) \nl
&=& (u_2^\dag)\ga^5\ga^0\ga^\mu\ga^0\ga^0 u_1 \nl
&=& (u_2^\dag)\ga^5\ga^0\ga^\mu(1) u_1 \nl
&=& -(u_2^\dag)\ga^0\ga^5\ga^\mu u_1 \nl
&=& -\ou_2\ga^5\ga^\mu u_1 \nl
&=& \ou_2\ga^\mu\ga^5 u_1
\eea
%
Therefore $(\ou_1\ga^\mu \ga^5 u_2)^*~=~\ou_2 \ga^\mu \ga^5 u_1$

We also are able to calculate $|A^\mu|^2$
\bea
|A^{\mu}|^2 &=& \Tr[(\ou_1\ga^\mu \ga^5 u_2)(\ou_2 \ga^\nu \ga^5 u_1)] \\
&=& \Tr[\ou_1\ga^\mu \ga^5 (\slashed{p}_2+m) \ga^\nu \ga^5 u_1]  \nl
&=&\Tr[u_1\ou_1\ga^\mu \ga^5 (\slashed{p}_2+m) \ga^\nu \ga^5 ]\nl
&=&\Tr[(\slashed{p}_1+m)\ga^\mu \ga^5 (\slashed{p}_2+m) \ga^\nu \ga^5 ]\nl
&=& \Tr[\slashed{p}_1\ga^\mu\ga^5 \slashed{p}_2\ga^\nu \ga^5 + m( \slashed{p}_1\ga^\mu\ga^5\ga^\nu \ga^5 +\ga^\mu\ga^5 \slashed{p}_2\ga^\nu \ga^5) + m^2(\ga^\mu\ga^5\ga^\nu \ga^5)] \nl
&=&\Tr[\slashed{p}_1\ga^\mu\ga^5 \slashed{p}_2\ga^\nu \ga^5 + m( \slashed{p}_1\ga^\mu(-\ga^5 \ga^5)\ga^\nu +\slashed{p}_2\ga^\mu(-\ga^5\ga^5) \ga^\nu ) + m^2(\ga^\mu\ga^5\ga^\nu \ga^5)] \nl
&=&\Tr[\slashed{p}_1\ga^\mu\ga^5 \slashed{p}_2\ga^\nu \ga^5 - m( \slashed{p}_1\ga^\mu\ga^\nu +\slashed{p}_2\ga^\mu\ga^\nu ) + m^2(\ga^\mu\ga^5\ga^\nu \ga^5)] \nl
&=&\Tr[\slashed{p}_1\ga^\mu\ga^5 \slashed{p}_2\ga^\nu \ga^5] -mTr[ \slashed{p}_1\ga^\mu\ga^\nu]-mTr[ \slashed{p}_2\ga^\mu\ga^\nu]+ m^2(\ga^\mu\ga^5\ga^\nu \ga^5)] \nl
&=&\Tr[\slashed{p}_1\ga^\mu\ga^5 \slashed{p}_2\ga^\nu \ga^5 + m^2(\ga^\mu\ga^5\ga^\nu \ga^5)] \nl
&=& \Tr[(p_1)_\la\ga^\la\ga^\mu\ga^5 (p_2)_\si\ga^\si\ga^\nu \ga^5] + m^2\Tr[\ga^\mu\ga^5\ga^\nu \ga^5]\nl
&=& (p_1)_\la(p_2)_\si \Tr[\ga^\la\ga^\mu\ga^5 \ga^\si\ga^\nu \ga^5] - m^2\Tr[\ga^\mu\ga^5\ga^5 \ga^\nu]\nl
&=& (p_1)_\la(p_2)_\si \Tr[\ga^\la\ga^\mu\ga^5\ga^5 \ga^\si\ga^\nu] - m^2\Tr[\ga^\mu\ga^\nu]\nl
&=& (p_1)_\la(p_2)_\si \Tr[\ga^\la\ga^\mu\ga^\si\ga^\nu] - m^2(g^{\mu\nu})\nl
&=& (p_1)_\la(p_2)_\si 4(g^{\mu\nu}g^{\la\si} - g^{\mu\la} g^{\nu\si} + g^{\mu\si}g^{\nu\la}) - 4m^2g^{\mu\nu}\nl
&=& 4[p_1^\mu p_2^\nu - g^{\mu\nu}(p_1 \cdot p_2) + p_2^\mu p_1^\nu] - 4m^2g^{\mu\nu} 
\eea

\item $ (\ou_1 u_2)^* ~=~ ?$
We let $S ~=~ \ou_1 u_2$
\bea
(S)^* &=& (S)^\dag \\
&=& (\ou_1 u_2)^\dag ~,~~ \nl
&=& ((u_1)^\dag\ga^0 u_2)^\dag \nl
&=& (u_2)^\dag(\ga^0)^\dag(u_1)\nl
&=& (u_2)^\dag\ga^0(u_1)\nl
&=& \ou_2(u_1)
\eea
Therefore$(\ou_1 u_2)^* ~=~ \ou_2 u_1$.
In order to find $|S|^2$ we simply do the following:
\bea
|S|^2&=& \Tr[\ou_1 u_2 \ou_2 u_1]  \\
&=& \Tr[(\slashed{p}_1+m)(\slashed{p}_2+m)] \nl
&=& \Tr[\slashed{p}_1\slashed{p}_2+m(\slashed{p}_1+\slashed{p}_2)+m^2]\nl
&=& \Tr[\slashed{p}_1\slashed{p}_2]+\Tr[m(\slashed{p}_1+\slashed{p}_2)]+\Tr[m^2] \nl
&=& \Tr[\slashed{p}_1\slashed{p}_2]+m(\Tr[\slashed{p}_1]+\Tr[\slashed{p}_2])+m^2\Tr[1] \nl
&=& \Tr[\slashed{p}_1\slashed{p}_2]+4m^2 \nl
&=& 4(p_1 \cdot p_2)+4m^2
\eea

\item  By the same reasoning as shown above it can be shown that $(\ou_1 \ga^5 u_2)^* ~=~ \ou_2 \ga^5 u_1$ \\
If we let $P ~=~ \ou_1\ga^5 u_2$ then:
\bea
(P)^* &=& (P)^\dag \\
&=& (\ou_1\ga^5 u_2)^\dag ~,~~ \nl
&=& ((u_1)^\dag\ga^0\ga^5 u_2)^\dag \nl
&=& (u_2)^\dag(\ga^5)^\dag (\ga^0)^\dag(u_1)\nl
&=& (u_2)^\dag(\ga^5) \ga^0(u_1)\nl
&=& -(u_2)^\dag \ga^0\ga^5(u_1)\nl
&=& -\ou_2\ga^5(u_1)
\eea
Therefore  $(\ou_1 \ga^5 u_2)^* ~=~ -\ou_2 \ga^5 u_1$

In order to square $P$ we do the following:
\bea
|P|^2&=& \Tr[\ou_1 \ga^5 u_2(-\ou_2 \ga^5 u_1)] \\
&=& \Tr[u_1 \ou_1 \ga^5(-\slashed{p}_2 - m) \ga^5] \nl
&=& \Tr[(\slashed{p}_1+m) \ga^5 (-\slashed{p}_2-m) \ga^5] \nl
&=& \Tr[((p_1)_\mu \ga^\mu+m) \ga^5 ((-p_2)_\nu\ga^\nu-m)  \ga^5] \nl
&=& \Tr[((p_1)_\mu \ga^\mu\ga^5+m\ga^5) ((-p_2)_\nu\ga^\nu\ga^5-m\ga^5)] \nl
&=& \Tr[(p_1)_\mu \ga^\mu\ga^5(-p_2)_\nu\ga^\nu\ga^5+ m\ga^5(-p_2)_\nu\ga^\nu\ga^5 -m\ga^5(p_1)_\mu \ga^\mu\ga^5 - \ga^5\ga^5m^2)] \nl
&=& \Tr[(p_1)_\mu \ga^\mu\ga^5(-p_2)_\nu\ga^\nu\ga^5] + Tr[ m\ga^5(-\ga^5)(-p_2)_\nu\ga^\nu] - Tr[m\ga^5(-\ga^5)(p_1)_\mu \ga^\mu] - Tr[m^2] \nl
&=& \Tr[(p_1)_\mu \ga^\mu\ga^5(-p_2)_\nu\ga^\nu\ga^5] -Tr[ m(-p_2)_\nu\ga^\nu] + Tr[m(p_1)_\mu \ga^\mu] - 4m^2 \nl
&=& \Tr[(p_1)_\mu \ga^\mu\ga^5(-p_2)_\nu\ga^\nu\ga^5] - 4m^2 \nl
&=& (p_1)_\mu(-p_2)_\nu \Tr[\ga^\mu\ga^5\ga^\nu\ga^5]- 4m^2 \nl
&=& (p_1)_\mu(-p_2)_\nu \Tr[\ga^\mu\ga^5(-\ga^5\ga^\nu)]- 4m^2\nl
&=& (p_1)_\mu(-p_2)_\nu (-\Tr[\ga^\mu\ga^\nu])- 4m^2\nl
&=& (p_1)_\mu(-p_2)_\nu (-4g^{\mu\nu})- 4m^2\nl
&=& 4(p_1)(p_2)- 4m^2
\eea
\item  While the above identities could be shown to be trivial, the identity: $(\ou_1 \si^{\mu\nu} u_2)^* ~=~\ou_2 \si^{\nu\mu} u_1$ is more difficult to solve. The identity:$(\si^{\mu\nu})^\dag =\si^{\mu\nu}$ is needed
\bea
(\si^{\mu\nu})^\dag &=& (\frac{i}{2}[\ga^\mu,\ga^\nu])^\dag \\
&=& -\frac{i}{2}([\ga^\mu,\ga^\nu])^\dag\nl
&=& -\frac{i}{2}(\ga^\mu\ga^\nu-\ga^\nu\ga^\mu)^\dag \nl
&=& -\frac{i}{2}((\ga^\nu)^\dag(\ga^\mu)^\dag-(\ga^\mu)^\dag(\ga^\nu)^\dag) \nl
&=& -\frac{i}{2}(\ga^0\ga^\nu\ga^0\ga^0\ga^\mu\ga^0-\ga^0\ga^\mu\ga^0\ga^0\ga^\nu\ga^0)\nl
&=& -\frac{i}{2}(\ga^0\ga^\nu\ga^\mu\ga^0-\ga^0\ga^\mu \ga^\nu\ga^0)\nl
&=& -\frac{i}{2}((-1)^2\ga^\nu\ga^\mu-(-1)^2\ga^\mu \ga^\nu) \nl
&=& -\frac{i}{2}(\ga^\nu\ga^\mu-\ga^\mu \ga^\nu)\nl
&=& -\si^{\nu\mu}
\eea
After showing $(\si^{\mu\nu})^\dag =-\si^{\nu\mu}$ is true it is trivial to show $(\ou_1 \si^{\mu\nu} u_2)^* ~=~ -\ou_2 \si^{\nu\mu} u_1$.\\
We let $T^{\mu\nu} ~=~ \ou_1 \si^{\mu\nu} u_2$
\bea
(T^{\mu\nu})^* &=& (T^{\mu\nu})^\dag ~,~~ \\
&=& (\ou_1 \si^{\mu\nu} u_2)^\dag \nl
&=& ((u_1)^\dag \ga^0 \si^{\mu\nu} u_2)^\dag \nl
&=& (u_2^\dag)(\si^{\mu\nu})^\dag (\ga^0)^\dag(u_1)\nl
&=& (u_2^\dag)(-\si^{\nu\mu})\ga^0(u_1)\nl
&=& (u_2^\dag)(-\ga^0)(-\si^{\nu\mu})(u_1)\nl
&=& \ou_2 \si^{\mu\nu}(u_1)
\eea

In order to find the value of $|T^{\mu\nu}|^2$ one needs to find the value of $Tr[\si^{\si\la}\si^{\mu\nu}]$\\
\bea
Tr[\si^{\si\la}\si^{\mu\nu}]&=& Tr[\frac{i}{2}(\ga^\si\ga^\la - \ga^\la\ga^\si)\frac{i}{2}(\ga^\mu\ga^\nu - \ga^\nu\ga^\mu)]\\
&=& Tr[\frac{i}{2}(\ga^\si\ga^\la - \ga^\la\ga^\si)\frac{i}{2}(\ga^\mu\ga^\nu - \ga^\nu\ga^\mu)]\nl
&=& -\frac{1}{4}Tr[(\ga^\si\ga^\la - \ga^\la\ga^\si)(\ga^\mu\ga^\nu - \ga^\nu\ga^\mu)]\nl
&=& -\frac{1}{4}Tr[\ga^\si\ga^\la\ga^\mu\ga^\nu]+\frac{1}{4}Tr[\ga^\si\ga^\la\ga^\nu\ga^\mu]+\frac{1}{4}Tr[\ga^\la\ga^\si\ga^\mu\ga^\nu]-\frac{1}{4}Tr [\ga^\la\ga^\si\ga^\nu\ga^\mu]\nn
\eea
Here we must label each of the traces individually: 
\bea
A &=&-\frac{1}{4}Tr[\ga^\si\ga^\la\ga^\mu\ga^\nu]\nl
&=&-(g^{\si\la}g^{\mu\nu}-g^{\si\mu}g^{\la\nu}+g^{\si\nu}g^{\la\mu})
\eea
\bea
B&=&+\frac{1}{4}Tr[\ga^\si\ga^\la\ga^\nu\ga^\mu]\nl
&=&+(g^{\si\la}g^{\nu\mu}-g^{\si\nu}g^{\la\mu}+g^{\si\mu}g^{\la\nu})
\eea
\bea
C&=& \frac{1}{4}Tr[\ga^\la\ga^\si\ga^\mu\ga^\nu]\nl
&=& (g^{\la\si}g^{\mu\nu}-g^{\la\mu}g^{\si\nu}+g^{\la\nu}g^{\si\mu})
\eea
\bea 
D&=& -\frac{1}{4}Tr [\ga^\la\ga^\si\ga^\nu\ga^\mu]\nl
&=& -(g^{\la\si}g^{\nu\mu}-g^{\la\nu}g^{\si\mu}+g^{\la\mu}g^{\si\nu})
\eea
\bea
Tr[\si^{\si\la}\si^{\mu\nu}]&=& A + B+ +C+D\nl
&=& 2g^{\si\mu}g^{\la\nu} - 2 g^{\si\nu}g^{\la\mu} - 2 g^{\la\mu}g^{\si\nu} +  2g^{\la\nu}g^{\si\mu})
\eea

In order to find the value of $|T^{\mu\nu}|^2$ one needs to do the following:
\bea
|T^{\mu\nu}|^2 &=& \Tr[\ou_1 \si^{\mu\nu} u_2\ou_2 \si^{\si\la}u_1]\\
&=& \Tr[\ou_1 \si^{\mu\nu}(\slashed{p}_2+m) \si^{\si\la} u_1]\nl
&=& \Tr[(\slashed{p}_1+m)\si^{\mu\nu}(\slashed{p}_2+m) \si^{\si\la}]\nl
&=& \Tr[((p_1)_\ka \ga^\ka +m)\si^{\mu\nu}((p_2)_\ga \ga^\ga+m) \si^{\si\la}]\nl
&=& \Tr[((p_1)_\ka  \ga^\ka\si^{\mu\nu} +m\si^{\mu\nu})((p_2)_\ga \ga^\ga\si^{\si\la}+m\si^{\si\la})]\nl
&=& \Tr[((p_1)_\ka \ga^\ka\si^{\mu\nu}(p_2)_\ga \ga^\ga\si^{\si\la} + (p_1)_\ka \ga^\ka\si^{\mu\nu}m\si^{\si\la} +(p_2)_\ga \ga^\ga\si^{\si\la}m\si^{\mu\nu}+m\si^{\si\la}m\si^{\mu\nu})]\nl
&=& \Tr[((p_1)_\ka  \ga^\ka\si^{\mu\nu}(p_2)_\ga \ga^\ga\si^{\si\la}] +Tr[(p_1)_\ka \ga^\ka\si^{\mu\nu}m\si^{\si\la} + (p_2)_\ga \ga^\ga\si^{\si\la}m\si^{\mu\nu}]+Tr[m\si^{\si\la}m\si^{\mu\nu}]\nl
&=& \Tr[((p_1)_\ka \ga^\ka\si^{\mu\nu}(p_2)_\ga \ga^\ga\si^{\si\la}] +Tr[m\si^{\si\la}m\si^{\mu\nu}]\nl
&=& \Tr[((p_1)_\ka  \ga^\ka\si^{\mu\nu}(p_2)_\ga \ga^\ga \si^{\si\la}] +m^2Tr[\si^{\si\la}\si^{\mu\nu}]{\rm~Let~ B~=m^2Tr[\si^{\si\la}\si^{\mu\nu}]}\nl
&=& \Tr[((p_1)_\ka  \ga^\ka\si^{\mu\nu}(p_2)_\ga \ga^\ga \si^{\si\la}] +B\nl
&=& (p_1)_\ka (p_2)_\ga \Tr[\ga^\ka \si^{\mu\nu}\ga^\ga \si^{\si\la}]+B\nl
&=& (p_1)_\ka (p_2)_\ga \Tr[\ga^\ka (\frac{i}{2}((\ga^\mu \ga^\nu - \ga^\nu\ga^\mu)\ga^\ga(\frac{i}{2}(\ga^\si \ga^\la - \ga^\la \ga^\si))]+B\nl
&=& -\frac{1}{4} (p_1)_\ka (p_2)_\ga\Tr[\ga^\ka (\ga^\mu \ga^\nu - \ga^\nu \ga^\mu) \ga^\ga (\ga^\si \ga^\la - \ga^\la \ga^\si)]+B{\rm~(Let~ A~=-\frac{1}{4} (p_1)_\ka (p_2)_\ga)}\nl
&=& (A)\Tr[(\ga^\ka \ga^\mu \ga^\nu -\ga^\ka \ga^\nu \ga^\mu) (\ga^\ga\ga^\si \ga^\la - \ga^\ga\ga^\la \ga^\si)]+B \nl
&=& (A)\Tr[(\ga^\ka \ga^\mu \ga^\nu\ga^\ga\ga^\si \ga^\la -\ga^\ka \ga^\mu \ga^\nu\ga^\ga\ga^\la \ga^\si -  \ga^\ka \ga^\nu \ga^\mu\ga^\ga\ga^\si \ga^\la + \ga^\ka \ga^\nu \ga^\mu\ga^\ga\ga^\la \ga^\si)]+B\nl
&=& (A)\Tr[(\ga^\ka \ga^\mu \ga^\nu\ga^\ga\ga^\si \ga^\la -\ga^\ka \ga^\mu \ga^\nu\ga^\ga(-\ga^\si \ga^\la) - \ga^\ka \ga^\nu \ga^\mu\ga^\ga\ga^\si \ga^\la + \ga^\ka \ga^\nu \ga^\mu\ga^\ga(-\ga^\si \ga^\la))]+B\nl
&=& (A)\Tr[(\ga^\ka \ga^\mu \ga^\nu\ga^\ga\ga^\si \ga^\la +\ga^\ka \ga^\mu \ga^\nu\ga^\ga\ga^\si \ga^\la - \ga^\ka \ga^\nu \ga^\mu\ga^\ga\ga^\si \ga^\la - \ga^\ka \ga^\nu \ga^\mu\ga^\ga\ga^\si \ga^\la))]+B\nl
&=& (A)\Tr[2\ga^\ka \ga^\mu \ga^\nu\ga^\ga\ga^\si\ga^\la- 2\ga^\ka \ga^\nu \ga^\mu\ga^\ga\ga^\si\ga^\la]+B\nl
&=& (A)\Tr[2\ga^\ka (-\ga^\nu \ga^\mu)\ga^\ga\ga^\si\ga^\la- 2\ga^\ka \ga^\nu \ga^\mu\ga^\ga\ga^\si\ga^\la]+B\nl
&=& (A)\Tr[-2\ga^\ka \ga^\nu \ga^\mu\ga^\ga\ga^\si\ga^\la- 2\ga^\ka \ga^\nu \ga^\mu\ga^\ga\ga^\si\ga^\la]+B\nl
&=& (A) \Tr[-4\ga^\ka \ga^\nu \ga^\mu \ga^\ga \ga^\si \ga^\la]+B\nl
&=& -\frac{1}{4} (p_1)_\ka (p_2)_\ga \Tr[-4\ga^\ka \ga^\mu \ga^\nu \ga^\ga \ga^\si \ga^\la]+B\nl
&=& (p_1)_\ka (p_2)_\ga\Tr[\ga^\ka \ga^\ga]+B\nl
&=& 4(p_1)_\ka (p_2)_\ga g^{\ka\ga}+B\nl
&=& 4p_1p_2 + m^2Tr[\si^{\si\la}\si^{\mu\nu}]\nl
&=& 4p_1p_2 + m^24(g^{\mu\nu}g^{\la\si}- g^{\mu\la}g^{\nu\si}+g^{\mu\si}g^{\nu\la})
\eea
\end{enumerate}
\end{document}
