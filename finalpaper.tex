\documentclass[12pt]{article}

\usepackage{hyperref}
\usepackage{framed}

\usepackage{graphicx}
\usepackage{amsmath}
%\usepackage{mathtools}
\usepackage{amssymb}
\usepackage{slashed}
\usepackage{bm}
\usepackage{cite}
\usepackage{setspace}
\usepackage{bigints}
\usepackage{color}
\usepackage{siunitx}

\setlength{\oddsidemargin}{0in}
\setlength{\textwidth}{6.5in}
\setlength{\topmargin}{0in}
\setlength{\textheight}{9in}
\voffset-1.5truecm

\def \beq{\begin{equation}}
\def \eeq{\end{equation}}
\def \bea{\begin{eqnarray}}
\def \eea{\end{eqnarray}}
\def \ba{\begin{array}}
\def \ea{\end{array}}
\def \beg{\begin{gather}}
\def \eeg{\end{gather}}
\def \bmat{\begin{matrix}}
\def \emat{\end{matrix}}
\def \bfra{\begin{framed}}
\def \efra{\end{framed}}

\def \({\left(}
\def \){\right)}
\def \[{\left[}
\def \]{\right]}
\def \<{\left\langle}
\def \>{\right\rangle}
\def \lp{\left|}
\def \rp{\right|}
\def \lb{\left\{}
\def \rb{\right\}}
\def \l.{\left.}
\def \r.{\right.}

\def \bma{\(\bmat}
\def \ema{\emat\)}

\def \Tr{{\rm Tr}}
\def \N{\rm N}
\def \hc{{\rm h.\ c.}}

\def \ph{\phantom}
\def \nn{\nonumber}
\def \nl{\nn \\}

\def \of{\frac{1}{4}}
\def \oth{\frac{1}{3}}
\def \hf{\frac{1}{2}}
\def \oet{\frac{1}{8}}
\def \s{\sqrt{2}}
\def \st{\sqrt{3}}
\def \sx{\sqrt{6}}

\def \cB{{\cal B}}
\def \cC{{\cal C}}
\def \cZ{{\cal Z}}
\def \cL{{\cal L}}
\def \cH{{\cal H}}
\def \cO{{\cal O}}
\def \cM{{\cal M}}
\def \cl{\ell}

\def \oq{\overline{q}}
\def \of{\overline{f}}
\def \os{\overline{s}}
\def \ob{\overline{b}}
\def \ou{\overline{u}}
\def \ov{\overline{v}}
\def \oBs{\overline{B^0_s}}
\def \otau{\overline{\tau}}
\def \omu{\overline{\mu}}
\def \onu{\overline{\nu}}
\def \osi{\overline{\si}}
\def \oell{\overline{\ell}}
\def \oL{\overline{L}}
\def \oQ{\overline{Q}}

\def \al{\alpha}
\def \be{\beta}
\def \ga{\gamma}
\def \Ga{\Gamma}
\def \de{\delta}
\def \ka{\kappa}
\def \De{\Delta}
\def \ep{\epsilon}
\def \tha{\theta}
\def \la{\lambda}
\def \La{\Lambda}
\def \si{\sigma}
\def \Si{\Sigma}
\def \Up{\Upsilon}
\def \om{\omega}
\def \1{1}%\mathbb{I}}

\def \pa{\partial}

\def \SM{{\rm SM}}
\def \elm{{\rm em}}
\def \NP{{\rm NP}}
\def \expt{{\rm expt}}
\def \eff{{\rm eff}}
\def \elm{{\rm em}}
\def \RFG{{\rm RFG}}
\def \CFG{{\rm CFG}}
\def \PT{{\rm PT}}
\def \hi{{\rm hi}}
\def \lo{{\rm lo}}

\def \Re{{\rm Re}}
\def \Im{{\rm Im}}

\def \keV{{\rm keV}}
\def \MeV{{\rm MeV}}
\def \GeV{{\rm GeV}}

\def \cH{{\cal H}}
\def \cI{{\cal I}}

\def \Rqp{R_{q'}}

\def\bwt{\begin{widetext}}
\def\ewt{\end{widetext}}

\def\vev#1{\langle {#1} \rangle}
\def\eq#1{eq.~(\ref{#1})}

\def \vp{{\bf p}}
\def \vq{{\bf q}}

\def \cre{\color{red}}
\def \cgr{\color{green}}

\usepackage{graphicx} %package to manage images
\graphicspath{ {images/} }

\usepackage[rightcaption]{sidecap}

\usepackage{wrapfig}

% Start of document
% -----------------
\pagestyle{plain}
\allowdisplaybreaks

\setlength{\abovecaptionskip}{5pt}
\setlength{\belowcaptionskip}{5pt}

\begin{document}
    \begin{center}
        \vspace*{1cm}
        
        \Huge
        \textbf{The $R_K$ Puzzle and Analysis}
        
        \vskip 20pt
        
        \large{Francis Walz}\footnote{frankwalz10@gmail.com; REU blah}\\ Department of Physics, Astronomy, and Geosciences, \\ Towson University, Towson, MD 21252, USA \\
          and \\
        Department of Physics and Astronomy, \\ Wayne State University, Detroit, MI 48201, USA
        
        \begin{abstract}
        Recent measurements of $R_K$ ($\mathcal{B}(B^+\rightarrow K^+\mu^+\mu^-)/ \mathcal{B}(B^+\rightarrow K^+e^+e^-) [1-6$  GeV$^2]$) shows substantial deviation from its standard model prediction. This deviation could be due to new physics at a high-energy scale, that cause deviations in Wilson Coefficients of low-energy operators. This article seeks to constrain and analyze the Wilson Coefficients that are present in the branching ratios of decays.  
        \end{abstract}
        \vfill
    \end{center}
    
\tableofcontents

\section{Introduction}

The Standard Model (SM) of particle physics works extremely well when calculating decay rates of subatomic particle and explaining different types of physical phenomena that occur at the subatomic level. Included in the SM are lists of rules that govern interactions and different conserved quantities that have aligned well with the observed experimental results. The SM spectrum includes 4 types of particles: gauge bosons, quarks, leptons, and the Higgs boson. The gauge bosons are particles with spin 1 that mediate the different forces. For instance, the W and Z bosons mediate the weak force. In Table \ref{table:1} we summarize the physical properties of this type of particles. 
%
\begin{table}[htbp!]
\centering
\begin{tabular}{ |p{2cm}||p{5cm}|p{1.5cm}|p{3cm}|  }
 \hline
 \multicolumn{4}{|c|}{Gauge Bosons} \\
 \hline
 Boson & Mass & Charge & Force\\
 \hline
 $\ga$ & $0$ MeV & $0$ & Electromagnetic\\ [1ex]
 $Z$ &  $91.1876\pm 0.00214$ GeV & $0$ & Weak\\ [1ex]
 $W^{\pm 1}$ &$80.385 \pm 0.01596$ GeV & $\pm 1$ & Weak\\[1ex]
 Gluon & $0$ MeV &$0$ & Strong \\[1ex]
 \hline 
\end{tabular}
\caption{Table of Gauge Bosons in the SM \cite{Agashe:2014kda}}
\label{table:1}
\end{table}

In addition to the gauge bosons, the SM contains other particles known as fermions. Fermions have half-integer spin and are comprised of quarks and leptons. There are 3 types of leptons (electron, muon and tau), each with its corresponding neutrino. Each fermion ($f$) has an antiparticle ($\bar f$), which has the same mass but opposite charge. These leptons can be seen in Table \ref{table:2}.
\begin{table}[htbp!]
\centering
\begin{tabular}{ |p{1cm}||p{4.5cm}|p{1.5cm}| }
 \hline
 \multicolumn{3}{|c|}{Leptons} \\
 \hline
 Type & Mass & Charge\\
 \hline
 $e$ & $0.5109989461(31)$ MeV & $-1$ \\ [1ex]
 $\mu$ &  $105.6583745(24)$ MeV & $-1$ \\ [1ex]
 $\tau$ &$1776.86\pm 0.12$ MeV & $-1$ \\[1ex]
 \hline 
\end{tabular}
\caption{Table of Leptons in the SM \cite{Agashe:2014kda}}
\label{table:2}
\end{table}

The other types of fermions are quarks. Quarks differ from leptons because quarks participate in the strong force, while leptons do not. Quarks also have either a $\frac{2}{3}$ or $-\frac{1}{3}$ charge and can be seen below in Table \ref{table:3}. Finally, there is the recently discovered Higgs boson, a spinless scalar particle responsible for generating gauge boson and fermion masses in the SM.
\begin{table}[htbp!]
\centering
\begin{tabular}{ |p{1cm}||p{4cm}|p{1.5cm}| }
 \hline
 \multicolumn{3}{|c|}{Quarks} \\
 \hline
 Type & Mass & Charge \\
 \hline
 $u$ & $2.2 ^{+0.6}_{\num{-0.4}}$ MeV & $\frac{2}{3}$ \\ [1ex]
 $d$ &  $4.7 ^{+0.5}_{\num{-0.4}}$ MeV & $-\frac{1}{3}$ \\ [1ex]
 $s$ &$96 ^{+8}_{\num{-4}}$ MeV & $-\frac{1}{3}$ \\[1ex]
 $c$ & $1.28 \pm 0.03$ GeV &$\frac{2}{3}$ \\[1ex]
 $b$ & $4.18 ^{+0.04}_{\num{-0.03}}$ GeV  &$-\frac{1}{3}$ \\[1ex]
 $t$ & $173.1 \pm 0.6$ GeV &$\frac{2}{3}$ \\[1ex]
 \hline 
\end{tabular}
\caption{Table of Quarks in the SM \cite{Agashe:2014kda}}
\label{table:3}
\end{table}

In spite of its remarkable success in explaining subatomic phenomena, recent observations made by BaBar, Belle, and LHCb collaborations provide hints that there is new physics beyond the SM. In this article, we will study decays of the $B^+$ ($\bar{b}u$) and $B_s$ ($\bar{b}s$) mesons. We will focus on a few interesting channels that have received quite a bit of attention in the literature.

The $R_K$ and $R_{D^{(*)}}$ are two ratios of these $B$ meson decays that show some puzzling experimental results. The $R_K$ ratio is defined as $\mathcal{B}(B^+\rightarrow K^+\mu^+\mu^-)/ \mathcal{B}(B^+\rightarrow K^+e^+e^-)$ The experimental value found by the LHCb Collaboration is \cite{Aaij:2014ora}
\begin{center}
$R^{expt}_{K} = 0.745 _{\num{-0.074}}^{+0.090}$ (stat) $\pm 0.036$ (syst)
\end{center}
This differs from the SM prediction of $R^{expt}_{K} = 1 \pm 0.01$ by $2.6\si$. \cite{Bordone:2016gaq} Another puzzle is the $R_{D^{(*)}}$ ratio, defined as $\mathcal{B}(\bar{B} \rightarrow D^{(*)}\tau^- \onu_{\tau}/ \mathcal{B}(\bar{B} \rightarrow D^{(*)}l^- \onu_{l})$ where $(l= \mu$ or $e)$  We will focus on analyzing the possible NP contributions to the Wilson Coefficients (WC) arising from the effective Hamiltonian of the $b \rightarrow s\mu^+\mu^-$ transition. The complete Hamiltonian is as follows:\cite{Bhattacharya:2016mcc}. 
\begin{center}
\bea
H = -\frac{\alpha G_f}{\sqrt{2}\pi} V_{tb}V^*_{ts}\sum_{a= 9,10}^{}(C_aO_a + C'_aO'_a)
\eea
\end{center}
This article will take the Hamiltonian and use the WC found in the branching ratio of the $B_s \rightarrow \mu^+ \mu^- $ decay to constrain the effects of NP parameters. The article begins by first exploring and computing different observables of the $\pi^+ \rightarrow l^+ \nu_l $ decay. (Where $l= \mu$ or $e$) This will provide us with a background in manipulating and understanding the theoretical expressions for branching ratios and decays. It will also serve as a guide to the format for the rest of the paper. In section 3 we will constrain the WC from the branching ratio of the $B_s \rightarrow \mu^+\mu^-$ decay. We will use the Python package Flavio\cite{flavio} to generate our data and construct our plots. Flavio's packages focus on flavor physics and allows the user to find the contributions of different WC for varying observable. Finally in section 4 plots of the Branching ratio based upon the WC will be shown and fits to the data will be provided. We conclude the article in section 5.
\section{The branching Ratio of a Charged Pion decay}
In order to calculate the Branching ratio of a charged pion, we will use the Feynman Rules of calculating amplitudes and the trace identities. We start by recognizing that the pion decay is a charged current interaction since pion is made of quarks, and the decay is mediated by a massive W boson. A Feynman diagram for the decay may be seen in Figure \ref{fig:piondecay}: 
\begin{figure}[h]
\centering
\includegraphics[scale=.60]{piondecay}
\caption{Pion Decay (where $\ell$ is $\mu$ or $e$)}
\label{fig:piondecay}
\end{figure}\\
Where the up and anti-down quark and the lepton and corresponding lepton neutrino are on the right.  \\
\\
The formula given to describe the branching ratio of this decay is given by\cite{Griffiths}: 
\bea 
\Ga &=& \frac{S|\textbf{p}|}{8\pi\hbar m_1^2c} |\mathcal{M}|^2
\eea 
Where $|\textbf{p}|$ is the outgoing momentum, $S$ is the product of statistical factors (in our case it will be equal to 1), $m_1$ is the mass of the pion, and $\mathcal{M}$ is the Feynman amplitude. In our notation, we will use $\hbar= c=1$ and  so our expression becomes: 
\bea 
\Ga &=& \frac{S|\textbf{p}|}{8\pi m_1^2} |\mathcal{M}|^2
\eea 
The next step is to determine $\mathcal{M}$. In the charged current decay, we have a certain notation for the vertices and propagators. The Feynman rules are:
\begin{enumerate}
    \item For each vertex add a factor of $\frac{-ig_w}{2\sqrt{2}}(\ga^\nu(1-\ga^5))$ where $g_w= \sqrt{4\pi\al_w}$
    \item For each propagator we add a factor of $\frac{-ig_{\mu\nu}-\frac{q_\mu q_\nu}{m^2}}{q^2-m^2}$ where m is the mass of the boson. In our case, $m_w \gg\ q$ so the expression simplifies to $\frac{ig_{\mu\nu}}{m_w^2}$
\end{enumerate}
From these rules and Figure 1 we are able to calculate the value of $\mathcal{M}$ 
\bea
-i\mathcal{M}&=&\[\ou(3)\(\frac{-ig_w}{2\sqrt{2}}(\ga^\nu(1-\ga^5)\)v(2)\] \[\frac{ig_{\mu\nu}}{m_w^2} \]\[\frac{-ig_w}{2\sqrt{2}}F^\mu\]
\eea
Where $F^\mu$ is the form factor of the coupling of the pion to the $W$ boson. $F^\mu$ has the form of $f_\pi p^\mu$. 
\bea
\mathcal{M}&=&\frac{g_w^2}{8 m_w^2}\[\ou(3)(\ga^\mu(1-\ga^5))v(2)\]F^\mu
\eea
In order to square the amplitude we do the following: 
\bea
|\mathcal{M}^2|&=&\(\frac{g_w^2}{8 m_w^2}f_\pi\)^2Tr[(\ou(3)(\ga^\mu(1-\ga^5))v(2)p_\mu(\ov(2)(\ga^\nu(1-\ga^5)u(3))]p_\nu\\
|\mathcal{M}^2|&=&\(\frac{g_w^2}{8m_w^2}f_\pi\)^2p_\mu p_\nu (8[p_3^\mu p_2^\nu + p_2^\mu p_3^\nu -(p_3 \cdot p_2)g^{\mu\nu}]+8i\epsilon^{\mu\la\nu\si}p_{3\la}p_{2\si}
\eea
Summing over the spins gives us: 
\bea 
|\mathcal{M}^2|&=&8\(\frac{g_w^2}{8m_w^2}f_\pi\)^2 \[2(p_1\cdot p_2)(p_1 \cdot p_3) -p^2(p_2 \cdot p_3)\] 
\eea 
Since $p_1 = p_2 + p_3$, we can simplify the equation further. For simplicity and consistency we will use the following notation: $p_1 = p_\pi , p_2=p_l, p_3=p_{\nu_l}$. The value of the 4-momentum squared is: 
\bea 
p_1 &=& (E, \vec{p}_1) \\
(p_1)^2 &=& E^2 - (\vec{p}_1)^2\nl 
(p_1)^2 &=& m_1^2
\eea 
Similarly:
$(p_\pi)^2 = m_\pi^2 ,(p_l)^2 = m_l^2 , (p_{\nu_l})^2 = (m_{\nu_l})^2 = 0$. Using this we can further simplify Equation 8:
\bea 
\frac{1}{2}[(m_\pi)^2 -(m_l)^2] &=& (p_2 \cdot p_3)\\
\frac{1}{2}[(m_\pi)^2 -(m_l)^2] &=& (p_1 \cdot p_3)\\
\frac{1}{2}[(m_\pi)^2 +(m_l)^2] &=& (p_1 \cdot p_2)
\eea 
Returning to Equation 8 we now have
\bea 
|\mathcal{M}^2|&=&8\(\frac{g_w^2}{8m_w^2}f_\pi\)^2\[\frac{1}{2}(m_l)^2((m_\pi)^2- (m_l)^2)\]
\eea 
With the Feynman amplitude, we are able to complete the expression for the branching ratio of a pion. Returning to equation (3)
\bea 
\Ga &=& \frac{S|\textbf{p}|}{8\pi m_1^2} |\mathcal{M}|^2\\
\Rightarrow \Ga &=& \frac{S|\textbf{p}|}{\pi m_\pi^2}\(\frac{g_w^2}{8m_w^2}f_\pi\)^2\[\frac{1}{2}(m_l)^2((m_\pi)^2 -(m_l)^2)\]
\eea 
 We then need to find the value of $|\textbf{p}|$ which is simply the momentum of each outgoing particle. 
\bea
|\textbf{p}| &=& \frac{\sqrt{m_1^4+m_2^4-m_1^2m_2^2}}{2m_1}
\eea
Because the neutrino is massless,
\bea
|\textbf{p}| &=& \frac{(m_\pi^2-m_l^2)}{2m_\pi}
\eea
Combining with Equation 14: 
\bea 
\Ga &=& \frac{|\textbf{p}|}{2\pi m_1^2}\(\frac{g_w^2}{8m_w^2}f_\pi\)^2\[(m_l)^2((m_\pi)^2 -(m_l)^2)\]\\
\Rightarrow \Ga &=& \frac{1}{4\pi}\(\frac{g_w^2f_\pi}{8m_w^2}\)^2\(\frac{1}{m_\pi^3}\)m_l^2(m_\pi^2-m_l^2)^2
\eea
Because $\frac{g_w^2}{8m_w^2}= \frac{G_F}{\sqrt{2}}$ we are able to write:
\bea
\Ga &=& \frac{1}{8\pi}\(G_F f_\pi\)^2\(\frac{1}{m_\pi^3}\)m_l^2(m_\pi^2-m_l^2)^2
\eea 
Expanding on this idea, we are able to graph $\frac{\Ga_l}{\Ga_\pi}$ by: 
\bea
\Ga &=& \frac{1}{8\pi}(G_F f_\pi)^2m_\pi^3\(\frac{m_l}{m_\pi}\)^2\(1-\(\frac{m_l}{m_\pi}\)^2\)^2
\eea
In order to find $\frac{\Ga_{l}}{\Ga_\pi}$ we need to divide the above expression by $\frac{1}{\tau_\pi}$ but normalize it with $\hbar =1$ using Natural Units so our final expression is: 
\bea
\frac{\Ga_l}{\Ga_\pi}&=& \frac{1}{8\pi}(G_F f_\pi)^2m_\pi^3\(\frac{\tau_\pi}{\hbar}\)\(\frac{m_l}{m_\pi}\)^2\(1-\(\frac{m_l}{m_\pi}\)^2\)^2
\eea 
\begin{figure}[htbp!]
\centering
\includegraphics[scale=.9]{brpionmumu.png}
\caption{Graph of $\frac{\Ga_{m_e}}{\Ga_\pi}$ and the plots of $\frac{m_e}{m_\pi}$ and the value of $Br(\pi^{-} \rightarrow e^{-} + \nu_e)$}
\label{fig:BRofme}
\end{figure}
\begin{figure}[htbp!]
\centering
\includegraphics[scale=.75]{brpionmu1}
\caption{Graph of $\frac{\Ga_{m_\mu}}{\Ga_\pi}$ and the plots of $\frac{m_\mu}{m_\pi}$ and the value of $Br(\pi^{-} \rightarrow \mu^{-} + \nu_\mu)$}
\label{fig:BRofmpi}
\end{figure}
The graph of the Equation 25 (with $m_l = m_e$) is found in Figure \ref{fig:BRofme}. and the graph of Equation 25 (with $m_l = m_\pi$) is found in Figure \ref{fig:BRofmpi}.
Using the lifetime and mass values from PDG \cite{Agashe:2014kda} we may compute ratios.
\begin{center}
 \begin{tabular}{||c c c c||} 
 \hline
 Observables & $e$ & $\mu$ & $\pi$  \\ [1.5ex] 
 \hline\hline
 $\tau$ (Lifetime) & $6.6 \times 10^{28}$ yr &$2.1969811(22) \times 10^{-6}$ s  & $2.6033(5) \times 10^{-8}$ s \\[1.5ex]
 \hline
 Mass(MeV) & $0.5109989461(31)$ & $105.6583745(24)$  & $139.57061(24)$ \\[1.5ex] 
 \hline
\end{tabular}
\end{center}
To calculate the ratio of the $\pi^{-} \rightarrow e^{-} + \nu_e$ and  $\pi^{-} \rightarrow \mu^{-} + \nu_\mu$ we do the following: 
\bea 
\frac{\Ga_e}{\Ga_\mu} &=& \frac{m_e^2(m_\pi^2-m_e^2)^2}{m_\mu^2(m_\pi^2-m_\mu^2)^2}\\
\frac{\Ga_e}{\Ga_\mu} &=& 1.28334(73) \times 10^{-4}
\eea 
This is an interesting observation because the value of $ \frac{\Ga_e}{\Ga_\mu}$ suggests that the probability of $\pi^{-} \rightarrow \mu^{-} + \nu_\mu$ is higher than $\pi^{-} \rightarrow e^{-} + \nu_e$. This is somewhat striking because the mass of a muon is greater than the mass of an electron, indicating that the pion does not decay into the lightest particle most frequently. 
\section{The Branching Ratio of the $B_s \rightarrow \mu^+\mu^-$ decay}
From \cite{CMS:2014xfa} the SM branching ratio value is $(3.66 \pm 0.23) \times 10 ^ {-9}$. An expression for the branching ratio for $B_s \rightarrow \mu^+\mu^-$ can be found in \cite{Dighe:2012df} including NP WC's. The expression for the branching ratio is seen below:
\begin{center}
\bea
\mathcal{B}(B_s\rightarrow \mu^+\mu^-) &=& \frac{G_F^2 \al_{em}^2 m_{B_s}^5 f_{B_s}^2 \tau_{B_s}}{64\pi^3}\sqrt{1-\frac{4m_\mu^2}{m_{B_s}^2}} \lb\(1-\frac{4m_\mu^2}{m_{B_s}^2}\) \right. \nl 
&& \bigg|\zeta\frac{C_S-C_S^{'}}{m_b+m_s}\bigg|^2+\bigg|\zeta\frac{C_P-C_P^{'}}{m_b+m_s}+\frac{2m_\mu}{m_{B_s}^2}[|V_{tb} V_{ts}^*|C_{10}+\zeta (C_A-C_A^{'})]\bigg|^2\bigg\}
\eea
\end{center}
Where $\zeta \equiv (\frac{g_{NP}^2}{\Lambda^2})(\frac{\sqrt{2}}{4G_F})(\frac{4\pi}{\alpha_{em}})$\\
\\
The four-fermi operator diagram of the decay can be seen below: 
\begin{figure}[h]
\centering
\includegraphics[scale=.9]{B_sdecay}
\caption{Diagram of $B_s \rightarrow \mu^+ \mu^-$ from four-fermi operators}
\end{figure}
Since we seek to find the constraints on the parameters $C_S, C_S^{'},C_P, C_P^{'},C_A,$ and $C_A^{'}$ we set the equation equal to the branching ratio given in \cite{flavio} and using the values for $C_{10},|V_{tb} V_{ts}^*|, g_{NP},$ and $\Lambda$ given in \cite{Dighe:2012df} thus we are able to find the constraints on the parameters. In order to solve for one parameter, we allow the other two to be equal to 0, this will simply our calculations. 
The values for the Branching ratio of $B_s\rightarrow \mu^+\mu^- $ decay is equal to $2.4_{-0.7}^{+0.9} \times 10^{-9}$ \cite{Agashe:2014kda}
\bea
C_S&=& \pm 1.6215i \times 10^{-4} \nl
&=& \pm 4.63981\times 10^{-5}\nn
\eea
\bea
C_P&=& 6.3832 \times 10^{-5}~ {\rm or}~ 4.1125\times 10^{-4}\nl
&=&-4.482\times 10^{-6}~ {\rm or}~ 4.7957\times 10^{-4}\nn
\eea
\bea
C_A&=& 2.0348\times 10^{-3}~ {\rm or}~ 1.31095\times 10^{-2}\nl
&=& -1.48237\times 10^{-4} ~ {\rm or}~ 1.5287\times 10^{-2}\nn
\eea
\textbf{Notes about the Wilson Coefficients:}\\
The first $C_S$ coefficient is equal to $\pm 1.6215i \times 10^{-4}$ which indicates that the branching ratio will receive a negative contribution from this coefficient when the branching ration is below the SM prediction. The $C_P$ has two values the first value is equal to $2.375\times 10^{-4}$ and the second value is equal to $2.375 \times 10^{-4}$ as well (We are able to calculate a simple average due to the small number of measurements).The last coefficient $C_{A}$ also has two values, the first one is equal to $7.572 \times 10^{-3}$ and the second value is equal to $7.569 \times 10^{-3}$. Both of these coefficients have two possible values because we take the  absolute values of the WC added to another factor. After finding the Wilson Coefficients, we use the Python package Flavio \cite{flavio} to compute the NP values of $\mathcal{B}(B_s\rightarrow \mu^+\mu^-)$. We again only apply one Wilson Coefficient at a time to find the NP branching ratio:
\begin{center}
 $C_S = \pm 1.6215i \times 10^{-4}$\\
 Flavio Prediction: BR = $3.610 \times 10^{-9}$\\
 $C_S = \pm 4.63981 \times 10^{-4}$\\
 Flavio Prediction: BR = $3.610 \times 10^{-9}$\\
\end{center}
\begin{center}
$C_{P} = 2.375 \times 10^{-4}$ \\
Flavio Prediction: $Br = 3.555 \times 10^{-9}$\\
\end{center}
\begin{center}
$C_{A} = 7.572 \times 10^{-3}$ \\
Flavio Prediction: $Br = 3.542 \times 10^{-9}$\\
$C_{A} = 7.569\times 10^{-3}$ \\
Flavio Prediction: $Br = 3.542 \times 10^{-9}$\\
\end{center}
\newpage
\section{Plots of $Br(WC)$}
\begin{figure}[htbp!]
\centering
\includegraphics[scale=.75]{B(WC)forB_smumu}
\caption{Graph of $Br(B_s\rightarrow \mu^+\mu^-)$ as functions of the $C_P$ and $C_S$ Wilson Coefficients}
\label{fig:Bsmumu1}
\centering
\includegraphics[scale=.75]{B(WC)forB_smumu2}
\caption{Graph of $Br(B_s\rightarrow \mu^+\mu^-)$ as functions of the $C_A$ Wilson Coefficient}
\label{fig:Bsmumu2}
\end{figure}
\begin{figure}[htbp!]
\centering
\includegraphics[scale=.75]{BrofB-Kmumu}
\caption{Graph of $Br(B\rightarrow K\mu^+\mu^-)$ as functions of the Wilson Coefficients}
\label{fig:Bkmumu}
\end{figure}
\begin{figure}
\centering
\includegraphics[scale=.75]{BrofB-Kstarmumu}
\caption{Graph of $Br(B\rightarrow K^*\mu^+\mu^-)$ as functions of the Wilson Coefficients}
\label{fig:bKstarmumu}
\end{figure}
Figures \ref{fig:Bsmumu1}, \ref{fig:Bsmumu2}, \ref{fig:Bkmumu}, and \ref{fig:bKstarmumu} give an indication of how the variation of a Wilson Coefficient could change the value of the Branching ratio. The variations follow a parabolic shape due to the branching ratio being dependent on the square of the WC.
\newpage
\section{Conclusion}
The work done with the Wilson Coefficients gives suggestions to the possibility of NP. In the future, the expectation is that work will be done using a chi squared minimization technique to help limit and constrain the WC. The chi squared technique would use the observables that have been measured experimentally from different experiments. In the future, we hope to provide a rigorous explanation that shows the existance of NP.  
\section{Acknowledgements}
I would like to thank the Department of Physics and Astronomy at Wayne State University for the opportunity to conduct this research.This project was supported through the Wayne State University REU program, under NSF grant phy-1460853. I would also like to thank Bhubanjyoti Bhattacharya for his guidance and suggestions and doctoral student Cody Grant who's advice and discussions were invaluable. Finally I would like to thank my parents Andrew and Liz Walz for giving me the love and support to pursue my interests and begin my career as a physicist.  
\newpage
\begin{thebibliography}{9}
\bibitem{Griffiths} 
Griffiths, D. (2014). Introduction to elementary particles. Weinheim: Wiley-VCH Verlag
%\cite{Agashe:2014kda}
\bibitem{Agashe:2014kda} 
  K.~A.~Olive {\it et al.} [Particle Data Group],
  %``Review of Particle Physics,''
  Chin.\ Phys.\ C {\bf 38}, 090001 (2014).
  doi:10.1088/1674-1137/38/9/090001
  %%CITATION = doi:10.1088/1674-1137/38/9/090001;%%
  %6385 citations counted in INSPIRE as of 03 Jul 2017
  %\cite{Dighe:2012df}
\bibitem{Dighe:2012df} 
  A.~Dighe and D.~Ghosh,
  %``How large can the branching ratio of $B_s \to \tau^+ \tau^-$ be ?,''
  Phys.\ Rev.\ D {\bf 86}, 054023 (2012)
  doi:10.1103/PhysRevD.86.054023
  [arXiv:1207.1324 [hep-ph]].
  %%CITATION = doi:10.1103/PhysRevD.86.054023;%%
  %20 citations counted in INSPIRE as of 06 Jul 2017
  %\cite{Dighe:2012df}
 \bibitem{flavio}
    David Straub, Peter Stangl, ChristophNiehoff, Ece Gurler, wzeren, Jacky Kumar, … Frederik Beaujean. (2017, June 29). flav-io/flavio v0.22.1. Zenodo. http://doi.org/10.5281/zenodo.821015
%\cite{Bhattacharya:2016mcc}
\bibitem{Bhattacharya:2016mcc} 
  B.~Bhattacharya, A.~Datta, J.~P.~Guévin, D.~London and R.~Watanabe,
  %``Simultaneous Explanation of the $R_K$ and $R_{D^{(*)}}$ Puzzles: a Model Analysis,''
  JHEP {\bf 1701}, 015 (2017)
  doi:10.1007/JHEP01(2017)015
  [arXiv:1609.09078 [hep-ph]].
  %%CITATION = doi:10.1007/JHEP01(2017)015;%%
%\cite{Aaij:2014ora}
\bibitem{Aaij:2014ora} 
  R.~Aaij {\it et al.} [LHCb Collaboration],
  %``Test of lepton universality using $B^{+}\rightarrow K^{+}\ell^{+}\ell^{-}$ decays,''
  Phys.\ Rev.\ Lett.\  {\bf 113}, 151601 (2014)
  doi:10.1103/PhysRevLett.113.151601
  [arXiv:1406.6482 [hep-ex]].
  %%CITATION = doi:10.1103/PhysRevLett.113.151601;%%
  %416 citations counted in INSPIRE as of 02 Aug 2017
  %\cite{Bordone:2016gaq}
\bibitem{Bordone:2016gaq} 
  M.~Bordone, G.~Isidori and A.~Pattori,
  %``On the Standard Model predictions for $R_K$ and $R_{K^*}$,''
  Eur.\ Phys.\ J.\ C {\bf 76}, no. 8, 440 (2016)
  doi:10.1140/epjc/s10052-016-4274-7
  [arXiv:1605.07633 [hep-ph]].
  %%CITATION = doi:10.1140/epjc/s10052-016-4274-7;%%
  %70 citations counted in INSPIRE as of 02 Aug 2017
  %\cite{Lees:2013uzd}
\bibitem{Lees:2013uzd} 
  J.~P.~Lees {\it et al.} [BaBar Collaboration],
  %``Measurement of an Excess of $\bar{B} \to D^{(*)}\tau^- \bar{\nu}_\tau$ Decays and Implications for Charged Higgs Bosons,''
  Phys.\ Rev.\ D {\bf 88}, no. 7, 072012 (2013)
  doi:10.1103/PhysRevD.88.072012
  [arXiv:1303.0571 [hep-ex]].
  %%CITATION = doi:10.1103/PhysRevD.88.072012;%%
  %250 citations counted in INSPIRE as of 02 Aug 2017
  %\cite{Huschle:2015rga}
\bibitem{Huschle:2015rga} 
  M.~Huschle {\it et al.} [Belle Collaboration],
  %``Measurement of the branching ratio of $\bar{B} \to D^{(\ast)} \tau^- \bar{\nu}_\tau$ relative to $\bar{B} \to D^{(\ast)} \ell^- \bar{\nu}_\ell$ decays with hadronic tagging at Belle,''
  Phys.\ Rev.\ D {\bf 92}, no. 7, 072014 (2015)
  doi:10.1103/PhysRevD.92.072014
  [arXiv:1507.03233 [hep-ex]].
  %%CITATION = doi:10.1103/PhysRevD.92.072014;%%
  %214 citations counted in INSPIRE as of 02 Aug 2017
  %\cite{Aaij:2015yra}
\bibitem{Aaij:2015yra} 
  R.~Aaij {\it et al.} [LHCb Collaboration],
  %``Measurement of the ratio of branching fractions $\mathcal{B}(\bar{B}^0 \to D^{*+}\tau^{-}\bar{\nu}_{\tau})/\mathcal{B}(\bar{B}^0 \to D^{*+}\mu^{-}\bar{\nu}_{\mu})$,''
  Phys.\ Rev.\ Lett.\  {\bf 115}, no. 11, 111803 (2015)
  Erratum: [Phys.\ Rev.\ Lett.\  {\bf 115}, no. 15, 159901 (2015)]
  doi:10.1103/PhysRevLett.115.159901, 10.1103/PhysRevLett.115.111803
  [arXiv:1506.08614 [hep-ex]].
  %%CITATION = doi:10.1103/PhysRevLett.115.159901, 10.1103/PhysRevLett.115.111803;%%
  %255 citations counted in INSPIRE as of 02 Aug 2017
  %\cite{Descotes-Genon:2015uva}
\bibitem{Descotes-Genon:2015uva} 
  S.~Descotes-Genon, L.~Hofer, J.~Matias and J.~Virto,
  %``Global analysis of $b\to s\ell\ell$ anomalies,''
  JHEP {\bf 1606}, 092 (2016)
  doi:10.1007/JHEP06(2016)092
  [arXiv:1510.04239 [hep-ph]].
  %%CITATION = doi:10.1007/JHEP06(2016)092;%%
  %190 citations counted in INSPIRE as of 04 Aug 2017
  %\cite{CMS:2014xfa}
\bibitem{CMS:2014xfa} 
  V.~Khachatryan {\it et al.} [CMS and LHCb Collaborations],
  %``Observation of the rare $B^0_s\to\mu^+\mu^-$ decay from the combined analysis of CMS and LHCb data,''
  Nature {\bf 522}, 68 (2015)
  doi:10.1038/nature14474
  [arXiv:1411.4413 [hep-ex]].
  %%CITATION = doi:10.1038/nature14474;%%
  %312 citations counted in INSPIRE as of 07 Aug 2017
\end{thebibliography}
\newpage
\section{Appendix}
\begin{center}
\underline{\textbf{\Large Practice with Spinors}}
\end{center}
Notation:
\begin{enumerate} \itemsep=-15pt
\item $S= \ou u$\\
\item $P=\ou \ga^5 u$ \\
\item $V^\mu = \ou \ga^\mu u$ \\
\item $A^\mu = \ou \ga^\mu \ga^5 u $ \\
\item $T^{\mu\nu} = \ou \si^{\mu\nu} u$
\end{enumerate}
\bigskip
It was necessary to practice with spinor notation and the different mathematical techniques before starting this project. This appendix summarizes what I was able to do in that regard. 
\begin{enumerate}

\item$(\ou_1\ga^\mu u_2)^*$ ~=~?

Note: $(\ga^0)^\dag~=~\ga^0$ and $(\ga^\mu)^\dag~=~ \ga^0\ga^\mu\ga^0$

$(\ou_1\ga^\mu u_2)$ is a $1\times1$ matrix. Therefore, its complex conjugate is the same as its Hermitian conjugate, i.e. if we call
$V^\mu = (\ou_1\ga^\mu u_2)$, then $(V^\mu)^* = (V^\mu)^\dag$. We can then express this quantity as follows:
% This is the first equation/set of equations
\bea
V^\mu &=& \ou_1\ga^\mu u_2 ~,~~ \\
\Rightarrow (V^\mu)^* &=& (V^\mu)^\dag ~,~~ \nl
&=& (\ou_1\ga^\mu u_2)^\dag ~,~~ \nl
&=& ((u_1)^\dag\ga^0\ga^\mu u_2)^\dag ~~~{\rm using}~(A\ldots Z)^\dag = Z^\dag\ldots A^\dag ~,~~ \nl
&=& (u_2^\dag)(\ga^\mu)^\dag(\ga^0)^\dag(u_1)\nl
&=& (u_2^\dag)\ga^0\ga^\mu\ga^0\ga^0(u_1)\nl
&=& (u_2^\dag)\ga^0\ga^\mu(u_1)\nl
&=& \ou_2\ga^\mu(u_1)
\eea
Therefore $(\ou_1\ga^\mu u_2)^*~=~\ou_2\ga^\mu u_1 $. To solve for $|V^\mu|^2$ we simply use $|V^\mu|^2 ~=~ \Tr[\ou_1 \ga^\mu u_2 \ou_2 \ga^\nu u_1]$.

Note: ~$\Tr[\ga^\mu\ga^\nu]~=~4 g^{\mu\nu}$, $\Tr[\ga^\mu \ga^\nu \ga^\la \ga^\si]= 4(g^{\mu\nu}g^{\la\si} - g^{\mu\la} g^{\nu\si} + g^{\mu\si}g^{\nu\la})$, The trace over the product of an odd number of gamma matrices is zero.

\bea
|V^{\mu}|^2 &=& \Tr[\ou_1 \ga^\mu u_2 \ou_2 \ga^\nu u_1] \\
&=& \Tr[\ou_1 \ga^\mu (\slashed{p}_2+m) \ga^\nu u_1]\nl
&=& \Tr[u_1\ou_1 \ga^\mu (\slashed{p}_2+m) \ga^\nu]\nl
&=& \Tr[(\slashed{p}_1+m) \ga^\mu (\slashed{p}_2+m) \ga^\nu]\nl
&=& \Tr[\slashed{p}_1\ga^\mu \slashed{p}_2 \ga^\nu] + m[\Tr(\ga^\mu \slashed{p}_1\ga^\nu) + \Tr(\ga^\mu \ga^\nu \slashed{p}_2)] + m^2\Tr[\ga^\mu \ga^\nu]\nl
&=& \Tr[\slashed{p}_1\ga^\mu \slashed{p}_2 \ga^\nu] + m^2\Tr[\ga^\mu \ga^\nu]\nl
&=& \Tr[(p_1)_\la\ga^\la \ga^\mu (p_2)_\si\ga^\si \ga^\nu  ] + 4m^2g^{\mu\nu}\nl
&=& (p_1)_\la (p_2)_\si \Tr[\ga^\la \ga^\mu\ga^\si \ga^\nu  ] + 4m^2g^{\mu\nu}\nl
&=& (p_1)_\la(p_2)_\si 4(g^{\mu\nu}g^{\la\si} - g^{\mu\la} g^{\nu\si} + g^{\mu\si}g^{\nu\la}) + 4m^2g^{\mu\nu}\nl
&=& 4[p_1^\mu p_2^\nu - g^{\mu\nu}(p_1 \cdot p_2) + p_2^\mu p_1^\nu] + 4m^2g^{\mu\nu} 
\eea

\item $(\ou_1\ga^\mu \ga^5 u_2)^*$ is also a $1\times1$ Matrix so the same reasoning applies as above in 1. Note: $(\ga^5)^\dag = \ga^5$
We define: $A^\mu$ as $\ou_1\ga^\mu\ga^5 u_2$ thus:
% This is the second equation/set of equations
\bea
(A^\mu)^* &=& (A^\mu)^\dag \\
&=&(\ou_1\ga^\mu\ga^5 u_2)^\dag \nl
&=& ((u_1)^\dag\ga^0\ga^\mu\ga^5 u_2)^\dag \nl
&=& (u_2^\dag)(\ga^5)^\dag(\ga^\mu)^\dag(\ga^0)^\dag(u_1) \nl
&=& (u_2^\dag)\ga^5\ga^0\ga^\mu\ga^0\ga^0 u_1 \nl
&=& (u_2^\dag)\ga^5\ga^0\ga^\mu(1) u_1 \nl
&=& -(u_2^\dag)\ga^0\ga^5\ga^\mu u_1 \nl
&=& -\ou_2\ga^5\ga^\mu u_1 \nl
&=& \ou_2\ga^\mu\ga^5 u_1
\eea
%
Therefore $(\ou_1\ga^\mu \ga^5 u_2)^*~=~\ou_2 \ga^\mu \ga^5 u_1$

We also are able to calculate $|A^\mu|^2$
\bea
|A^{\mu}|^2 &=& \Tr[(\ou_1\ga^\mu \ga^5 u_2)(\ou_2 \ga^\nu \ga^5 u_1)] \\
&=& \Tr[\ou_1\ga^\mu \ga^5 (\slashed{p}_2+m) \ga^\nu \ga^5 u_1]  \nl
&=&\Tr[u_1\ou_1\ga^\mu \ga^5 (\slashed{p}_2+m) \ga^\nu \ga^5 ]\nl
&=&\Tr[(\slashed{p}_1+m)\ga^\mu \ga^5 (\slashed{p}_2+m) \ga^\nu \ga^5 ]\nl
&=& \Tr[\slashed{p}_1\ga^\mu\ga^5 \slashed{p}_2\ga^\nu \ga^5 + m( \slashed{p}_1\ga^\mu\ga^5\ga^\nu \ga^5 +\ga^\mu\ga^5 \slashed{p}_2\ga^\nu \ga^5) + m^2(\ga^\mu\ga^5\ga^\nu \ga^5)] \nl
&=&\Tr[\slashed{p}_1\ga^\mu\ga^5 \slashed{p}_2\ga^\nu \ga^5 + m( \slashed{p}_1\ga^\mu(-\ga^5 \ga^5)\ga^\nu +\slashed{p}_2\ga^\mu(-\ga^5\ga^5) \ga^\nu ) + m^2(\ga^\mu\ga^5\ga^\nu \ga^5)] \nl
&=&\Tr[\slashed{p}_1\ga^\mu\ga^5 \slashed{p}_2\ga^\nu \ga^5 - m( \slashed{p}_1\ga^\mu\ga^\nu +\slashed{p}_2\ga^\mu\ga^\nu ) + m^2(\ga^\mu\ga^5\ga^\nu \ga^5)] \nl
&=&\Tr[\slashed{p}_1\ga^\mu\ga^5 \slashed{p}_2\ga^\nu \ga^5] -mTr[ \slashed{p}_1\ga^\mu\ga^\nu]-mTr[ \slashed{p}_2\ga^\mu\ga^\nu]+ m^2(\ga^\mu\ga^5\ga^\nu \ga^5)] \nl
&=&\Tr[\slashed{p}_1\ga^\mu\ga^5 \slashed{p}_2\ga^\nu \ga^5 + m^2(\ga^\mu\ga^5\ga^\nu \ga^5)] \nl
&=& \Tr[(p_1)_\la\ga^\la\ga^\mu\ga^5 (p_2)_\si\ga^\si\ga^\nu \ga^5] + m^2\Tr[\ga^\mu\ga^5\ga^\nu \ga^5]\nl
&=& (p_1)_\la(p_2)_\si \Tr[\ga^\la\ga^\mu\ga^5 \ga^\si\ga^\nu \ga^5] - m^2\Tr[\ga^\mu\ga^5\ga^5 \ga^\nu]\nl
&=& (p_1)_\la(p_2)_\si \Tr[\ga^\la\ga^\mu\ga^5\ga^5 \ga^\si\ga^\nu] - m^2\Tr[\ga^\mu\ga^\nu]\nl
&=& (p_1)_\la(p_2)_\si \Tr[\ga^\la\ga^\mu\ga^\si\ga^\nu] - m^2(g^{\mu\nu})\nl
&=& (p_1)_\la(p_2)_\si 4(g^{\mu\nu}g^{\la\si} - g^{\mu\la} g^{\nu\si} + g^{\mu\si}g^{\nu\la}) - 4m^2g^{\mu\nu}\nl
&=& 4[p_1^\mu p_2^\nu - g^{\mu\nu}(p_1 \cdot p_2) + p_2^\mu p_1^\nu] - 4m^2g^{\mu\nu} 
\eea

\item $ (\ou_1 u_2)^* ~=~ ?$
We let $S ~=~ \ou_1 u_2$
\bea
(S)^* &=& (S)^\dag \\
&=& (\ou_1 u_2)^\dag ~,~~ \nl
&=& ((u_1)^\dag\ga^0 u_2)^\dag \nl
&=& (u_2)^\dag(\ga^0)^\dag(u_1)\nl
&=& (u_2)^\dag\ga^0(u_1)\nl
&=& \ou_2(u_1)
\eea
Therefore$(\ou_1 u_2)^* ~=~ \ou_2 u_1$.
In order to find $|S|^2$ we simply do the following:
\bea
|S|^2&=& \Tr[\ou_1 u_2 \ou_2 u_1]  \\
&=& \Tr[(\slashed{p}_1+m)(\slashed{p}_2+m)] \nl
&=& \Tr[\slashed{p}_1\slashed{p}_2+m(\slashed{p}_1+\slashed{p}_2)+m^2]\nl
&=& \Tr[\slashed{p}_1\slashed{p}_2]+\Tr[m(\slashed{p}_1+\slashed{p}_2)]+\Tr[m^2] \nl
&=& \Tr[\slashed{p}_1\slashed{p}_2]+m(\Tr[\slashed{p}_1]+\Tr[\slashed{p}_2])+m^2\Tr[1] \nl
&=& \Tr[\slashed{p}_1\slashed{p}_2]+4m^2 \nl
&=& 4(p_1 \cdot p_2)+4m^2
\eea

\item  By the same reasoning as shown above it can be shown that $(\ou_1 \ga^5 u_2)^* ~=~ \ou_2 \ga^5 u_1$ \\
If we let $P ~=~ \ou_1\ga^5 u_2$ then:
\bea
(P)^* &=& (P)^\dag \\
&=& (\ou_1\ga^5 u_2)^\dag ~,~~ \nl
&=& ((u_1)^\dag\ga^0\ga^5 u_2)^\dag \nl
&=& (u_2)^\dag(\ga^5)^\dag (\ga^0)^\dag(u_1)\nl
&=& (u_2)^\dag(\ga^5) \ga^0(u_1)\nl
&=& -(u_2)^\dag \ga^0\ga^5(u_1)\nl
&=& -\ou_2\ga^5(u_1)
\eea
Therefore  $(\ou_1 \ga^5 u_2)^* ~=~ -\ou_2 \ga^5 u_1$

In order to square $P$ we do the following:
\bea
|P|^2&=& \Tr[\ou_1 \ga^5 u_2(-\ou_2 \ga^5 u_1)] \\
&=& \Tr[u_1 \ou_1 \ga^5(-\slashed{p}_2 - m) \ga^5] \nl
&=& \Tr[(\slashed{p}_1+m) \ga^5 (-\slashed{p}_2-m) \ga^5] \nl
&=& \Tr[((p_1)_\mu \ga^\mu+m) \ga^5 ((-p_2)_\nu\ga^\nu-m)  \ga^5] \nl
&=& \Tr[((p_1)_\mu \ga^\mu\ga^5+m\ga^5) ((-p_2)_\nu\ga^\nu\ga^5-m\ga^5)] \nl
&=& \Tr[(p_1)_\mu \ga^\mu\ga^5(-p_2)_\nu\ga^\nu\ga^5+ m\ga^5(-p_2)_\nu\ga^\nu\ga^5 -m\ga^5(p_1)_\mu \ga^\mu\ga^5 - \ga^5\ga^5m^2)] \nl
&=& \Tr[(p_1)_\mu \ga^\mu\ga^5(-p_2)_\nu\ga^\nu\ga^5] + Tr[ m\ga^5(-\ga^5)(-p_2)_\nu\ga^\nu] - Tr[m\ga^5(-\ga^5)(p_1)_\mu \ga^\mu] - Tr[m^2] \nl
&=& \Tr[(p_1)_\mu \ga^\mu\ga^5(-p_2)_\nu\ga^\nu\ga^5] -Tr[ m(-p_2)_\nu\ga^\nu] + Tr[m(p_1)_\mu \ga^\mu] - 4m^2 \nl
&=& \Tr[(p_1)_\mu \ga^\mu\ga^5(-p_2)_\nu\ga^\nu\ga^5] - 4m^2 \nl
&=& (p_1)_\mu(-p_2)_\nu \Tr[\ga^\mu\ga^5\ga^\nu\ga^5]- 4m^2 \nl
&=& (p_1)_\mu(-p_2)_\nu \Tr[\ga^\mu\ga^5(-\ga^5\ga^\nu)]- 4m^2\nl
&=& (p_1)_\mu(-p_2)_\nu (-\Tr[\ga^\mu\ga^\nu])- 4m^2\nl
&=& (p_1)_\mu(-p_2)_\nu (-4g^{\mu\nu})- 4m^2\nl
&=& 4(p_1)(p_2)- 4m^2
\eea
\end{enumerate}
\end{document}
